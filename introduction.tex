
\section{Introduction}

%
\begin{comment}
remove all soft O notation and always write logs explicitly? 
\end{comment}
{}Let $\mathbf{F}\in\mathbb{K}\left[\left[x\right]\right]^{m\times n}$
be a matrix of power series over a field $\mathbb{K}$ with $m\le n$.
Given a non-negative integer $\sigma$, we say a vector $\mathbf{p}\in\mathbb{K}\left[x\right]^{n\times1}$
of polynomials gives an \emph{order} $\sigma$ approximation of $\mathbf{F}$,
or $\mathbf{p}$ has order $\left(\mathbf{F},\sigma\right)$, if \[
\mathbf{F}\cdot\mathbf{p}\equiv\mathbf{0}\mod x^{\sigma},\]
 that is, the first $\sigma$ terms of $\mathbf{F}\cdot\mathbf{p}$
are zero. Historically such problems date back to their use in Hermite's
proof of the transcendence of $e$ in 1873. In 1893 Pad�, a student
of Hermite, formalized the concepts introduced by Hermite and defined
what is now known as Hermite-Pad� approximants (where $m=1$), Pad�
approximants (where $m=1,n=2$) and simultaneous Pad� approximants
(where $\mathbf{F}$ has a special structure). Such rational approximations
also specified degree constraints on the polynomials $\mathbf{p}$
and had their order conditions related to these degree constraints.
%Pad\'e also introduced the notion of a Pad\'e
%table for the case $m=1,n=2$ which described all Pad\'e approximants 
%in terms of $2$ nearby approximants. 
Additional order problems include vector and matrix versions of rational
approximation, partial realizations of matrix sequences and vector
rational reconstruction just to name a few. As an example, the factorization
of differential operators algorithm of \citet{vanHoeij} makes use
of vector Hermite-Pad� approximation to reconstruct differential factorizations
over rational functions from factorizations of differential operators
over power series domains.

The set of all such order $\left(\mathbf{F},\sigma\right)$ approximations
forms a module over $\mathbb{K}\left[x\right]$. An {\em order basis}
- or minimal approximant basis or $\sigma$-basis - is a basis of
this module having a type of minimal degree property (called reduced
order basis in \citep{BL1997}). The minimal degree property parameterizes
solutions to an order problem by the degrees of the columns of the
order basis. In the case of rational approximation, order bases can
be viewed as a natural generalization of the Pad� table of a power
series \citep{gravesmorris} since they are able to describe {\em
all} solutions to such problems given particular degree bounds \citep{BL1997}.
Order bases are used in such diverse applications as the inversion
of structured matrices \citep{La92}, normal forms of matrix polynomials
\citep{BLV:1999,BLV:jsc06}, and other important problems in matrix
polynomial arithmetic including matrix inversion, determinant and
nullspace computation \citep{Giorgi2003,storjohann-villard:2005}.
In our case we also allow the minimal degree property to include a
shift $\vec{s}$. Such a shift is important for matrix normal form
problems \citep{BLV:1999,BLV:jsc06}.

In this paper we focus on the efficient computation of order basis.
Algorithms for fast computation of order basis include that of \citet{BeLa94}
which converts the matrix problem into a vector problem of higher
order (which they called the Power Hermite-Pad� problem). Their divide
and conquer algorithm has complexity of $O^{\sim}(n^{2}m\sigma+nm^{2}\sigma)$
field operations. As usual, the soft-$O$ notation $O^{\sim}$ is
simply Big-$O$ with log factors omitted. By working more directly
on the input $m\times n$ input matrix, \citet{Giorgi2003} give a
divide and conquer method with cost $O\left(\MM\left(n,\sigma\right)\log\sigma\right)\subset O^{\sim}\left(\MM\left(n,\sigma\right)\right)$
arithmetic operations, where $\MM\left(n,\sigma\right)\in O^{\sim}\left(n^{\omega}\sigma\right)$
denotes the cost of multiplying two polynomial matrices with dimension
$n$ and degree $\sigma$. Their method is very efficient if $m$
is close to the size of $n$ but can be improved if $m$ is small.

In a novel construction, \citet{Storjohann:2006} effectively reverses
the approach of Beckermann and Labahn. Namely, rather than convert
a high dimension matrix order problem into a lower dimension vector
problem of higher order, Storjohann converts a low dimension problem
to a high dimension problem with lower order. For example, computing
an order basis for a $1\times n$ vector input $\mathbf{f}$ and order
$\sigma$ can be converted to a problem of order basis computation
with an $O\left(n\right)\times O\left(n\right)$ input matrix and
an order $O\left(\left\lceil \sigma/n\right\rceil \right)$. Combining
this conversion with the method of Giorgi et al. can then be used
effectively for problems with small row dimensions to achieve a cost
of $O^{\sim}\left(\MM\left(n,\left\lceil m\sigma/n\right\rceil \right)\right)$.

However, while order bases of the original problem can have degree
up to $\sigma$, the nature of Storjohann's conversion limits the
degree of an order basis of the converted problem to $O\left(\left\lceil m\sigma/n\right\rceil \right)$
in order to be computationally efficient. In other words, this approach
does not in general compute a complete order basis. Rather, in order
to achieve efficiency, it only computes a partial order basis containing
basis elements with degrees within $O\left(\left\lceil m\sigma/n\right\rceil \right)$,
referred to by Storjohann as a {\em minbasis}. Fast methods for
computing a minbasis are particularly useful for certain problems,
for example, in the case of inversion of structured block matrices
where one needs only precisely the minbasis \citep{La92}. However,
in other applications, such as those arising in matrix polynomial
arithmetic, one needs a complete basis which specifies all solutions
of a given order, not just those within a particular degree bound.

In this paper we present two algorithms which compute an entire order
basis with a cost of $O^{\sim}\left(\MM\left(n,\left\lceil m\sigma/n\right\rceil \right)\right)$
field operations. The two algorithms differ depending on the nature
of the degree shift required for the reduced order basis. In the first
case we use a transformation that can be considered as an extension
of Storjohann's transformation. This new transformation provides a
way to extend the results from one Storjohann's transformed problem
to another Storjohann's transformed problem of a higher degree. This
enables us to use an idea from the null space basis algorithm of Storjohann
and Villard \citep{storjohann-villard:2005} in order to achieve efficient
computation. At each iteration, basis elements within a specified
degree bound are computed via a Storjohann transformed problem. Then
the partial result is used to simplify the next Storjohann transformed
problem of a higher degree, allowing basis elements within a higher
degree bound to be computed efficiently. This is repeated until all
basis elements are computed.

In order to compute an order basis efficiently, the first algorithm
requires that the degree shifts are balanced. In the case where the
shift is not balanced, the row degrees of the basis can also become
unbalanced in addition to the unbalanced column degrees. We give a
second algorithm that balances the high degree rows and uses $O^{\sim}(\MM(n,\lceil m\sigma/n\rceil))$
field operations when the shift $\vec{s}$ is unbalanced but satisfies
the condition $\sum_{i=1}^{n}(\max(\vec{s})-\vec{s}_{i})\le m\sigma$.
This condition essentially allows us to locate the high degree unbalanced
rows that need to be balanced. %
\begin{comment}
In fact, every problem with any unbalanced shift can be reformulated
to a problem with a shift satisfying this condition if the degrees
of the resulting order basis is known. 
\end{comment}
{} The algorithm converts a problem of unbalanced shift to one with
balanced shift, based on a second idea from \citep{Storjohann:2006}.
Then the first algorithm is used to efficiently compute the elements
of an order basis whose shifted degrees exceed a specified parameter.
The problem is then reduced to one where we remove the computed elements.
This results in a new problem with smaller dimension and higher degree.
The same process is repeated again on this new problem in order to
compute the elements with the next highest shifted degrees.

The remaining paper is structured as follows. Basic definitions and
properties of order bases are given in the next section. \prettyref{sec:transform}
provides an extension to Storjohann's transformation to allow higher
degree basis elements to be computed. Based on this new transformation,
\prettyref{sec:Order-Basis-Computation} establishes a link between
two Storjohann transformed problems of different degrees, from which
an recursive method and then an iterative algorithm are derived. The
time complexity is analyzed in the next section. After this, \prettyref{sec:Unbalanced-Shift}
describes an algorithm which handles problems with a type of unbalanced
shift. This is followed by a conclusion along with a description for
topics for future research. 
