
\section{Computation of Order Bases}

\label{sec:Order-Basis-Computation}

In this section, we establish a link between two different Storjohann
transformed problems by dividing the transformed problem from the
previous section into two subproblems and then simplifying the second
subproblem. This leads to a recursive method for computing order bases.
We also present an equivalent, iterative method for computing order bases.
The iterative approach is usually more efficient in practice, as it uses 
just one iteration in the generic case.

\subsection{\label{sub:Dividing-to-Subproblems}Dividing into Subproblems }

In \prettyref{sec:transform} we have shown that the problem of computing a 
$\left(\mathbf{F},\sigma,\vec{s}\right)$-basis can be converted to the problem 
of computing a $(\mathbf{F}',\vec{\omega},\vec{s'})$-basis and, more generally, 
that the computation of a $(\bar{\mathbf{F}}^{\left(i\right)},2\delta^{\left(i\right)},\vec{s}^{\left(i\right)})$-basis,
a Storjohann transformed problem with degree parameter $\delta^{\left(i\right)}$,
can be converted to the problem of computing a $(\mathbf{F}'^{\left(i\right)},\vec{\omega}^{\left(i\right)},\vec{s}^{\left(i-1\right)})$-basis.
We now consider dividing the new converted problem into two subproblems.

The first subproblem is to compute a $(\mathbf{F}'^{\left(i\right)},2\delta^{\left(i-1\right)},\vec{s}^{\left(i-1\right)})$-basis
or equivalently a $(\bar{\mathbf{F}}^{\left(i-1\right)},2\delta^{\left(i-1\right)},\vec{s}^{\left(i-1\right)})$-basis
$\bar{\mathbf{P}}^{\left(i-1\right)}$, a Storjohann
transformed problem with degree parameter $\delta^{\left(i-1\right)}$.
The second subproblem is computing a $(\mathbf{F}'^{\left(i\right)}\bar{\mathbf{P}}^{\left(i-1\right)},\vec{\omega}^{\left(i\right)},\vec{t}^{\left(i-1\right)})$-basis
$\bar{\mathbf{Q}}^{\left(i\right)}$ using the residual $\mathbf{F}'^{\left(i\right)}\bar{\mathbf{P}}^{\left(i-1\right)}$
from the first subproblem along with a degree shift $\vec{t}^{\left(i-1\right)}=\deg_{\vec{s}^{\left(i-1\right)}}\bar{\mathbf{P}}^{\left(i-1\right)}$.
From Theorem 5.1 in \citep{BL1997} we then know that the product
$\bar{\mathbf{P}}^{\left(i-1\right)}\bar{\mathbf{Q}}^{\left(i\right)}$
is a $(\mathbf{F}'^{\left(i\right)},\vec{\omega}^{\left(i\right)},\vec{s}^{\left(i-1\right)})$-basis
and $\deg_{\vec{s}^{\left(i-1\right)}}\bar{\mathbf{P}}^{\left(i-1\right)}\bar{\mathbf{Q}}^{\left(i\right)}=\deg_{\vec{t}^{\left(i-1\right)}}\bar{\mathbf{Q}}^{\left(i\right)}$. 
\begin{exmp}
\label{exm:subproblems} Let us continue with 
\prettyref{exm:StorjohannTransformation}
and \prettyref{exm:auxiliaryTransformation} in order to compute a $\left(\mathbf{F},8,\vec{0}\right)$-basis
(or equivalently a $(\bar{\mathbf{F}}^{\left(2\right)},8,\vec{0})$-basis).
This can be determined by computing a $(\mathbf{F}'^{\left(2\right)},[8,4,4],\vec{0})$-basis
as shown in \prettyref{exm:auxiliaryTransformation} where we have $\mathbf{F}'^{\left(2\right)}=\mathbf{F}'$.
Computing a $(\mathbf{F}'^{\left(2\right)},[8,4,4],\vec{0})$-basis
can be divided into two subproblems. The first subproblem is computing
a $(\bar{\mathbf{F}}^{\left(1\right)},4,\vec{0})$-basis $\bar{\mathbf{P}}^{\left(1\right)}$,
the Storjohann partial linearized problem in \prettyref{exm:StorjohannTransformation}.
The residual \[
\mathbf{F}'^{(2)}\bar{\mathbf{P}}^{(1)}=\left[{\begin{array}{rcccccccccc}
0 & \  & x^{8} & \  & x^{6}+x^{9} & \  & x^{4}+x^{6}+x^{9} & \  & x^{6}+x^{8}+x^{9}+x^{10} & \  & x^{5}+x^{8}\\
0 &  & 0 &  & x^{5} &  & x^{4}+x^{6} &  & x^{4}+x^{6} &  & x^{5}+x^{6}\\
0 &  & x^{4} &  & x^{5} &  & x^{5} &  & x^{4}+x^{5}+x^{6} &  & x^{4}\end{array}}\right]\]
 is then used as the input matrix for the second subproblem. The shift
for the second subproblem $\vec{t}^{(1)}=[0,1,2,3,3,3]$ is the list
of column degrees of $\bar{\mathbf{P}}^{(1)}$ and so the second subproblem
is to compute a $(\mathbf{F}'^{(2)}\bar{\mathbf{P}}^{(1)},\left[8,4,4\right],[0,1,2,3,3,3])$-basis,
which is \begin{equation}
\bar{\mathbf{Q}}^{(2)}=\left[{\begin{array}{cccccc}
~~1~ & ~~0~ & ~0~ & ~0~ & ~0~ & ~0~\\
0 & 1 & 0 & 0 & 0 & 0\\
0 & 0 & 1 & ~x^{2}~ & ~x~ & ~1~\\
0 & 0 & 0 & 0 & x & 0\\
0 & 0 & 1 & 0 & 0 & 0\\
0 & 0 & 0 & 0 & 1 & ~x~\end{array}}\right].\label{eq:Qbar2}\end{equation}
Then $\bar{\mathbf{P}}^{(1)}\bar{\mathbf{Q}}^{(2)}$ gives the $(\mathbf{F}'^{\left(2\right)},[8,4,4],\vec{0})$-basis
shown in \prettyref{exm:auxiliaryTransformation}. 
\end{exmp}
We now show that the dimension of the second subproblem can be significantly
reduced. First, the row dimension can be reduced by over a half. Let
$\hat{\mathbf{P}}^{\left(i-1\right)}=\mathbf{E}^{\left(i\right)}\bar{\mathbf{P}}^{\left(i-1\right)}$. 
\begin{lem}
\label{lem:simplifySecondSubproblem}A $(\bar{\mathbf{F}}^{\left(i\right)}\hat{\mathbf{P}}^{\left(i-1\right)},2\delta^{\left(i\right)},\vec{t}^{\left(i-1\right)})$-basis
is a $(\mathbf{F}'^{\left(i\right)}\bar{\mathbf{P}}^{\left(i-1\right)},\vec{\omega}^{\left(i\right)},\vec{t}^{\left(i-1\right)})$-basis.\end{lem}
\begin{pf}
This follows because 
$\bar{\mathbf{F}}^{\left(i\right)}\hat{\mathbf{P}}^{\left(i-1\right)}$
is a submatrix of 
$\mathbf{F}'^{\left(i\right)}\bar{\mathbf{P}}^{\left(i-1\right)}$
after removing rows which already have the correct order 
$2\delta^{\left(i-1\right)}$. 
\end{pf}
The column dimension of the second subproblem can be reduced by disregarding
the $(\bar{\mathbf{F}}^{\left(i\right)},2\delta^{\left(i\right)},\vec{s}^{\left(i\right)})_{\delta^{\left(i-1\right)}-1}$-basis
which has already been computed. More specifically, after sorting the columns of
$\bar{\mathbf{P}}^{\left(i-1\right)}$ in an increasing order of their
$\vec{s}^{\left(i-1\right)}$-degrees, let $[\bar{\mathbf{P}}_{1}^{\left(i-1\right)},\bar{\mathbf{P}}_{2}^{\left(i-1\right)}]=\bar{\mathbf{P}}^{\left(i-1\right)}$
be such that $\deg_{\vec{s}^{\left(i-1\right)}}\bar{\mathbf{P}}_{1}^{\left(i-1\right)}\le\delta^{\left(i-1\right)}-1$
and $\deg_{\vec{s}^{\left(i-1\right)}}\bar{\mathbf{P}}_{2}^{\left(i-1\right)}\ge\delta^{\left(i-1\right)}$.
Then $\hat{\mathbf{P}}_{1}^{\left(i-1\right)}=\mathbf{E}^{\left(i\right)}\bar{\mathbf{P}}_{1}^{\left(i-1\right)}$
is a $(\bar{\mathbf{F}}^{\left(i\right)},2\delta^{\left(i\right)},\vec{s}^{\left(i\right)})_{\delta^{\left(i-1\right)}-1}$-basis
by \prettyref{lem:linkStorjohanTransform}. In the second subproblem,
the remaining basis elements of a $(\bar{\mathbf{F}}^{\left(i\right)},2\delta^{\left(i\right)},\vec{s}^{\left(i\right)})$-basis
can then be computed without $\bar{\mathbf{P}}_{1}^{\left(i-1\right)}$.

Let $\hat{\mathbf{P}}_{2}^{\left(i-1\right)}=\mathbf{E}^{\left(i\right)}\bar{\mathbf{P}}_{2}^{\left(i-1\right)}$,
$\vec{b}^{\left(i-1\right)}=\deg_{\vec{s}^{\left(i-1\right)}}\bar{\mathbf{P}}_{2}^{\left(i-1\right)}$,
$\bar{\mathbf{Q}}_{2}^{\left(i\right)}$ be a $(\bar{\mathbf{F}}^{\left(i\right)}\hat{\mathbf{P}}_{2}^{\left(i-1\right)},2\delta^{\left(i\right)},\vec{b}^{\left(i-1\right)})$-basis
(or equivalently a $(\mathbf{F}'^{\left(i\right)}\bar{\mathbf{P}}_{2}^{\left(i-1\right)},\vec{\omega}^{\left(i\right)},\vec{b}^{\left(i-1\right)})$-basis),
and $k^{\left(i-1\right)}$ be the column dimension of $\bar{\mathbf{P}}_{1}^{\left(i-1\right)}$.
We then have the following result.
\begin{lem}
\label{lem:disregardComputedBasisElements} The matrix \[
\bar{\mathbf{Q}}^{\left(i\right)}=\left[\begin{array}{cc}
\mathbf{I}_{k^{\left(i-1\right)}}\\
 & \bar{\mathbf{Q}}_{2}^{\left(i\right)}\end{array}\right]\]
 is a $(\bar{\mathbf{F}}^{\left(i\right)}\hat{\mathbf{P}}^{\left(i-1\right)},2\delta^{\left(i\right)},\vec{t}^{\left(i-1\right)})$-basis
(equivalently a $(\mathbf{F}'^{\left(i\right)}\bar{\mathbf{P}}^{\left(i-1\right)},\vec{\omega}^{\left(i\right)},\vec{t}^{\left(i-1\right)})$-basis).\end{lem}
\begin{pf}
First note that $\bar{\mathbf{Q}}^{\left(i\right)}$ has order $(\bar{\mathbf{F}}^{\left(i\right)}\hat{\mathbf{P}}^{\left(i-1\right)},2\delta^{\left(i\right)})$
as \[
\bar{\mathbf{F}}^{\left(i\right)}\hat{\mathbf{P}}^{\left(i-1\right)}\bar{\mathbf{Q}}^{\left(i\right)}=[\bar{\mathbf{F}}^{\left(i\right)}\hat{\mathbf{P}}_{1}^{\left(i-1\right)},\bar{\mathbf{F}}^{\left(i\right)}\hat{\mathbf{P}}_{2}^{\left(i-1\right)}\bar{\mathbf{Q}}_{2}^{\left(i\right)}]\equiv0\mod x^{2\delta^{\left(i\right)}}.\]
 In addition, $\bar{\mathbf{Q}}^{\left(i\right)}$ has minimal $\vec{t}^{\left(i-1\right)}$
degrees as $\bar{\mathbf{Q}}_{2}^{\left(i\right)}$ is $\vec{b}$-minimal.
Hence, by \prettyref{lem:orderBasisEquivalence}, 
$\bar{\mathbf{Q}}^{\left(i\right)}$ is a 
$(\bar{\mathbf{F}}^{\left(i\right)}\cdot\hat{\mathbf{P}}^{\left(i-1\right)},2\delta^{\left(i\right)},\vec{t}^{\left(i-1\right)})$-basis.
\end{pf}
%
\begin{comment}
Alternatively, one can also argue that since $\hat{\mathbf{P}}_{1}^{\left(i-1\right)}$
already has order $(\bar{\mathbf{F}}^{\left(i\right)},2\delta^{\left(i\right)})$,
it cannot contribute in any way to and cannot be affected in any way
by the computations of a $(\bar{\mathbf{F}}^{\left(i\right)}\hat{\mathbf{P}}^{\left(i-1\right)},2\delta^{\left(i\right)},\vec{b}^{\left(i-1\right)})$-basis,
hence it is sufficient to just use $\bar{\mathbf{F}}^{\left(i\right)}\hat{\mathbf{P}}_{2}^{\left(i-1\right)}$
to compute a $(\bar{\mathbf{F}}^{\left(i\right)}\cdot\hat{\mathbf{P}}_{2}^{\left(i-1\right)},2\delta^{\left(i\right)},\vec{b}^{\left(i-1\right)})$-basis. 
\end{comment}
{}

\prettyref{lem:disregardComputedBasisElements} immediately leads
to the following. 
\begin{lem}
\label{lem:computationAtTopLevel}Let $\hat{\mathbf{S}}=[\hat{\mathbf{P}}_{1}^{\left(i-1\right)},\hat{\mathbf{P}}_{2}^{\left(i-1\right)}\bar{\mathbf{Q}}_{2}^{\left(i\right)}]$,
and let $I$ be the column rank profile of $\lcoeff(x^{\vec{s}^{\left(i\right)}}\hat{\mathbf{S}})$.
Then $\hat{\mathbf{S}}_{I}$ is a $(\bar{\mathbf{F}}^{\left(i\right)},2\delta^{\left(i\right)},\vec{s}^{\left(i\right)})$-basis.\end{lem}
\begin{pf}
From \prettyref{lem:disregardComputedBasisElements}, $\bar{\mathbf{Q}}^{\left(i\right)}$
is a $(\mathbf{F}'^{\left(i\right)}\bar{\mathbf{P}}^{\left(i-1\right)},\vec{\omega}^{\left(i\right)},\vec{t}^{\left(i-1\right)})$-basis
and hence $\bar{\mathbf{P}}^{\left(i-1\right)}\bar{\mathbf{Q}}^{\left(i\right)}$
is a $(\mathbf{F}'^{\left(i\right)},\vec{\omega}^{\left(i\right)},\vec{s}^{\left(i-1\right)})$-basis.
Since $[\hat{\mathbf{P}}_{1}^{\left(i-1\right)},\hat{\mathbf{P}}_{2}^{\left(i-1\right)}\bar{\mathbf{Q}}_{2}^{\left(i\right)}]=\mathbf{E}^{\left(i\right)}\bar{\mathbf{P}}^{\left(i-1\right)}\bar{\mathbf{Q}}^{\left(i\right)}$,
the result follows from \prettyref{thm:extractingOrderBasis}. 
%
\begin{comment}
$=\mathbf{E}^{\left(i\right)}[\bar{\mathbf{P}}_{1}^{\left(i-1\right)},\bar{\mathbf{P}}_{2}^{\left(i-1\right)}\bar{\mathbf{Q}}_{2}^{\left(i\right)}]$ 
\end{comment}
{}\end{pf}
\begin{exmp}
Continuing with \prettyref{exm:StorjohannTransformation}, \prettyref{exm:auxiliaryTransformation},
and \prettyref{exm:subproblems}, %
\begin{comment}
notice that after the first subproblem, the second subproblem of computing
$\bar{\mathbf{Q}}^{(2)}$ in \prettyref{exm:subproblems} is really
a smaller problem of computing the lower right $4\times4$ submatrix
$\bar{\mathbf{Q}}_{2}^{(2)}$, which is a $(\bar{\mathbf{F}}^{\left(2\right)}\hat{\mathbf{P}}_{2}^{\left(1\right)},8,\vec{b}^{\left(1\right)})$-basis
(or equivalently a $(\mathbf{F}'^{\left(2\right)}\bar{\mathbf{P}}_{2}^{\left(1\right)},[8,4,4],\vec{b}^{\left(1\right)})$-basis),
where $\bar{\mathbf{P}}_{2}^{\left(1\right)}$ is the last $4$ columns
of $\bar{\mathbf{P}}^{(1)}$, $\vec{b}^{(1)}=[2,3,3,3]$ is the list
of column degrees of $\bar{\mathbf{P}}_{2}^{(1)}$, and $\hat{\mathbf{P}}_{2}^{\left(1\right)}$
is the first $4$ rows of $\bar{\mathbf{P}}_{2}^{(1)}$. 
\end{comment}
{}notice that in the computation of the second subproblem, instead of
using $\mathbf{F}'^{\left(2\right)},$ $\bar{\mathbf{P}}^{\left(1\right)}$,
$\bar{\mathbf{Q}}^{(2)}$, and $\bar{\mathbf{P}}^{(1)}\bar{\mathbf{Q}}^{(2)}$,
the previous lemmas show that we can just use their submatrices, $\bar{\mathbf{F}}^{(2)}$
the top left $1\times4$ submatrix of $\mathbf{F}'^{\left(2\right)}$,
$\hat{\mathbf{P}}_{2}^{(1)}$ the top right $4\times4$ submatrix
of $\bar{\mathbf{P}}^{(1)}$, $\bar{\mathbf{Q}}_{2}^{(2)}$ the bottom
right $4\times4$ submatrix of $\bar{\mathbf{Q}}^{(2)}$, and $\hat{\mathbf{P}}_{2}^{(1)}\bar{\mathbf{Q}}_{2}^{(2)}$
the top right $4\times4$ submatrix of $\bar{\mathbf{P}}^{(1)}\bar{\mathbf{Q}}^{(2)}$of
lower dimensions. 
\end{exmp}
\prettyref{lem:computationAtTopLevel} gives us a way of computing
a $\left(\mathbf{F},\sigma,\vec{s}\right)$-basis. We can set $i$
to $\log\left(n/m\right)-1$ so that $(\bar{\mathbf{F}}^{\left(i\right)},2\delta^{\left(i\right)},\vec{s}^{\left(i\right)})$=$\left(\mathbf{F},\sigma,\vec{s}\right)$,
and compute a $(\bar{\mathbf{F}}^{\left(i\right)},2\delta^{\left(i\right)},\vec{s}^{\left(i\right)})$-basis.
By \prettyref{lem:computationAtTopLevel}, this can be divided into
two subproblems. The first produces $[\hat{\mathbf{P}}_{1}^{\left(i-1\right)},\hat{\mathbf{P}}_{2}^{\left(i-1\right)}]=\hat{\mathbf{P}}^{\left(i-1\right)}=\mathbf{E}^{\left(i\right)}\bar{\mathbf{P}}^{\left(i-1\right)}$
from computing a $(\bar{\mathbf{F}}^{\left(i-1\right)},2\delta^{\left(i-1\right)},\vec{s}^{\left(i-1\right)})$-basis
$\bar{\mathbf{P}}^{\left(i-1\right)}$. The second subproblem then
computes a $(\bar{\mathbf{F}}^{\left(i\right)}\hat{\mathbf{P}}_{2}^{\left(i-1\right)},2\delta^{\left(i\right)},\vec{b}^{\left(i-1\right)})$-basis
$\bar{\mathbf{Q}}_{2}^{\left(i\right)}$. Note the first subproblem
of computing a $(\bar{\mathbf{F}}^{\left(i-1\right)},2\delta^{\left(i-1\right)},\vec{s}^{\left(i-1\right)})$-basis
can again be divided into two subproblems just as before. This can
be repeated recursively until we reach the base case with degree parameter
$\delta^{\left(1\right)}=2d$. The total number of recursion levels
is therefore $\log\left(n/m\right)-1$.

Notice that the transformed matrix $\mathbf{F}'^{\left(i\right)}$
is not used explicitly in the computation, even though it is crucial
for deriving our results.


\subsection{The Iterative View}

In this subsection we present our algorithm, which uses an iterative
version of the computation discussed above.%
\begin{comment}
The recursive top-down approach of the previous subsection is useful
for giving an overall picture of the computation process. Algorithm
\prettyref{alg:mab} uses an equivalent corresponding bottom-up iterative
approach. 
\end{comment}
{}%
\begin{comment}
, allowing the complexity to be more easily analyzed. 
\end{comment}
{}%
\begin{comment}
. In practice, bottom-up iterative approaches are more efficient than
the corresponding top-down recursive approaches. For our purpose,
it is also easier to analyze the computational cost of the iterative
procedure. 
\end{comment}
{} The iterative version is usually more efficient in practice, considering
that the generic case has balanced output that can be computed with
just one iteration, whereas the recursive method has to go through
$\log(n/m)-1$ levels of recursion.

\prettyref{alg:mab} uses a subroutine $\mab$, the algorithm from
Giorgi et al. \citeyearpar{Giorgi2003}, for computing order bases
with balanced input. Specifically, $\left[\mathbf{Q},\vec{a}\right]=\mab(\mathbf{G},\sigma,\vec{b})$
computes a $(\mathbf{G},\sigma,\vec{b})$-basis and also returns its
$\vec{b}$-column degrees $\vec{a}$. The other subroutine $\StorjohannTransform$
is the transformation described in \prettyref{sub:storjohannTransformation}.

\prettyref{alg:mab} proceeds as follows. In the first iteration,
which is the base case of the recursive approach, we set the degree
parameter $\delta^{\left(1\right)}$ to be twice the average degree
$d$ and apply Storjohann's transformation to produce a new input
matrix $\bar{\mathbf{F}}^{\left(1\right)}$, which has $l^{\left(1\right)}$
block rows. Then a $(\bar{\mathbf{F}}^{\left(1\right)},2\delta^{\left(1\right)},\vec{s}^{\left(1\right)})$-basis
$\bar{\mathbf{P}}^{\left(1\right)}$ is computed. Note this is in
fact the first subproblem of computing a $(\bar{\mathbf{F}}^{\left(2\right)},2\delta^{\left(2\right)},\vec{s}^{\left(2\right)})$-basis,
which is another Storjohann transformed problem and also the problem
of the second iteration. At the second iteration, we work on a
new Storjohann transformed problem with the degree doubled and the
number of block rows $l^{\left(2\right)}=(l^{\left(1\right)}-1)/2$
reduced by over a half. The column dimension is reduced by using the
result from the previous iteration. More specifically, we know that
the basis $\bar{\mathbf{P}}^{\left(1\right)}$ already provides a
$(\bar{\mathbf{F}}^{\left(2\right)},2\delta^{\left(2\right)},\vec{s}^{\left(2\right)})_{\delta^{\left(1\right)}-1}$-basis
$\hat{\mathbf{P}}_{1}^{\left(1\right)}$, which can be disregarded
in the remaining computation. The remaining work in the second
iteration is to compute a $(\bar{\mathbf{F}}^{\left(2\right)}\hat{\mathbf{P}}_{2}^{\left(1\right)},2\delta^{\left(2\right)},\vec{b}^{\left(1\right)})$-basis
$\bar{\mathbf{Q}}^{\left(2\right)}$, where $\vec{b}^{\left(1\right)}=\deg_{\vec{s}^{\left(1\right)}}\bar{\mathbf{P}}_{2}^{\left(1\right)}$,
and then to combine it with the result from the previous iteration
to form a matrix $[\hat{\mathbf{P}}_{1}^{\left(1\right)},\hat{\mathbf{P}}_{2}^{\left(1\right)}\bar{\mathbf{Q}}^{\left(2\right)}]$
in order to extract a $(\bar{\mathbf{F}}^{\left(2\right)},2\delta^{\left(2\right)},\vec{s}^{\left(2\right)})$-basis
$\bar{\mathbf{P}}^{\left(2\right)}$.

With a $(\bar{\mathbf{F}}^{\left(2\right)},2\delta^{\left(2\right)},\vec{s}^{\left(2\right)})$-basis
computed, we can repeat the same process to use it for computing a
$(\bar{\mathbf{F}}^{\left(3\right)},2\delta^{\left(3\right)},\vec{s}^{\left(3\right)})$-basis.
Continue, using the computed $(\bar{\mathbf{F}}^{\left(i-1\right)},2\delta^{\left(i-1\right)},\vec{s}^{\left(i-1\right)})$-basis
to compute a $(\bar{\mathbf{F}}^{\left(i\right)},2\delta^{\left(i\right)},\vec{s}^{\left(i\right)})$-basis,
until all $n$ elements of a $\left(\mathbf{F},\sigma,\vec{s}\right)$-basis
have been determined.

\input{algorithm1.tex}

%
\begin{comment}
\begin{thm}
Algorithm \prettyref{alg:mab} computes a $\left(\mathbf{F},\sigma,\vec{s}\right)$-basis
correctly.\end{thm}
\begin{pf}
This follows from \prettyref{lem:simplifySecondSubproblem}, \prettyref{lem:disregardComputedBasisElements},
and \prettyref{lem:computationAtTopLevel}. 
\end{pf}

\end{comment}
{} 
