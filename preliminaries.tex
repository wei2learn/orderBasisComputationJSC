
\section{Preliminaries}

\label{sec:Background}

The computational cost in this paper is analysed by bounding the number
of arithmetic operations (additions, subtractions, multiplications,
and divisions) in the coefficient field $\mathbb{K}$ on an algebraic
random access machine. We use $\MM(n,d)$ to denote the cost of multiplying
two polynomial matrices with dimension $n$ and degree $d$, and $\M(n)$
to denote the cost of multiplying two polynomials with degree $d$.
We define a cost function $\bar{\M}(d)=d\log d\log\log d$, then $\bar{\M}(ab)\in O\left(\bar{\M}(a)\bar{\M}(b)\right)$
and $\bar{\M}(t)\in O(n^{\omega-1})$. We take $\MM(n,d)\in O\left(n^{\omega}\M(d)\right)\subset O(n^{\omega}\M'(d))$,
where the multiplication exponent $\omega$ is assumed to satisfy
$2<\omega\le3$. We refer to the book by \citet{vonzurgathen} for
more details and reference about the cost of polynomial multiplication
and matrix multiplication.

In the remaining of this section, we provide some of the background
needed in order to understand the basic concepts and tools needed
for order basis computation. This includes basic definitions and a
look at the size of the input and the output for computing such bases.%
\begin{comment}
, which provide lower bounds for the computational cost 
\end{comment}
{} The challenges of balancing input and handling unbalanced output
are discussed along with the techniques which we plan to use to overcome
the difficulties. We review the construction by \citet{Storjohann:2006}
which transforms the inputs to those having dimensions and degree
balance better suited for fast computation and discuss an idea from
\citet{storjohann-villard:2005} for handling the case where the output
degree is unbalanced.


\subsection{Order Basis}

Let $\mathbb{K}$ be a field, $\mathbf{F}\in\mathbb{K}\left[\left[x\right]\right]^{m\times n}$
a matrix of power series and $\vec{\sigma}=\left[\sigma_{1},\dots,\sigma_{m}\right]$
a vector of non-negative integers. 
\begin{defn}
A vector of polynomials $\mathbf{p}\in\mathbb{K}\left[x\right]^{n\times1}$
has \emph{order} $\left(\mathbf{F},\vec{\sigma}\right)$ (or \emph{order}
$\vec{\sigma}$ with respect to $\mathbf{F}$) if $\mathbf{F}\cdot\mathbf{p}\equiv\mathbf{0}\mod x^{\vec{\sigma}}$,
that is, \[
\mathbf{F}\cdot\mathbf{p}=x^{\vec{\sigma}}\mathbf{r}=\begin{bmatrix}x^{\sigma_{1}}\\
 & \ddots\\
 &  & x^{\sigma_{m}}\end{bmatrix}\mathbf{r}\]
 for some $\mathbf{r}\in\mathbb{K}\left[\left[x\right]\right]^{m\times1}$.
If $\vec{\sigma}=\left[\sigma,\dots,\sigma\right]$ is uniform, then
we say that $\mathbf{p}$ has order $\left(\mathbf{F},\sigma\right).$
%
\begin{comment}
The vector of power series $\mathbf{r}$ is called the order $\left(\mathbf{F},\sigma\right)$-residual
of \textbf{$\mathbf{p}$}. 
\end{comment}
{} The set of all order $\left(\mathbf{F},\vec{\sigma}\right)$ vectors
is a $\mathbb{K}\left[x\right]$-module denoted by $\left\langle \left(\mathbf{F},\vec{\sigma}\right)\right\rangle $. 
\end{defn}
An order basis for $\mathbf{F}$ and $\vec{\sigma}$ is simply a basis
for the module $\left\langle \left(\mathbf{F},\vec{\sigma}\right)\right\rangle $.
In this paper we compute those order bases having a type of minimality
degree condition (also referred to as a reduced order basis in \citep{BL1997}).
While minimality is often given in terms of the degrees alone it is
sometimes important to consider this in terms of shifted degrees \citep{BLV:jsc06}.

The shifted column degree of a column polynomial vector $\mathbf{p}$
with shift $\vec{s}=\left[s_{1},\dots,s_{n}\right]\in\mathbb{Z}^{n}$
is given by \[
\deg_{\vec{s}}\mathbf{p}=\max_{1\le i\le n}[\deg p^{\left(i\right)}+s_{i}]=\deg(x^{\vec{s}}\cdot\mathbf{p}).\]
 We call this the \emph{$\vec{s}$-column degree}, or simply the \emph{$\vec{s}$-degree}
of $\mathbf{p}$. A shifted column degree defined this way is equivalent
to the notion of \emph{defect} commonly used in the literature. Our
definition of $\vec{s}$-degree is also equivalent to the notion of
$\mathbf{H}$-degree from \citep{BL1997} for $\mathbf{H}=x^{\vec{s}}$.
As in the uniform shift case, we say a matrix is \emph{$\vec{s}$-column
reduced} or \emph{$\vec{s}$-reduced} if its $\vec{s}$-degrees cannot
be decreased by unimodular column operations. More precisely, if $\mathbf{P}$
is a $\vec{s}$-column reduced and $[d_{1},\dots,d_{n}]$ are the
$\vec{s}$-degrees of columns of $\mathbf{P}$ sorted in nondecreasing
order, then $[d_{1},\dots,d_{n}]$ is lexicographically minimal among
all matrices right equivalent to $\mathbf{P}$. Note that a matrix
$\mathbf{P}$ is $\vec{s}$-column reduced if and only if $x^{\vec{s}}\cdot\mathbf{P}$
is column reduced. Similarly, $\mathbf{P}$ is in $\vec{s}$-Popov
form if $x^{\vec{s}}\cdot\mathbf{P}$ is in Popov form \citep{BLV:1999,BLV:jsc06}. 

An \emph{order basis} \citep{BeLa94,BL1997} $\mathbf{P}$ of $\mathbf{F}$
with order $\vec{\sigma}$ and shift $\vec{s}$, or simply an $\left(\mathbf{F},\vec{\sigma},\vec{s}\right)$-basis,
is a basis for the module $\left\langle \left(\mathbf{F},\vec{\sigma}\right)\right\rangle $
%
\begin{comment}
\[
\left\langle \left(\mathbf{F},\vec{\sigma}\right)\right\rangle =\{\mathbf{p}\in\mathbb{K}\left[x\right]^{n\times1}\|\mathbf{F}\cdot\mathbf{p}=x^{\vec{\sigma}}\mathbf{r},\mathbf{r}\in\mathbb{K}[[x]]^{m\times1}\}\]

\end{comment}
{} having minimal $\vec{s}$-column degrees. If $\vec{\sigma}=\left[\sigma,\dots,\sigma\right]$
are constant vectors then we simply write $\left(\mathbf{F},\sigma,\vec{s}\right)$-basis.
The precise definition of an $\left(\mathbf{F},\vec{\sigma},\vec{s}\right)$-basis
is as follows.
\begin{defn}
A polynomial matrix $\mathbf{P}$ is an order basis of $\mathbf{F}$
of order $\sigma$ and shift $\vec{s}$, denoted by $\left(\mathbf{F},\vec{\sigma},\vec{s}\right)$-basis,
if the following properties hold:
\begin{enumerate}
\item $\mathbf{P}$ is a nonsigular matrix of dimension $n$.
\item $\mathbf{P}$ is $\vec{s}$-column reduced. 
\item $\mathbf{P}$ has order $\left(\mathbf{F},\vec{\sigma}\right)$ (or
equivalently, each column of $\mathbf{P}$ is in $\left\langle (\mathbf{F},\vec{\sigma})\right\rangle $). 
\item Any $\mathbf{q}\in\left\langle \left(\mathbf{F},\vec{\sigma}\right)\right\rangle $
can be expressed as a linear combination of the columns of $\mathbf{P}$,
given by $\mathbf{P}^{-1}\mathbf{q}$. 
\end{enumerate}
\end{defn}
%
\begin{comment}
Note that the module $\left\langle \left(\mathbf{F},\vec{\sigma}\right)\right\rangle $
does not depend on the shift $\vec{s}$. 
\end{comment}
{}

Although we allow different orders for each row in this definition,
we focus on order basis computation problems having uniform order.
However special cases of non-uniform order problems are still needed
in our analysis. We also assume $m\le n$ for simplicity. The case
of $m>n$ can be transformed to the case of $m\le n$ by compression
\citep{storjohann-villard:2005}. We further assume, without any loss
of generality, that $n/m$ and $\sigma$ are powers of two. This can
be achieved by padding zero rows to the input matrix and multiplying
it by some power of $x$.

From \citep{BL1997} we have the following lemma. 
\begin{lem}
\label{lem:orderBasisProperty} 

\label{lem:orderBasisEquivalence}The following are equivalent for
a polynomial matrix \textbf{$\mathbf{P}$}: 
\begin{enumerate}
\item $\mathbf{P}$ is a $\left(\mathbf{F},\vec{\sigma},\vec{s}\right)$-basis. 
\item $\mathbf{P}$ is comprised of a set of $n$ minimal $\vec{s}$-degree
polynomial vectors that are linearly independent and each having order
$\left(\mathbf{F},\vec{\sigma}\right)$. 
\item \label{enu:reduced+generator}$\mathbf{P}$ does not contain a zero
column, has order $\left(\mathbf{F},\vec{\sigma}\right)$, is $\vec{s}$-column
reduced, and any $\mathbf{q}\in\left\langle \left(\mathbf{F},\vec{\sigma}\right)\right\rangle $
can be expressed as a linear combination of the columns of $\mathbf{P}$. 
\end{enumerate}
\end{lem}
In some cases an entire order basis is unnecessary and instead one
looks for a minimal basis that generates only the elements of $\left\langle \left(\mathbf{F},\vec{\sigma}\right)\right\rangle $
with $\vec{s}$-degrees bounded by a given $\delta$. Such a minimal
basis is a partial $\left(\mathbf{F},\vec{\sigma},\vec{s}\right)$-basis
comprised of elements of a $\left(\mathbf{F},\vec{\sigma},\vec{s}\right)$-basis
with $\vec{s}$-degrees bounded by $\delta$. This is called a \emph{minbasis}
in \citet{Storjohann:2006}. 
\begin{defn}
\label{def:genset} Let $\left\langle \left(\mathbf{F},\vec{\sigma},\vec{s}\right)\right\rangle _{\delta}\subset\left\langle \left(\mathbf{F},\vec{\sigma}\right)\right\rangle $
denote the set of order $\left(\mathbf{F},\vec{\sigma}\right)$ polynomial
vectors with $\vec{s}$-degree bounded by $\delta$. A $\left(\mathbf{F},\vec{\sigma},\vec{s}\right)_{\delta}$-basis
is a polynomial matrix $\mathbf{P}$ not containing a zero column
and satisfying: 
\begin{enumerate}
\item $\mathbf{P}$ has order $\left(\mathbf{F},\vec{\sigma}\right).$ 
\item Any element of $\left\langle \left(\mathbf{F},\vec{\sigma},\vec{s}\right)\right\rangle _{\delta}$
can be expressed as a linear combination of the columns of $\mathbf{P}$. 
\item $\mathbf{P}$ is $\vec{s}$-column reduced. 
\end{enumerate}
\end{defn}
%
\begin{comment}
As before, the linear combination here is in fact unique. 
\end{comment}
{}A $\left(\mathbf{F},\vec{\sigma},\vec{s}\right)_{\delta}$-basis is,
in general, not square unless $\delta$ is large enough to contain
all $n$ basis elements in which case it is a complete $\left(\mathbf{F},\vec{\sigma},\vec{s}\right)$-basis.


\subsection{Balancing Input with Storjohann's Transformation}

\label{sub:storjohannTransformation}

For computing a $\left(\mathbf{F},\sigma,\vec{s}\right)$-basis with
input matrix $\mathbf{F}\in\mathbb{K}\left[\left[x\right]\right]^{m\times n}$,
shift $\vec{s}$ and order $\sigma$ one can view $\mathbf{F}$ as
a polynomial matrix with degree $\sigma-1$, as higher order terms
are not needed in the computation. As such the total input size of
an order basis problem is $mn\sigma$ coefficients. One can apply
the method of \citet{Giorgi2003} directly, which gives a cost of
\begin{align*}
\sum_{i=0}^{\log\sigma}2^{i}\MM(n,2^{-i}\sigma)= & \sum_{i=0}^{\log\sigma}2^{-i}\sigma\MM(n,2^{i})\\
\subset & O\left(\sum_{i=0}^{\log\sigma}2^{-i}n^{\omega}\sigma2^{i}\log2^{i}\log\log2^{i}\right)\\
= & O\left(n^{\omega}\sigma\sum_{i=0}^{\log\sigma}i\log i\right)\\
\subset & O\left(n^{\omega}\sigma\sum_{i=0}^{\log\sigma}\log\sigma\log\log\sigma\right)\\
= & O\left(n^{\omega}\sigma\log^{2}\sigma\log\log\sigma\right)=O(n^{\omega}\bar{\M}(\sigma)\log\sigma)\end{align*}
field operations, close to the cost of multiplying two matrices with
dimension $n$ and degree $\sigma$. Note that this cost is independent
of the degree shift. This is very efficient if $m\in\Theta\left(n\right)$.
However, for small $m$, say $m=1$ as in Hermite Pad� approximation,
the total input size is only $n\sigma$ coefficients. Matrix multiplication
cannot be used effectively on a such vector input.

\citet{Storjohann:2006} provides a novel way to transform an order
basis problem with small row dimension to a problem with higher row
dimension and possibly lower degree to take advantage of \citet{Giorgi2003}'s
algorithm. We provide a quick overview of a slightly modified version
of Storjohann's method. Our small modification allows a nonuniform
degree shift for the input and provides a slightly simpler degree
shift, degree, and order for the transformed problem. The proof of
its correctness is provided in \prettyref{sec:transform}. In order
to compute a $\left(\mathbf{F},\sigma,\vec{s}\right)$-basis, assuming
without loss of generality that $\min\left(\vec{s}\right)=0$, we
first write \[
\mathbf{F}=\mathbf{F}_{0}+\mathbf{F}_{1}x^{\delta}+\mathbf{F}_{2}x^{2\delta}+\cdots+\mathbf{F}_{l}x^{l\delta},\]
 with $\deg\mathbf{F}_{i}<\delta$%
\begin{comment}
I used $\deg\mathbf{F}_{i}\le\delta-1$ before, but the reviewer suggested
changing to $\deg\mathbf{F}_{i}<\delta$ and said it's slightly easier
to read. A reason to use $\le$ is to make it consistent with the
definition of minbasis. For example, $\left(\mathbf{F},\sigma,\vec{s}\right)_{\delta-1}$-basis
indicates that the degree bound is $\delta-1$. I still prefer to
use $\le$. 
\end{comment}
{} for a positive integer $\delta$, and where we assume (again without
loss of generality) that $\sigma=\left(l+1\right)\delta$. Set \[
{\bar{\mathbf{F}}}=\left[\begin{array}{c|cccc}
\mathbf{F}_{0}+\mathbf{F}_{1}x^{\delta} & \mathbf{0}_{m} & \mathbf{0}_{m} & \cdots & \mathbf{0}_{m}\\
\hline \mathbf{F}_{1}+\mathbf{F}_{2}x^{\delta} & \mathbf{I}_{m} & \mathbf{0}_{m}\\
\mathbf{F}_{2}+\mathbf{F}_{3}x^{\delta} & \mathbf{0}_{m} & \mathbf{I}_{m}\\
\vdots &  &  & \ddots\\
\mathbf{F}_{l-1}+\mathbf{F}_{l}x^{\delta} &  &  &  & \mathbf{I}_{m}\end{array}\right]_{ml\times(n+m(l-1))}.\]
 On the left side of $\bar{\mathbf{F}}$, each block $\mathbf{F}_{i}+\mathbf{F}_{i+1}x^{\delta}$
has dimension $m\times n$. On the right side, there are $l\times(l-1)$
blocks of $\mathbf{0}_{m}$'s or $\mathbf{I}_{m}$'s each having dimension
$m\times m$. The overall dimension of $\bar{\mathbf{F}}$ is $ml\times(n+m(l-1))$.
Set $\vec{s'}=\left[\vec{s},0,\dots,0\right]$ ($\vec{s}$ followed
by $m\left(l-1\right)$ $0$'s). A $({\bar{\mathbf{F}}},2\delta,\vec{s'})$-basis
can then be computed by the method of Giorgi et al. with a cost of
$O^{\sim}\left(n^{\omega}\delta\right)$ for $\delta\ge\left\lceil m\sigma/n\right\rceil $.
This transformation of Storjohann can be viewed as a partial linearization
of the original problem, where $\bar{\mathbf{F}}$ is analogous to
the coefficient matrix of $\mathbf{F}$. Note that $\bar{\mathbf{F}}$
has $l$ block rows each containing $m$ rows. We continue to use
each block row to represent $m$ rows for the remainder of the paper.

Clearly a $(\bar{\mathbf{F}},2\delta,\vec{s'})$-basis $\bar{\mathbf{P}}$
of the transformed problem is not a $\left(\mathbf{F},\sigma,\vec{s}\right)$-basis
of the original problem, as $\bar{\mathbf{P}}$ has a higher dimension
and lower degree. However, the first $n$ rows of the $(\bar{\mathbf{F}},2\delta,\vec{s'})_{\delta-1}$-basis
contained in $\bar{\mathbf{P}}$ is a $\left(\mathbf{F},\sigma,\vec{s}\right)_{\delta-1}$-basis.

Note that there is no need to set the degree parameter $\delta$ to
less than $\left\lceil m\sigma/n\right\rceil $, as this produces
fewer basis elements without a better cost. The lowest cost is achieved
when $\bar{\mathbf{F}}$ is close to square so matrix multiplication
can be used most effectively. This requires the number of block rows
$l$ of $\bar{\mathbf{F}}$ to be close to $n/m$, which requires
$\delta=\Theta\left(\left\lceil m\sigma/n\right\rceil \right)$. Recall
that $mn\sigma$ is the total size of the original $m\times n$ input
matrix $\mathbf{F}$, hence $d=mn\sigma/n^{2}=m\sigma/n$ is the average
degree of each entry of $\mathbf{F}$ if the $m$ rows of $\mathbf{F}$
are spread out over $n$ rows. Choosing $\delta=\Theta\left(\left\lceil d\right\rceil \right)$,
the cost of computing a $({\bar{\mathbf{F}}},2\delta,\vec{s'})$-basis
is then $O^{\sim}\left(n^{\omega}\left\lceil d\right\rceil \right)=O^{\sim}\left(n^{\omega}\left\lceil m\sigma/n\right\rceil \right)$.
The ceiling function here is used to take care of the case of $m\sigma<n$.
For the remainder of the paper, we assume that $m\sigma\ge n$ in
order to avoid the need for the ceiling function and so simplify the
presentation. % avoid this case for simplicity. 
Together with the assumption that $\sigma$ and $n/m$ are both powers
of two, $m\sigma/n$ is then always a positive integer in this paper.

%Let us now look at a concrete example that illustrate Storjohann's method. 

\begin{exmp}
\label{exm:StorjohannTransformation}Let $\mathbb{K}=\mathbb{Z}_{2}$,
$\sigma=8$, $\delta=2$ and \[
\mathbf{F}=[x+x^{2}+x^{3}+x^{4}+x^{5}+x^{6},~1+x+x^{5}+x^{6}+x^{7},~1+x^{2}+x^{4}+x^{5}+x^{6}+x^{7},~1+x+x^{3}+x^{7}]\]
 a vector of size $1\times4$. Then \[
\bar{\mathbf{F}}=\left[{\begin{array}{cccc|cc}
x+x^{2}+x^{3} & 1+x & 1+x^{2} & 1+x+x^{2} & ~~0~ & ~0~\\
\hline 1+x+x^{2}+x^{3} & x^{3} & 1+x^{2}+x^{3} & x & ~~1~ & ~0~\\
1+x+x^{2} & x+x^{2}+x^{3} & 1+x+x^{2}+x^{3} & x^{3} & ~~0~ & ~1~\end{array}}\right]_{3\times6}\]
 and a $\left(\bar{\mathbf{F}},4,\vec{0}\right)$-basis is given by
\[
\bar{\mathbf{P}}=\left[{\begin{array}{cc|cccc}
~1~ & ~x~ & 1 & x^{2}+x^{3} & 0 & x+x^{2}+x^{3}\\
0 & 1 & 0 & x^{2} & x^{2}+x^{3} & 0\\
1 & 1+x & x+x^{2} & x^{2} & x^{2} & x^{2}\\
1 & 0 & 0 & 0 & 0 & 0\\
\hline 0 & 1 & 1 & 0 & x^{2} & x+x^{2}+x^{3}\\
0 & 1 & 1+x^{2} & 0 & x^{2} & x+x^{2}\end{array}}\right].\]
 The first two columns of $\bar{\mathbf{P}}$ have degree less than
$2$, hence its top left $4\times2$ submatrix is a $\left(\mathbf{F},8,\vec{0}\right)_{1}$-basis.
This is a low degree part of the\textbf{ $(\mathbf{F},8,\vec{0})$}-basis\[
\mathbf{P}=\begin{bmatrix}1 & x & 1 & x^{2}\\
0 & 1 & x^{2}+x^{3} & 0\\
1 & 1+x & x & x^{3}+x^{4}\\
1 & 0 & 0 & 0\end{bmatrix}.\]
 Note that if $\delta$ is set to $\sigma/2=4$, then the transformed
problem is the same as the original problem. 
\end{exmp}

\subsection{\label{sub:Unbalanced-Output}Unbalanced Output }

Storjohann's transformation can be used to efficiently compute a $\left(\mathbf{F},\sigma,\vec{s}\right)_{\delta-1}$-basis
if the degree parameter $\delta$ is close to the average degree $d=m\sigma/n$.
However, if $\delta$ is large, say $\delta=\Theta\left(\sigma\right)$,
or if we want to compute a complete $\left(\mathbf{F},\sigma,\vec{s}\right)$-basis,
then the current analysis for the computation still gives the cost
estimate of  $O^{\sim}\left(n^{\omega}\sigma\right)$.

The underlying difficulty with computing a complete order basis is
that the basis can have degree up to $\sigma$. As the output of this
problem has dimension $n\times n$ and degree up to $\Theta\left(\sigma\right)$,
this may seem to suggest $O^{\sim}\left(n^{\omega}\sigma\right)$
is about the best that can be done. However, the total size of the
output, that is, the total number of coefficients of all $n^{2}$
polynomial entries can still be bounded by $O\left(mn\sigma\right)$,
the same as the size of the input. This gives some hope for a more
efficient method. 
\begin{lem}
\label{lem:size}Let $\vec{t}$ be the $\vec{s}$-column degrees of
a $\left(\mathbf{F},\sigma,\vec{s}\right)$-basis. Then $\sum_{i}\left(\vec{t}_{i}-\vec{s}_{i}\right)~\le~m\sigma$%
\begin{comment}
 and $\max_{i}\left(\vec{t}_{i}-\vec{s}_{i}\right)\le\sigma$
\end{comment}
{}\textup{}%
\begin{comment}
need to permute the columns to put the pivots on the diagonal.
\end{comment}
{}. In addition, the total size of any $\left(\mathbf{F},\sigma,\vec{s}\right)$-basis
in $\vec{s}$-Popov form is bounded by $nm\sigma$. \end{lem}
\begin{pf}
This can be shown by considering the sizes of the pivots in the iterative
order basis computation given in \citep{BeLa94,Giorgi2003}. %
\begin{comment}
Consider the degree of pivot entries in the iterative order basis
computation from \citet{BeLa94}. Starting at order zero, for each
order increasing by one, the sum of the degrees of all the pivot entries
is increased by at most $m$. Hence the sum is at most $m\sigma$
after $\sigma$ increases for an order $\sigma$ order basis. Now
consider any particular column of the basis. Each entry in the column
has degree bounded by the corresponding pivot entry of the row, so
the total degree of the entries in the column is bounded by $m\sigma$.
Therefore, the size of any order $\sigma$ basis with $n$ columns
is bounded by $nm\sigma$ coefficients. 
\end{comment}
{} 
\end{pf}
%
\begin{comment}
As a result, the average degree of the entries of the output matrix
can be also bounded by $d=m\sigma/n$. 
\end{comment}
{}

Let us now look at the average column degree of the output. In the
first part of this paper, we assumed, without loss of generality,
that $\min\left(\vec{s}\right)=0$ so $\deg\mathbf{q}\le\deg_{\vec{s}}\mathbf{q}$
for any $\mathbf{q}\in\mathbb{K}\left[x\right]^{n}$. The situation
is simpler if the shift $\vec{s}$ is uniform since then $\sum_{i}\vec{t}_{i}\le m\sigma$
by \prettyref{lem:size} and the average column degree is therefore
bounded by $d=m\sigma/n$. In the first part of this paper, we consider
a slightly more general case, when the shift $\vec{s}$ is \emph{balanced},
which is defined as follows. 
\begin{defn}
A shift $\vec{s}$ is balanced if $\max\vec{s}-\min\vec{s}\in O(d)=O(m\sigma/n)$. 
\end{defn}
By assuming $\min\vec{s}=0$, $\vec{s}$ is balanced if $\max\vec{s}\in O(d)$.
In this case, \prettyref{lem:size} implies $\sum_{i}\left(\vec{t}_{i}\right)\le m\sigma+\sum_{i}\left(\vec{s}_{i}\right)\in O\left(m\sigma+nd\right)=O\left(m\sigma\right)$.
Hence the average column degree of the output basis remains $O\left(d\right)$.

%From the iterative algorithms for computing order basis, computing
%a order basis to order $\sigma$ requires up to $\sigma$ iterations,
%each iteration increases the sum of column degrees of the order basis
%by at most $m$. Therefore, the sum of column degrees of an order
%$\sigma$ order basis is at most $m\sigma$.


%
\begin{comment}
In fact, if $\mathbf{F}\left(0\right)$ is full rank, the sum of column
degrees of an order $\sigma$ order basis is exactly $m\sigma$, as
we need exactly $\sigma$ iterations, each increases the sum of the
column degrees by exactly $m$. 
\end{comment}
{}

The fact that a $\left(\mathbf{F},\sigma,\vec{s}\right)$-basis can
have degree up to $\sigma$ while its average column degree is $O\left(m\sigma/n\right)$
implies that an order basis can have quite unbalanced column degrees,
especially if $m$ is small. A similar problem with unbalanced output
is encountered in null space basis computation. \citet{storjohann-villard:2005}
deal with this in the following way.

Let $d$ be the average column degree of the output. Set the degree
parameter $\delta$ to twice that of $d$. This allows one to compute
at least half the columns of a basis (since the number of columns
with degree at least $\delta$ must be at most a half of the total
number of columns). One can then simplify the problem, so that the
computed basis elements are completely removed from the problem. This
reduces the dimension of the problem by at least a factor of $2$.
One then doubles the degree bound $\delta$ in order to have at least
$3/4$ of the basis elements computed. Repeating this, at iteration
$i$, at most $1/2^{i}$ of the basis elements are remaining. Therefore,
no more than $\log n$ iterations are needed to compute all basis
elements.

%We now have two ideas for efficient computation. Storjohann's transformation
%for balancing the input and Storjohann and Villard's method for handling
%unbalanced output. 
%In this paper, we discuss a way to compute order basis involving the
%ideas of \citep{storjohann-villard:2005} and \citep{Storjohann:2006}. 

