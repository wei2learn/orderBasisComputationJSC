%% LyX 1.6.8 created this file.  For more info, see http://www.lyx.org/.
%% Do not edit unless you really know what you are doing.
\documentclass[english,amsthm,seceqn,secthm]{elsart}
\usepackage[T1]{fontenc}
\usepackage[latin9]{inputenc}
\usepackage{verbatim}
\usepackage{prettyref}
\usepackage{float}
\usepackage{amsmath}
\usepackage{amssymb}
\usepackage[authoryear]{natbib}

\makeatletter

%%%%%%%%%%%%%%%%%%%%%%%%%%%%%% LyX specific LaTeX commands.
\floatstyle{ruled}
\newfloat{algorithm}{tbp}{loa}
\floatname{algorithm}{Algorithm}

%%%%%%%%%%%%%%%%%%%%%%%%%%%%%% Textclass specific LaTeX commands.
 \let\AND\relax 
 \usepackage{algorithmic}
 \usepackage{algolyx}
 \usepackage{algolyx}

%%%%%%%%%%%%%%%%%%%%%%%%%%%%%% User specified LaTeX commands.
\usepackage{yjsco}\journal{JournalofSymbolicComputation}

\renewcommand{\algorithmicrequire}{\textbf{Input:}}\renewcommand{\algorithmicensure}{\textbf{Output:}}\renewcommand{\algorithmicensure}{\textbf{if}}\renewcommand{\algorithmicensure}{\textbf{Uses:}}

\def\diag{\mbox{diag}}\def\cdeg{\qopname\relax n{cdeg}}\keycomment{\{\{}{\}\}}\def\MM{\qopname\relax n{MM}}\def\M{\qopname\relax n{M}}\def\ord{\qopname\relax n{ord}}

\def\StorjohannTransform{\qopname\relax n{StorjohannTransform}}\def\TransformUnbalanced{\qopname\relax n{TransformUnbalanced}}\def\rowDimension{\qopname\relax n{rowDimension}}\def\columnDimension{\qopname\relax n{columnDimension}}\DeclareMathOperator{\re}{rem}\DeclareMathOperator{\coeff}{coeff}\DeclareMathOperator{\lcoeff}{lcoeff}\def\mab{\qopname\relax n{OrderBasis}}\def\mmab{\qopname\relax n{FastBasis}}\def\umab{\qopname\relax n{UnbalancedFastBasis}}\newcommand{\bb}{\\}

\makeatother

\makeatother

\usepackage{babel}

\begin{document}
\begin{frontmatter}

\title{Efficient Algorithms for Order Bases Computation}


\author{Wei Zhou and George Labahn}


\address{{Cheriton School of Computer Science}\\
 {University of Waterloo},\\
 {Waterloo, Ontario, Canada}}


\ead{\{w2zhou,glabahn\}@uwaterloo.ca }
\begin{abstract}
In this paper we present two algorithms for the computation of a shifted
order basis of an $m\times n$ matrix of power series over a field
$\mathbb{K}$ with $m\le n$. For a given order $\sigma$ and balanced
shift $\vec{s}$ the first algorithm determines an order basis with
a cost of $O^{\sim}(\MM(n,\lceil m\sigma/n\rceil))\subset O^{\sim}(n^{\omega}\lceil m\sigma/n\rceil)$
field operations in $\mathbb{K}$ where $\MM\left(n,d\right)$ denotes
the cost of multiplying two polynomial matrices with dimension $n$
and degree $d$. Here an input shift is balanced when $\max(\vec{s})-\min(\vec{s})\in O(m\sigma/n)$.
%Here $O^{\sim}$
%is just $O$ with log factors omitted and 
%$\MM\left(n,d\right)$ denotes
%the cost of multiplying two polynomial matrices with dimension $n$
%and degree $d$. 
This extends earlier work of Storjohann which only determines a subset
of an order basis that is within a specified degree bound $\delta$
using $O^{\sim}(\MM(n,\delta))$ field operations for $\delta\ge\lceil m\sigma/n\rceil$.

While the first algorithm addresses the case when the column degrees
of a complete order basis are unbalanced given a balanced input shift,
it is not efficient in the case when a unbalanced shift results in
the row degrees also becoming unbalanced. %The column degrees of a complete basis may be unbalanced, which is
%a major issue we address in the first algorithm. When the input shift
%is unbalanced, the row degrees of the basis can also be unbalanced
%in addition to the unbalanced column degrees. For this, we present
We present a second algorithm which balances the high degree rows
and computes an order basis using $O^{\sim}(\MM(n,\lceil m\sigma/n\rceil))$
field operations in the case that the shift is unbalanced but satisfies
the condition $\sum_{i=1}^{n}(\max(\vec{s})-\vec{s}_{i})\le m\sigma$.%
\begin{comment}
Every problem with any unbalanced shift can be in fact reduced to
a problem with a shift that satisfying this condition if the degrees
of a resulting order basis is known. 
\end{comment}
{} %
\begin{comment}
Many unbalanced shift problems can be in fact converted to problems
satisfying this condition. 
\end{comment}
{} This condition essentially allows us to locate those high degree
rows that need to be balanced. %
\begin{comment}
In more general unbalanced shift cases, this algorithm may not work
well directly since we do not know in advance which are the high degree
rows need to be balanced. But it may work efficiently if we have an
effective way of estimating the resulting row degrees. 
\end{comment}
{} 
\end{abstract}
\end{frontmatter}

\section{Introduction}

%
\begin{comment}
remove all soft O notation and always write logs explicitly? 
\end{comment}
{}Let $\mathbf{F}\in\mathbb{K}\left[\left[x\right]\right]^{m\times n}$
be a matrix of power series over a field $\mathbb{K}$ with $m\le n$.
Given a non-negative integer $\sigma$, we say a vector $\mathbf{p}\in\mathbb{K}\left[x\right]^{n\times1}$
of polynomials gives an \emph{order} $\sigma$ approximation of $\mathbf{F}$,
or $\mathbf{p}$ has order $\left(\mathbf{F},\sigma\right)$, if \[
\mathbf{F}\cdot\mathbf{p}\equiv\mathbf{0}\mod x^{\sigma},\]
 that is, the first $\sigma$ terms of $\mathbf{F}\cdot\mathbf{p}$
are zero. Historically such problems date back to their use in Hermite's
proof of the transcendence of $e$ in 1873. In 1893 Pad�, a student
of Hermite, formalized the concepts introduced by Hermite and defined
what is now known as Hermite-Pad� approximants (where $m=1$), Pad�
approximants (where $m=1,n=2$) and simultaneous Pad� approximants
(where $\mathbf{F}$ has a special structure). Such rational approximations
also specified degree constraints on the polynomials $\mathbf{p}$
and had their order conditions related to these degree constraints.
%Pad\'e also introduced the notion of a Pad\'e
%table for the case $m=1,n=2$ which described all Pad\'e approximants 
%in terms of $2$ nearby approximants. 
Additional order problems include vector and matrix versions of rational
approximation, partial realizations of matrix sequences and vector
rational reconstruction just to name a few (cf. the references in
\citet{BL1997}). As an example, the factorization of differential
operators algorithm of \citet{vanHoeij} makes use of vector Hermite-Pad�
approximation to reconstruct differential factorizations over rational
functions from factorizations of differential operators over power
series domains.

The set of all such order $\left(\mathbf{F},\sigma\right)$ approximations
forms a module over $\mathbb{K}\left[x\right]$. An {\em order basis}
- or minimal approximant basis or $\sigma$-basis - is a basis of
this module having a type of minimal degree property (called reduced
order basis in \citep{BL1997}). The minimal degree property parameterizes
solutions to an order problem by the degrees of the columns of the
order basis. In the case of rational approximation, order bases can
be viewed as a natural generalization of the Pad� table of a power
series \citep{gravesmorris} since they are able to describe {\em
all} solutions to such problems given particular degree bounds \citep{BL1997}.
Order bases are used in such diverse applications as the inversion
of structured matrices \citep{La92}, normal forms of matrix polynomials
\citep{BLV:1999,BLV:jsc06}, and other important problems in matrix
polynomial arithmetic including matrix inversion, determinant and
nullspace computation \citep{Giorgi2003,storjohann-villard:2005}.
In our case we also allow the minimal degree property to include a
shift $\vec{s}$. Such a shift is important, for example, for matrix
normal form problems \citep{BLV:1999,BLV:jsc06}.

In this paper we focus on the efficient computation of order basis.
Algorithms for fast computation of order basis include that of \citet{BeLa94}
which converts the matrix problem into a vector problem of higher
order (which they called the Power Hermite-Pad� problem). Their divide
and conquer algorithm has complexity of $O^{\sim}(n^{2}m\sigma+nm^{2}\sigma)$
field operations. As usual, the soft-$O$ notation $O^{\sim}$ is
simply Big-$O$ with log factors omitted. By working more directly
on the input $m\times n$ input matrix, \citet{Giorgi2003} give a
divide and conquer method with cost $O\left(\MM\left(n,\sigma\right)\log\sigma\right)\subset O^{\sim}\left(\MM\left(n,\sigma\right)\right)$
arithmetic operations, where $\MM\left(n,\sigma\right)\in O^{\sim}\left(n^{\omega}\sigma\right)$
denotes the cost of multiplying two polynomial matrices with dimension
$n$ and degree $\sigma$. Their method is very efficient if $m$
is close to the size of $n$ but can be improved if $m$ is small.

In a novel construction, \citet{Storjohann:2006} effectively reverses
the approach of Beckermann and Labahn. Namely, rather than convert
a high dimension matrix order problem into a lower dimension vector
problem of higher order, Storjohann converts a low dimension problem
to a high dimension problem with lower order. For example, computing
an order basis for a $1\times n$ vector input $\mathbf{f}$ and order
$\sigma$ can be converted to a problem of order basis computation
with an $O\left(n\right)\times O\left(n\right)$ input matrix and
an order $O\left(\left\lceil \sigma/n\right\rceil \right)$. Combining
this conversion with the method of Giorgi et al. can then be used
effectively for problems with small row dimensions to achieve a cost
of $O^{\sim}\left(\MM\left(n,\left\lceil m\sigma/n\right\rceil \right)\right)$.

However, while order bases of the original problem can have degree
up to $\sigma$, the nature of Storjohann's conversion limits the
degree of an order basis of the converted problem to $O\left(\left\lceil m\sigma/n\right\rceil \right)$
in order to be computationally efficient. In other words, this approach
does not in general compute a complete order basis. Rather, in order
to achieve efficiency, it only computes a partial order basis containing
basis elements with degrees within $O\left(\left\lceil m\sigma/n\right\rceil \right)$,
referred to by Storjohann as a {\em minbasis}. Fast methods for
computing a minbasis are particularly useful for certain problems,
for example, in the case of inversion of structured block matrices
where one needs only precisely the minbasis \citep{La92}. However,
in other applications, such as those arising in matrix polynomial
arithmetic, one needs a complete basis which specifies all solutions
of a given order, not just those within a particular degree bound
(cf. \citet{BL1997}).

In this paper we present two algorithms which compute an entire order
basis with a cost of $O^{\sim}\left(\MM\left(n,\left\lceil m\sigma/n\right\rceil \right)\right)$
field operations. This work extends the previous results first reported
in \citet{za2009}. The two algorithms differ depending on the nature
of the degree shift required for the reduced order basis. In the first
case we use a transformation that can be considered as an extension
of Storjohann's transformation. This new transformation provides a
way to extend the results from one transformed problem to another
transformed problem of a higher degree. This enables us to use an
idea from the null space basis algorithm found in \citep{storjohann-villard:2005}
in order to achieve efficient computation. At each iteration, basis
elements within a specified degree bound are computed via a Storjohann
transformed problem. Then the partial result is used to simplify the
next Storjohann transformed problem of a higher degree, allowing basis
elements within a higher degree bound to be computed efficiently.
This is repeated until all basis elements are computed.

In order to compute an order basis efficiently, the first algorithm
requires that the degree shifts are balanced. In the case where the
shift is not balanced, the row degrees of the basis can also become
unbalanced in addition to the unbalanced column degrees. We give a
second algorithm that balances the high degree rows and uses $O^{\sim}(\MM(n,\lceil m\sigma/n\rceil))$
field operations when the shift $\vec{s}$ is unbalanced but satisfies
the condition $\sum_{i=1}^{n}(\max(\vec{s})-\vec{s}_{i})\le m\sigma$.
This condition essentially allows us to locate the high degree unbalanced
rows that need to be balanced. %
\begin{comment}
In fact, every problem with any unbalanced shift can be reformulated
to a problem with a shift satisfying this condition if the degrees
of the resulting order basis is known. 
\end{comment}
{} The algorithm converts a problem of unbalanced shift to one with
balanced shift, based on a second idea from \citep{Storjohann:2006}.
Then the first algorithm is used to efficiently compute the elements
of an order basis whose shifted degrees exceed a specified parameter.
The problem is then reduced to one where we remove the computed elements.
This results in a new problem with smaller dimension and higher degree.
The same process is repeated again on this new problem in order to
compute the elements with the next highest shifted degrees.

The remaining paper is structured as follows. Basic definitions and
properties of order bases are given in the next section. \prettyref{sec:transform}
provides an extension to Storjohann's transformation to allow higher
degree basis elements to be computed. Based on this new transformation,
\prettyref{sec:Order-Basis-Computation} establishes a link between
two Storjohann transformed problems of different degrees, from which
an recursive method and then an iterative algorithm are derived. The
time complexity is analyzed in the next section. After this, \prettyref{sec:Unbalanced-Shift}
describes an algorithm which handles problems with a type of unbalanced
shift. This is followed by a conclusion along with a description for
topics for future research. 



\section{Preliminaries}

\label{sec:Background}

The computational cost in this paper is analysed by bounding the number
of arithmetic operations (additions, subtractions, multiplications,
and divisions) in the coefficient field $\mathbb{K}$ on an algebraic
random access machine. We use $\MM(n,d)$ to denote the cost of multiplying
two polynomial matrices with dimension $n$ and degree $d$, and $\M(n)$
to denote the cost of multiplying two polynomials with degree $d$.
We define $\M'(d)=d\log d\log\log d$, then $\M'(ab)\subset O\left(\M'(a)\M'(b)\right)$
and $\M'(t)\in O(n^{\omega-1})$. We take $\MM(n,d)\in O\left(n^{\omega}\M(d)\right)\subset O(n^{\omega}\M'(d))$,
where the multiplication exponent $\omega$ is assumed to satisfy
$2<\omega\le3$. We refer to the book by \citet{vonzurgathen} for
more details and reference about the cost of polynomial multiplication
and matrix multiplication.

In this section, we provide some of the background needed in order
to understand the basic concepts and tools needed for order basis
computation. This includes basic definitions and a look at the size
of the input and the output for computing such bases.%
\begin{comment}
, which provide lower bounds for the computational cost 
\end{comment}
{} The challenges of balancing input and handling unbalanced output
are discussed along with the techniques which we plan to use to overcome
the difficulties. We review the construction by \citet{Storjohann:2006}
which transforms the inputs to those having dimensions and degree
balance better suited for fast computation and discuss an idea from
\citet{storjohann-villard:2005} for handling the case where the output
degree is unbalanced.


\subsection{Order Basis}

Let $\mathbb{K}$ be a field, $\mathbf{F}\in\mathbb{K}\left[\left[x\right]\right]^{m\times n}$
a matrix of power series and $\vec{\sigma}=\left[\sigma_{1},\dots,\sigma_{m}\right]$
a vector of non-negative integers. 
\begin{defn}
A vector of polynomials $\mathbf{p}\in\mathbb{K}\left[x\right]^{n\times1}$
has \emph{order} $\left(\mathbf{F},\vec{\sigma}\right)$ (or \emph{order}
$\vec{\sigma}$ with respect to $\mathbf{F}$) if $\mathbf{F}\cdot\mathbf{p}\equiv\mathbf{0}\mod x^{\vec{\sigma}}$,
that is, \[
\mathbf{F}\cdot\mathbf{p}=x^{\vec{\sigma}}\mathbf{r}=\begin{bmatrix}x^{\sigma_{1}}\\
 & \ddots\\
 &  & x^{\sigma_{m}}\end{bmatrix}\mathbf{r}\]
 for some $\mathbf{r}\in\mathbb{K}\left[\left[x\right]\right]^{m\times1}$.
If $\vec{\sigma}=\left[\sigma,\dots,\sigma\right]$ is uniform, then
we say that $\mathbf{p}$ has order $\left(\mathbf{F},\sigma\right).$
%
\begin{comment}
The vector of power series $\mathbf{r}$ is called the order $\left(\mathbf{F},\sigma\right)$-residual
of \textbf{$\mathbf{p}$}. 
\end{comment}
{} The set of all order $\left(\mathbf{F},\vec{\sigma}\right)$ vectors
is a $\mathbb{K}\left[x\right]$-module denoted by $\left\langle \left(\mathbf{F},\vec{\sigma}\right)\right\rangle $. 
\end{defn}
An order basis for $\mathbf{F}$ and $\vec{\sigma}$ is simply a basis
for the module $\left\langle \left(\mathbf{F},\vec{\sigma}\right)\right\rangle $.
In this paper we compute those order bases having a type of minimality
degree condition (also referred to as a reduced order basis in \citep{BL1997}).
While minimality is often given in terms of the degrees alone it is
sometimes important to consider this in terms of shifted degrees \citep{BLV:jsc06}.

The shifted column degree of a column polynomial vector $\mathbf{p}$
with shift $\vec{s}=\left[s_{1},\dots,s_{n}\right]\in\mathbb{Z}^{n}$
is given by \[
\deg_{\vec{s}}\mathbf{p}=\max_{1\le i\le n}[\deg p^{\left(i\right)}+s_{i}]=\deg(x^{\vec{s}}\cdot\mathbf{p}).\]
 We call this the \emph{$\vec{s}$-column degree}, or simply the \emph{$\vec{s}$-degree}
of $\mathbf{p}$. A shifted column degree defined this way is equivalent
to the notion of \emph{defect} commonly used in the literature. Our
definition of $\vec{s}$-degree is also equivalent to the notion of
$\mathbf{H}$-degree from \citep{BL1997} for $\mathbf{H}=x^{\vec{s}}$.
As in the uniform shift case, we say a matrix is \emph{$\vec{s}$-column
reduced} or \emph{$\vec{s}$-reduced} if its $\vec{s}$-degrees cannot
be decreased by unimodular column operations. More precisely, if $\mathbf{P}$
is a $\vec{s}$-column reduced and $[d_{1},\dots,d_{n}]$ are the
$\vec{s}$-degrees of columns of $\mathbf{P}$ sorted in nondecreasing
order, then $[d_{1},\dots,d_{n}]$ is lexicographically minimal among
all matrices right equivalent to $\mathbf{P}$. Note that a matrix
$\mathbf{P}$ is $\vec{s}$-column reduced if and only if $x^{\vec{s}}\cdot\mathbf{P}$
is column reduced. Similarly, $\mathbf{P}$ is in $\vec{s}$-Popov
form if $x^{\vec{s}}\cdot\mathbf{P}$ is in Popov form \citep{BLV:1999,BLV:jsc06}. 

An \emph{order basis} \citep{BeLa94,BL1997} $\mathbf{P}$ of $\mathbf{F}$
with order $\vec{\sigma}$ and shift $\vec{s}$, or simply an $\left(\mathbf{F},\vec{\sigma},\vec{s}\right)$-basis,
is a basis for the module $\left\langle \left(\mathbf{F},\vec{\sigma}\right)\right\rangle $
%
\begin{comment}
\[
\left\langle \left(\mathbf{F},\vec{\sigma}\right)\right\rangle =\{\mathbf{p}\in\mathbb{K}\left[x\right]^{n\times1}\|\mathbf{F}\cdot\mathbf{p}=x^{\vec{\sigma}}\mathbf{r},\mathbf{r}\in\mathbb{K}[[x]]^{m\times1}\}\]

\end{comment}
{} having minimal $\vec{s}$-column degrees. If $\vec{\sigma}=\left[\sigma,\dots,\sigma\right]$
are constant vectors then we simply write $\left(\mathbf{F},\sigma,\vec{s}\right)$-basis.
The precise definition of an $\left(\mathbf{F},\vec{\sigma},\vec{s}\right)$-basis
is as follows.
\begin{defn}
A polynomial matrix $\mathbf{P}$ is an order basis of $\mathbf{F}$
of order $\sigma$ and shift $\vec{s}$, denoted by $\left(\mathbf{F},\vec{\sigma},\vec{s}\right)$-basis,
if the following properties hold:
\begin{enumerate}
\item $\mathbf{P}$ is a nonsigular matrix of dimension $n$.
\item $\mathbf{P}$ is $\vec{s}$-column reduced. 
\item $\mathbf{P}$ has order $\left(\mathbf{F},\vec{\sigma}\right)$ (or
equivalently, each column of $\mathbf{P}$ is in $\left\langle (\mathbf{F},\vec{\sigma})\right\rangle $). 
\item Any $\mathbf{q}\in\left\langle \left(\mathbf{F},\vec{\sigma}\right)\right\rangle $
can be expressed as a linear combination of the columns of $\mathbf{P}$,
given by $\mathbf{P}^{-1}\mathbf{q}$. 
\end{enumerate}
\end{defn}
%
\begin{comment}
Note that the module $\left\langle \left(\mathbf{F},\vec{\sigma}\right)\right\rangle $
does not depend on the shift $\vec{s}$. 
\end{comment}
{}

Although we allow different orders for each row in this definition,
we focus on order basis computation problems having uniform order.
However special cases of non-uniform order problems are still needed
in our analysis. We also assume $m\le n$ for simplicity. The case
of $m>n$ can be transformed to the case of $m\le n$ by compression
\citep{storjohann-villard:2005}. We further assume, without any loss
of generality, that $n/m$ and $\sigma$ are powers of two. This can
be achieved by padding zero rows to the input matrix and multiplying
it by some power of $x$.

From \citep{BL1997} we have the following lemma. 
\begin{lem}
\label{lem:orderBasisProperty} 

\label{lem:orderBasisEquivalence}The following are equivalent for
a polynomial matrix \textbf{$\mathbf{P}$}: 
\begin{enumerate}
\item $\mathbf{P}$ is a $\left(\mathbf{F},\vec{\sigma},\vec{s}\right)$-basis. 
\item $\mathbf{P}$ is comprised of a set of $n$ minimal $\vec{s}$-degree
polynomial vectors that are linearly independent and each having order
$\left(\mathbf{F},\vec{\sigma}\right)$. 
\item \label{enu:reduced+generator}$\mathbf{P}$ does not contain a zero
column, has order $\left(\mathbf{F},\vec{\sigma}\right)$, is $\vec{s}$-column
reduced, and any $\mathbf{q}\in\left\langle \left(\mathbf{F},\vec{\sigma}\right)\right\rangle $
can be expressed as a linear combination of the columns of $\mathbf{P}$. 
\end{enumerate}
\end{lem}
In some cases an entire order basis is unnecessary and instead one
looks for a minimal basis that generates only the elements of $\left\langle \left(\mathbf{F},\vec{\sigma}\right)\right\rangle $
with $\vec{s}$-degrees bounded by a given $\delta$. Such a minimal
basis is a partial $\left(\mathbf{F},\vec{\sigma},\vec{s}\right)$-basis
comprised of elements of a $\left(\mathbf{F},\vec{\sigma},\vec{s}\right)$-basis
with $\vec{s}$-degrees bounded by $\delta$. This is called a \emph{minbasis}
in \citet{Storjohann:2006}. 
\begin{defn}
\label{def:genset} Let $\left\langle \left(\mathbf{F},\vec{\sigma},\vec{s}\right)\right\rangle _{\delta}\subset\left\langle \left(\mathbf{F},\vec{\sigma}\right)\right\rangle $
denote the set of order $\left(\mathbf{F},\vec{\sigma}\right)$ polynomial
vectors with $\vec{s}$-degree bounded by $\delta$. A $\left(\mathbf{F},\vec{\sigma},\vec{s}\right)_{\delta}$-basis
is a polynomial matrix $\mathbf{P}$ not containing a zero column
and satisfying: 
\begin{enumerate}
\item $\mathbf{P}$ has order $\left(\mathbf{F},\vec{\sigma}\right).$ 
\item Any element of $\left\langle \left(\mathbf{F},\vec{\sigma},\vec{s}\right)\right\rangle _{\delta}$
can be expressed as a linear combination of the columns of $\mathbf{P}$. 
\item $\mathbf{P}$ is $\vec{s}$-column reduced. 
\end{enumerate}
\end{defn}
%
\begin{comment}
As before, the linear combination here is in fact unique. 
\end{comment}
{}A $\left(\mathbf{F},\vec{\sigma},\vec{s}\right)_{\delta}$-basis is,
in general, not square unless $\delta$ is large enough to contain
all $n$ basis elements in which case it is a complete $\left(\mathbf{F},\vec{\sigma},\vec{s}\right)$-basis.


\subsection{Balancing Input with Storjohann's Transformation}

\label{sub:storjohannTransformation}

For computing a $\left(\mathbf{F},\sigma,\vec{s}\right)$-basis with
input matrix $\mathbf{F}\in\mathbb{K}\left[\left[x\right]\right]^{m\times n}$,
shift $\vec{s}$ and order $\sigma$ one can view $\mathbf{F}$ as
a polynomial matrix with degree $\sigma-1$, as higher order terms
are not needed in the computation. As such the total input size of
an order basis problem is $mn\sigma$ coefficients. One can apply
the method of \citet{Giorgi2003} directly, which gives a cost of
\begin{align*}
\sum_{i=0}^{\log\sigma}2^{i}\MM(n,2^{-i}\sigma)= & \sum_{i=0}^{\log\sigma}2^{-i}\sigma\MM(n,2^{i})\\
\subset & O\left(\sum_{i=0}^{\log\sigma}2^{-i}n^{\omega}\sigma2^{i}\log2^{i}\log\log2^{i}\right)\\
= & O\left(n^{\omega}\sigma\sum_{i=0}^{\log\sigma}i\log i\right)\\
\subset & O\left(n^{\omega}\sigma\sum_{i=0}^{\log\sigma}\log\sigma\log\log\sigma\right)\\
= & O\left(n^{\omega}\sigma\log^{2}\sigma\log\log\sigma\right)=O(n^{\omega}\M'(\sigma)\log\sigma)\end{align*}
field operations, close to the cost of multiplying two matrices with
dimension $n$ and degree $\sigma$. Note that this cost is independent
of the degree shift. This is very efficient if $m\in\Theta\left(n\right)$.
However, for small $m$, say $m=1$ as in Hermite Pad� approximation,
the total input size is only $n\sigma$ coefficients. Matrix multiplication
cannot be used effectively on a such vector input.

\citet{storjohann:2006} provides a novel way to transform an order
basis problem with small row dimension to a problem with higher row
dimension and possibly lower degree to take advantage of \citet{Giorgi2003}'s
algorithm. We provide a quick overview of a slightly modified version
of Storjohann's method. Our small modification allows a nonuniform
degree shift for the input and provides a slightly simpler degree
shift, degree, and order for the transformed problem. The proof of
its correctness is provided in \prettyref{sec:transform}. In order
to compute a $\left(\mathbf{F},\sigma,\vec{s}\right)$-basis, assuming
without loss of generality that $\min\left(\vec{s}\right)=0$, we
first write \[
\mathbf{F}=\mathbf{F}_{0}+\mathbf{F}_{1}x^{\delta}+\mathbf{F}_{2}x^{2\delta}+\cdots+\mathbf{F}_{l}x^{l\delta},\]
 with $\deg\mathbf{F}_{i}<\delta$%
\begin{comment}
I used $\deg\mathbf{F}_{i}\le\delta-1$ before, but the reviewer suggested
changing to $\deg\mathbf{F}_{i}<\delta$ and said it's slightly easier
to read. A reason to use $\le$ is to make it consistent with the
definition of minbasis. For example, $\left(\mathbf{F},\sigma,\vec{s}\right)_{\delta-1}$-basis
indicates that the degree bound is $\delta-1$. I still prefer to
use $\le$. 
\end{comment}
{} for a positive integer $\delta$, and where we assume (again without
loss of generality) that $\sigma=\left(l+1\right)\delta$. Set \[
{\bar{\mathbf{F}}}=\left[\begin{array}{c|cccc}
\mathbf{F}_{0}+\mathbf{F}_{1}x^{\delta} & \mathbf{0}_{m} & \mathbf{0}_{m} & \cdots & \mathbf{0}_{m}\\
\hline \mathbf{F}_{1}+\mathbf{F}_{2}x^{\delta} & \mathbf{I}_{m} & \mathbf{0}_{m}\\
\mathbf{F}_{2}+\mathbf{F}_{3}x^{\delta} & \mathbf{0}_{m} & \mathbf{I}_{m}\\
\vdots &  &  & \ddots\\
\mathbf{F}_{l-1}+\mathbf{F}_{l}x^{\delta} &  &  &  & \mathbf{I}_{m}\end{array}\right]_{ml\times(n+m(l-1))}.\]
 On the left side of $\bar{\mathbf{F}}$, each block $\mathbf{F}_{i}+\mathbf{F}_{i+1}x^{\delta}$
has dimension $m\times n$. On the right side, there are $l\times(l-1)$
blocks of $\mathbf{0}_{m}$'s or $\mathbf{I}_{m}$'s each having dimension
$m\times m$. The overall dimension of $\bar{\mathbf{F}}$ is $ml\times(n+m(l-1))$.
Set $\vec{s'}=\left[\vec{s},0,\dots,0\right]$ ($\vec{s}$ followed
by $m\left(l-1\right)$ $0$'s). A $({\bar{\mathbf{F}}},2\delta,\vec{s'})$-basis
can then be computed by the method of Giorgi et al. with a cost of
$O^{\sim}\left(n^{\omega}\delta\right)$ for $\delta\ge\left\lceil m\sigma/n\right\rceil $.
This transformation of Storjohann can be viewed as a partial linearization
of the original problem, where $\bar{\mathbf{F}}$ is analogous to
the coefficient matrix of $\mathbf{F}$. Note that $\bar{\mathbf{F}}$
has $l$ block rows each containing $m$ rows. We continue to use
each block row to represent $m$ rows for the remainder of the paper.

Clearly a $(\bar{\mathbf{F}},2\delta,\vec{s'})$-basis $\bar{\mathbf{P}}$
of the transformed problem is not a $\left(\mathbf{F},\sigma,\vec{s}\right)$-basis
of the original problem, as $\bar{\mathbf{P}}$ has a higher dimension
and lower degree. However, the first $n$ rows of the $(\bar{\mathbf{F}},2\delta,\vec{s'})_{\delta-1}$-basis
contained in $\bar{\mathbf{P}}$ is a $\left(\mathbf{F},\sigma,\vec{s}\right)_{\delta-1}$-basis.

Note that there is no need to set the degree parameter $\delta$ to
less than $\left\lceil m\sigma/n\right\rceil $, as this produces
fewer basis elements without a better cost. The lowest cost is achieved
when $\bar{\mathbf{F}}$ is close to square so matrix multiplication
can be used most effectively. This requires the number of block rows
$l$ of $\bar{\mathbf{F}}$ to be close to $n/m$, which requires
$\delta=\Theta\left(\left\lceil m\sigma/n\right\rceil \right)$. Recall
that $mn\sigma$ is the total size of the original $m\times n$ input
matrix $\mathbf{F}$, hence $d=mn\sigma/n^{2}=m\sigma/n$ is the average
degree of each entry of $\mathbf{F}$ if the $m$ rows of $\mathbf{F}$
are spread out over $n$ rows. Choosing $\delta=\Theta\left(\left\lceil d\right\rceil \right)$,
the cost of computing a $({\bar{\mathbf{F}}},2\delta,\vec{s'})$-basis
is then $O^{\sim}\left(n^{\omega}\left\lceil d\right\rceil \right)=O^{\sim}\left(n^{\omega}\left\lceil m\sigma/n\right\rceil \right)$.
The ceiling function here is used to take care of the case of $m\sigma<n$.
For the remainder of the paper, we assume that $m\sigma\ge n$ in
order to avoid the need for the ceiling function and so simplify the
presentation. % avoid this case for simplicity. 
Together with the assumption that $\sigma$ and $n/m$ are both powers
of two, $m\sigma/n$ is then always a positive integer in this paper.

%Let us now look at a concrete example that illustrate Storjohann's method. 

\begin{exmp}
\label{exm:StorjohannTransformation}Let $\mathbb{K}=\mathbb{Z}_{2}$,
$\sigma=8$, $\delta=2$ and \[
\mathbf{F}=[x+x^{2}+x^{3}+x^{4}+x^{5}+x^{6},~1+x+x^{5}+x^{6}+x^{7},~1+x^{2}+x^{4}+x^{5}+x^{6}+x^{7},~1+x+x^{3}+x^{7}]\]
 a vector of size $1\times4$. Then \[
\bar{\mathbf{F}}=\left[{\begin{array}{cccc|cc}
x+x^{2}+x^{3} & 1+x & 1+x^{2} & 1+x+x^{2} & ~~0~ & ~0~\\
\hline 1+x+x^{2}+x^{3} & x^{3} & 1+x^{2}+x^{3} & x & ~~1~ & ~0~\\
1+x+x^{2} & x+x^{2}+x^{3} & 1+x+x^{2}+x^{3} & x^{3} & ~~0~ & ~1~\end{array}}\right]_{3\times6}\]
 and a $\left(\bar{\mathbf{F}},4,\vec{0}\right)$-basis is given by
\[
\bar{\mathbf{P}}=\left[{\begin{array}{cc|cccc}
~1~ & ~x~ & 1 & x^{2}+x^{3} & 0 & x+x^{2}+x^{3}\\
0 & 1 & 0 & x^{2} & x^{2}+x^{3} & 0\\
1 & 1+x & x+x^{2} & x^{2} & x^{2} & x^{2}\\
1 & 0 & 0 & 0 & 0 & 0\\
\hline 0 & 1 & 1 & 0 & x^{2} & x+x^{2}+x^{3}\\
0 & 1 & 1+x^{2} & 0 & x^{2} & x+x^{2}\end{array}}\right].\]
 The first two columns of $\bar{\mathbf{P}}$ have degree less than
$2$, hence its top left $4\times2$ submatrix is a $\left(\mathbf{F},8,\vec{0}\right)_{1}$-basis.
This is a low degree part of the\textbf{ $(\mathbf{F},8,\vec{0})$}-basis\[
\mathbf{P}=\begin{bmatrix}1 & x & 1 & x^{2}\\
0 & 1 & x^{2}+x^{3} & 0\\
1 & 1+x & x & x^{3}+x^{4}\\
1 & 0 & 0 & 0\end{bmatrix}.\]
 Note that if $\delta$ is set to $\sigma/2=4$, then the transformed
problem is the same as the original problem. 
\end{exmp}

\subsection{\label{sub:Unbalanced-Output}Unbalanced Output }

Storjohann's transformation can be used to efficiently compute a $\left(\mathbf{F},\sigma,\vec{s}\right)_{\delta-1}$-basis
if the degree parameter $\delta$ is close to the average degree $d=m\sigma/n$.
However, if $\delta$ is large, say $\delta=\Theta\left(\sigma\right)$,
or if we want to compute a complete $\left(\mathbf{F},\sigma,\vec{s}\right)$-basis,
then the current analysis for the computation still gives the cost
estimate of  $O^{\sim}\left(n^{\omega}\sigma\right)$.

The underlying difficulty with computing a complete order basis is
that the basis can have degree up to $\sigma$. As the output of this
problem has dimension $n\times n$ and degree up to $\Theta\left(\sigma\right)$,
this may seem to suggest $O^{\sim}\left(n^{\omega}\sigma\right)$
is about the best that can be done. However, the total size of the
output, that is, the total number of coefficients of all $n^{2}$
polynomial entries can still be bounded by $O\left(mn\sigma\right)$,
the same as the size of the input. This gives some hope for a more
efficient method. 
\begin{lem}
\label{lem:size}Let $\vec{t}$ be the $\vec{s}$-column degrees of
a $\left(\mathbf{F},\sigma,\vec{s}\right)$-basis. Then $\sum_{i}\left(\vec{t}_{i}-\vec{s}_{i}\right)~\le~m\sigma$%
\begin{comment}
 and $\max_{i}\left(\vec{t}_{i}-\vec{s}_{i}\right)\le\sigma$
\end{comment}
{}\textup{}%
\begin{comment}
need to permute the columns to put the pivots on the diagonal.
\end{comment}
{}. In addition, the total size of any $\left(\mathbf{F},\sigma,\vec{s}\right)$-basis
in $\vec{s}$-Popov form is bounded by $nm\sigma$. \end{lem}
\begin{pf}
This can be shown by considering the sizes of the pivots in the iterative
order basis computation given in \citep{BeLa94,Giorgi2003}. %
\begin{comment}
Consider the degree of pivot entries in the iterative order basis
computation from \citet{BeLa94}. Starting at order zero, for each
order increasing by one, the sum of the degrees of all the pivot entries
is increased by at most $m$. Hence the sum is at most $m\sigma$
after $\sigma$ increases for an order $\sigma$ order basis. Now
consider any particular column of the basis. Each entry in the column
has degree bounded by the corresponding pivot entry of the row, so
the total degree of the entries in the column is bounded by $m\sigma$.
Therefore, the size of any order $\sigma$ basis with $n$ columns
is bounded by $nm\sigma$ coefficients. 
\end{comment}
{} 
\end{pf}
%
\begin{comment}
As a result, the average degree of the entries of the output matrix
can be also bounded by $d=m\sigma/n$. 
\end{comment}
{}

Let us now look at the average column degree of the output. In the
first part of this paper, we assumed, without loss of generality,
that $\min\left(\vec{s}\right)=0$ so $\deg\mathbf{q}\le\deg_{\vec{s}}\mathbf{q}$
for any $\mathbf{q}\in\mathbb{K}\left[x\right]^{n}$. The situation
is simpler if the shift $\vec{s}$ is uniform since then $\sum_{i}\vec{t}_{i}\le m\sigma$
by \prettyref{lem:size} and the average column degree is therefore
bounded by $d=m\sigma/n$. In the first part of this paper, we consider
a slightly more general case, when the shift $\vec{s}$ is \emph{balanced},
which is defined as follows. 
\begin{defn}
A shift $\vec{s}$ is balanced if $\max\vec{s}-\min\vec{s}\in O(d)=O(m\sigma/n)$. 
\end{defn}
By assuming $\min\vec{s}=0$, $\vec{s}$ is balanced if $\max\vec{s}\in O(d)$.
In this case, \prettyref{lem:size} implies $\sum_{i}\left(\vec{t}_{i}\right)\le m\sigma+\sum_{i}\left(\vec{s}_{i}\right)\in O\left(m\sigma+nd\right)=O\left(m\sigma\right)$.
Hence the average column degree of the output basis remains $O\left(d\right)$.

%From the iterative algorithms for computing order basis, computing
%a order basis to order $\sigma$ requires up to $\sigma$ iterations,
%each iteration increases the sum of column degrees of the order basis
%by at most $m$. Therefore, the sum of column degrees of an order
%$\sigma$ order basis is at most $m\sigma$.


%
\begin{comment}
In fact, if $\mathbf{F}\left(0\right)$ is full rank, the sum of column
degrees of an order $\sigma$ order basis is exactly $m\sigma$, as
we need exactly $\sigma$ iterations, each increases the sum of the
column degrees by exactly $m$. 
\end{comment}
{}

The fact that a $\left(\mathbf{F},\sigma,\vec{s}\right)$-basis can
have degree up to $\sigma$ while its average column degree is $O\left(m\sigma/n\right)$
implies that an order basis can have quite unbalanced column degrees,
especially if $m$ is small. A similar problem with unbalanced output
is encountered in null space basis computation. \citet{storjohann-villard:2005}
deal with this in the following way.

Let $d$ be the average column degree of the output. Set the degree
parameter $\delta$ to twice that of $d$. This allows one to compute
at least half the columns of a basis (since the number of columns
with degree at least $\delta$ must be at most a half of the total
number of columns). One can then simplify the problem, so that the
computed basis elements are completely removed from the problem. This
reduces the dimension of the problem by at least a factor of $2$.
One then doubles the degree bound $\delta$ in order to have at least
$3/4$ of the basis elements computed. Repeating this, at iteration
$i$, at most $1/2^{i}$ of the basis elements are remaining. Therefore,
no more than $\log n$ iterations are needed to compute all basis
elements.

%We now have two ideas for efficient computation. Storjohann's transformation
%for balancing the input and Storjohann and Villard's method for handling
%unbalanced output. 
%In this paper, we discuss a way to compute order basis involving the
%ideas of \citep{storjohann-villard:2005} and \citep{Storjohann:2006}. 




\section{Extending Storjohann's Transformation}

\label{sec:transform}

%
\begin{comment}
notations: 
\begin{itemize}
\item Matrices

\begin{itemize}
\item Polynomial matrices: bold capital $\mathbf{A,B,F,G,P,R}$

\begin{itemize}
\item $\mathbf{F}$: input matrix,

\begin{itemize}
\item $\bar{\mathbf{F}}$: Storjohann transformed matrix, 
\item $\check{\mathbf{F}}$: the transformation that extends $\bar{\mathbf{F}}$. 
\end{itemize}
\item $\mathbf{G}$: input matrix for the second subproblem. 
\end{itemize}
\item scalar matrices: non-bold Capital, used to distinguish coefficient
matrix 
\item Identity matrix: $\mathbf{I}$ this is to be consistent with polynomial
matrices, as identity is considered as an important element of polynomial
matrices. 
\end{itemize}
\item order

\begin{itemize}
\item $\sigma,\omega$ 
\end{itemize}
\item vectors

\begin{itemize}
\item polynomial vectors: $\mathbf{p,q,r,s,t}$ 
\item $\mathbf{p}$: element in basis $\mathbf{P}$ 
\item $\mathbf{q}:$ some test element in $\left\langle \left(\mathbf{F},\sigma\right)\right\rangle $ 
\item $\bar{\mathbf{t}}$: test element in $\left\langle \left(\check{\mathbf{F}},\vec{\omega}\right)\right\rangle $ 
\end{itemize}
\item shifts:

\begin{itemize}
\item $\vec{e}$: uniform 0 shift used for transformation. 
\item $\vec{s}$: original input shift 
\item $\vec{t}$: output shifted degrees of the current basis, with result
from previous subproblems included. i.e., $\left[\vec{s},0,\dots,0\right]$-degrees
of the current basis. 
\item $\vec{b},\vec{a}$: input shift and output shifted degrees for each
reduced subproblem that calls OrderBasis. In the iterative computational
process, input shift $\vec{b}^{\left(i-1\right)}$ corresponds to
input shift $\vec{a}^{\left(i\right)}$. The high degree entries of
$\vec{a}^{\left(i\right)}$ at iteration $i$ is then used as the
input shift $\vec{b}^{\left(i\right)}$ for iteration $i$. 
\end{itemize}
\item dimension:

\begin{itemize}
\item $m,n$: row, column dimension 
\item $k$: number of columns with $\left[\vec{s},0,\dots,0\right]$-degrees
less than degree bound 
\item $l$: number of block rows 
\end{itemize}
\item degrees, order:

\begin{itemize}
\item $\sigma$: order 
\item $\delta$: degree bound 
\item $d$: average degree (=$m\sigma/n$) 
\end{itemize}
\end{itemize}

\end{comment}
{}

In this section, we introduce a transformation that can be viewed
as an extension of Storjohann's transformation which allows for computation
of a full, rather than partial, order basis. More generally (as discussed
in the next section) this transformation provides a link between two
Storjohann transformed problems constructed using different degree
parameters. For easier understanding, we first focus on a particular
case of this transformation in \prettyref{sub:particularCase} and
then generalize this in \prettyref{sub:generalTransform}.


\subsection{\label{sub:particularCase}A Particular Case}

Consider the problem of computing a $\left(\mathbf{F},\sigma,\vec{s}\right)$-basis.
We assume $\sigma=4\delta$ %
\begin{comment}
(The results below hold for any positive integer great than one, but
4 is used for simplicity. ) 
\end{comment}
{}for a positive integer $\delta$ and write the input matrix polynomial
as $\mathbf{F}=\mathbf{F}_{0}+\mathbf{F}_{1}x^{\delta}+\mathbf{F}_{2}x^{2\delta}+\mathbf{F}_{3}x^{3\delta}$
with $\deg\mathbf{F}_{i}<\delta$.%
\begin{comment}
again, changed from $\deg\mathbf{F}_{i}\le\delta-1$ to $\deg\mathbf{F}_{i}<\delta$
even though I prefer $\le$ 
\end{comment}
{} In the following, we show that computing a $\left(\mathbf{F},\sigma,\vec{s}\right)$-basis
can be done by computing a $(\mathbf{F}',\vec{\omega},\vec{s'})$-basis
where \begin{equation}
\mathbf{F}'=\left[\begin{array}{cc}
\mathbf{F} & \mathbf{0}\\
\mathbf{F}'_{21} & \mathbf{F}'_{22}\end{array}\right]=\left[\begin{array}{c|cr}
\mathbf{F}_{0}+\mathbf{F}_{1}x^{\delta}+\mathbf{F}_{2}x^{2\delta}+\mathbf{F}_{3}x^{3\delta} & \mathbf{0} & \mathbf{0}\\
\hline \mathbf{F}_{1}+\mathbf{F}_{2}x^{\delta} & \mathbf{I}_{m} & \mathbf{0}\\
\mathbf{F}_{2}+\mathbf{F}_{3}x^{\delta} & \mathbf{0} & \mathbf{I}_{m}\end{array}\right]\label{eq:extendedSTransformTop}\end{equation}
 with order $\vec{\omega}=\left[4\delta,\dots,4\delta,2\delta,\dots,2\delta\right]$
(with $m$ $4\delta$'s and $2m$ $2\delta$'s) and degree shift $\vec{s'}=\left[\vec{s},e,\dots,e\right]$
(with $2m$ $e$'s), where $e$ is an integer less than or equal to
$1$. %
\begin{comment}
actually true for $e\le\min\vec{s}+1$. keeping this $e$ helps to
show that this hold true in more general situations, while Storjohann
only used $e=1$. it can be negative as well. 
\end{comment}
{} We set $e$ to $0$ in this paper for simplicity%
\footnote{Storjohann used $e=1$ in \citep{Storjohann:2006}. All results in
this section still hold for any other $e\le1.$%
}. %
\begin{comment}
In fact, it is quite easy to construct a $(\check{\mathbf{F}},\vec{\omega},\vec{s'})$-basis
from a $\left(\mathbf{F},\sigma,\vec{s}\right)$-basis, as we show
later in \prettyref{lem:FtoAbasis}. However, it requires more work
to extract a $\left(\mathbf{F},\sigma,\vec{s}\right)$-basis from
a $(\check{\mathbf{F}},\vec{\omega},\vec{s'})$-basis, which is addressed
eventually in \prettyref{cor:extractingFbasisFromGbasis}. Note that
although constructing a $(\check{\mathbf{F}},\vec{\omega},\vec{s'})$-basis
from a $\left(\mathbf{F},\sigma,\vec{s}\right)$-basis in \prettyref{lem:FtoAbasis}
is the reverse of what we want, this well-formed $(\check{\mathbf{F}},\vec{\omega},\vec{s'})$-basis
restricts the elements of $\langle(\check{\mathbf{F}},\vec{\omega},\vec{s'})\rangle$
to a simple form shown in \prettyref{cor:FtauBasisForm}, which helps
to establish a close correspondence between a $(\check{\mathbf{F}},\vec{\omega},\vec{s'})$-basis
and a $\left(\mathbf{F},\sigma,\vec{s}\right)$-basis in \prettyref{lem:2delta-1Basis},
\prettyref{lem:2deltaBasis}, and \prettyref{thm:mainTheorem}. 
\end{comment}
{}

We first look at the correspondence bettween the elements of $\left\langle \left(\mathbf{F},\sigma,\vec{s}\right)\right\rangle _{\tau}$
and the elements of $\langle(\mathbf{F}',\vec{\omega},\vec{s'})\rangle_{\tau}$
in \prettyref{lem:qToBqOrder} to \prettyref{lem:bqToqOrder}. The
correspondence between $\left(\mathbf{F},\sigma,\vec{s}\right)$-bases
and $(\mathbf{F}',\vec{\omega},\vec{s'})$-bases is then considered
in \prettyref{cor:FtauBasisForm} to \prettyref{thm:mainTheorem}.

Let \[
\mathbf{B}=\left[\begin{array}{c}
\mathbf{I}_{n}\\
x^{-\delta}\mathbf{F}_{0}\\
x^{-2\delta}\left(\mathbf{F}_{0}+\mathbf{F}_{1}x^{\delta}\right)\end{array}\right].\]

\begin{lem}
\label{lem:qToBqOrder}If $\mathbf{q}\in\left\langle \left(\mathbf{F},\sigma\right)\right\rangle $,
then $\mathbf{B}\mathbf{q}\in\langle(\mathbf{F}'\vec{,\omega})\rangle$.\end{lem}
\begin{pf}
The lemma follows from \[
\mathbf{F}'\mathbf{B}\mathbf{q}=\left[\begin{array}{r}
\mathbf{F}_{0}+\mathbf{F}_{1}x^{\delta}+\mathbf{F}_{2}x^{2\delta}+\mathbf{F}_{3}x^{3\delta}\\
\mathbf{F}_{0}x^{-\delta}+\mathbf{F}_{1}+\mathbf{F}_{2}x^{\delta}\\
\mathbf{F}_{0}x^{-2\delta}+\mathbf{F}_{1}x^{-\delta}+\mathbf{F}_{2}+\mathbf{F}_{3}x^{\delta}\end{array}\right]\mathbf{q}\equiv\mathbf{0}\mod x^{\vec{\omega}}.\]
 Note that the bottom rows of $\mathbf{B}$ may not be polynomials.
However, $\mathbf{B}\mathbf{q}$ is a polynomial vector since $\mathbf{q}\in\left\langle \left(\mathbf{F},\sigma\right)\right\rangle $
implies $\mathbf{q}\in\left\langle \left(\mathbf{F}_{0},\delta\right)\right\rangle $
and $\mathbf{q}\in\left\langle \left(\mathbf{F}_{0}+\mathbf{F}_{1}x^{\delta},2\delta\right)\right\rangle $. 
\end{pf}
\noindent The following lemma shows that the condition $e\le1$ forces
$\deg_{\vec{s'}}\mathbf{B}\mathbf{q}$ to be determined by $\mathbf{q}$. 
\begin{lem}
\label{lem:qToBqDegree}If $\mathbf{q}\in\left\langle \left(\mathbf{F},\sigma,\vec{s}\right)\right\rangle _{\tau}$
for any degree bound $\tau\in\mathbb{Z}$, then $\deg_{\vec{s'}}\mathbf{B}\mathbf{q}=\deg_{\vec{s}}\mathbf{q}$.\end{lem}
\begin{pf}
By assumption $s_{i}\ge0$, so $\deg\mathbf{q}\le\deg_{\vec{s}}\mathbf{q}.$
Now consider the degree of the bottom $2m$ entries, $\mathbf{q}_{2},\mathbf{q}_{3}$,
of \[
\begin{bmatrix}\mathbf{q}\\
\mathbf{q}_{2}\\
\mathbf{q}_{3}\end{bmatrix}=\mathbf{B}\mathbf{q}=\left[\begin{array}{r}
\mathbf{q}\\
x^{-\delta}\mathbf{F}_{0}\cdot\mathbf{q}\\
x^{-2\delta}\left(\mathbf{F}_{0}+\mathbf{F}_{1}x^{\delta}\right)\cdot\mathbf{q}\end{array}\right].\]
 Our goal is to show $\deg_{\vec{e}}\left[\mathbf{q}_{2}^{T},\mathbf{q}_{3}^{T}\right]^{T}\le\deg_{\vec{s}}\mathbf{q}$.
Note that \[
\deg\mathbf{q}_{2}=\deg\left(\mathbf{F}_{0}\mathbf{q}/x^{\delta}\right)\le\deg\mathbf{q}+\delta-1-\delta\le\deg_{\vec{s}}\mathbf{q}-1,\]
 and similarly $\deg\mathbf{q}_{3}\le\deg_{\vec{s}}\mathbf{q}-1$.
Therefore \[
\deg_{\vec{e}}\begin{bmatrix}\mathbf{q}_{2}\\
\mathbf{q}_{3}\end{bmatrix}=\deg\begin{bmatrix}\mathbf{q}_{2}\\
\mathbf{q}_{3}\end{bmatrix}+e\le\deg_{\vec{s}}\mathbf{q}-1+e\le\deg_{\vec{s}}\mathbf{q}.\]
 \end{pf}
\begin{cor}
\label{cor:sToBs}If $\mathbf{q}\in\left\langle \left(\mathbf{F},\sigma,\vec{s}\right)\right\rangle _{\tau}$
for any degree bound $\tau\in\mathbb{Z}$ , then\textup{ }$\mathbf{B}\mathbf{q}\in\langle(\mathbf{F}',\vec{\omega},\vec{s'})\rangle_{\tau}$\textup{.} 
\end{cor}
%\begin{pf}
%This follows from \prettyref{lem:qToBqOrder} and \prettyref{lem:qToBqDegree}.\end{pf}

\begin{cor}
\label{cor:linearCombinationOfFirstnRows}Let $\bar{\mathbf{S}}_{\tau}$
be a $(\mathbf{F}',\vec{\omega},\vec{s'})_{\tau}$-basis and $\mathbf{S}_{\tau}$
be the top $n$ rows of $\bar{\mathbf{S}}_{\tau}$ for any bound $\tau\in\mathbb{Z}$.
Then any $\mathbf{q}\in\left\langle \left(\mathbf{F},\sigma,\vec{s}\right)\right\rangle _{\tau}$
is a linear combination of the columns of $\mathbf{S}_{\tau}$.\end{cor}
\begin{pf}
By \prettyref{cor:sToBs}, $\mathbf{B}\mathbf{q}\in\langle(\mathbf{F}',\vec{\omega},\vec{s'})\rangle_{\tau}$,
and so is a linear combination of columns of $\bar{\mathbf{S}}_{\tau}$.
That is, there exists a polynomial vector $\mathbf{u}$ such that
$\mathbf{B}\mathbf{q}=\bar{\mathbf{S}}_{\tau}\mathbf{u}$. This remains
true if we restrict the equation to the top $n$ rows, that is, $\mathbf{q}=\left[\mathbf{I}_{n},\mathbf{0}\right]\mathbf{B}\mathbf{q}=\left[\mathbf{I}_{n},\mathbf{0}\right]\bar{\mathbf{S}}_{\tau}\mathbf{u}=\mathbf{S}_{\tau}\mathbf{u}$.\end{pf}
\begin{lem}
\label{lem:bqToqOrder}Let $\bar{\mathbf{q}}\in\langle(\mathbf{F}',\vec{\omega},\vec{s'})\rangle_{\tau}$
for any degree bound $\tau\in\mathbb{Z}$, and $\mathbf{q}_{1}$ the
first $n$ entries of $\mathbf{\bar{q}}$. Then $\mathbf{q}_{1}\in\left\langle \left(\mathbf{F},\sigma,\vec{s}\right)\right\rangle _{\tau}$.\end{lem}
\begin{pf}
The top rows of \[
\mathbf{F}'\mathbf{q}=\left[\begin{array}{cc}
\mathbf{F} & \mathbf{0}\\
\mathbf{F}'_{21} & \mathbf{F}'_{22}\end{array}\right]\left[\begin{array}{c}
\mathbf{q}_{1}\\
\mathbf{q}_{2}\end{array}\right]=\begin{bmatrix}\mathbf{F}\mathbf{q}_{1}\\
\mathbf{F}'_{21}\mathbf{q}_{1}+\mathbf{F}'_{22}\mathbf{q}_{2}\end{bmatrix}\equiv\mathbf{0}\mod x^{\vec{\omega}}\]
 give $\mathbf{F}\mathbf{q}_{1}\equiv\mathbf{0}\mod x^{\sigma}$. 
\end{pf}
The next lemma shows a $(\mathbf{F}',\vec{\omega},\vec{s'})$-basis
can be constructed from a $\left(\mathbf{F},\sigma,\vec{s}\right)$-basis.
This well-formed $(\mathbf{F}',\vec{\omega},\vec{s'})$-basis restricts
the elements of $\langle(\mathbf{F}',\vec{\omega},\vec{s'})\rangle$
to a simple form shown in \prettyref{cor:FtauBasisForm}. This in
turn helps to establish a close correspondence between a $(\mathbf{F}',\vec{\omega},\vec{s'})$-basis
and a $\left(\mathbf{F},\sigma,\vec{s}\right)$-basis in \prettyref{lem:2delta-1Basis},
\prettyref{lem:2deltaBasis}, and \prettyref{thm:mainTheorem}. 
\begin{lem}
\label{lem:FtoAbasis}If $\mathbf{P}$ is a $\left(\mathbf{F},\sigma,\vec{s}\right)$-basis,
then \begin{eqnarray*}
\bar{\mathbf{T}} & = & \left[\mathbf{B}\mathbf{P}~\begin{array}{|c}
\mathbf{0}_{n\times2m}\\
x^{2\delta}\mathbf{I}_{2m}\end{array}\right]=\left[\begin{array}{r|cc}
\mathbf{P} & \mathbf{0}_{n\times m} & \mathbf{0}_{n\times m}\\
\hline x^{-\delta}\mathbf{F}_{0}\cdot\mathbf{P} & x^{2\delta}\mathbf{I}_{m} & \mathbf{0}_{m}\\
x^{-2\delta}\left(\mathbf{F}_{0}+\mathbf{F}_{1}x^{\delta}\right)\cdot\mathbf{P} & \mathbf{0}_{m} & x^{2\delta}\mathbf{I}_{m}\end{array}\right]\end{eqnarray*}
 is a $(\mathbf{F}',\vec{\omega},\vec{s'})$-basis.\end{lem}
\begin{pf}
By \prettyref{lem:qToBqOrder}, $\bar{\mathbf{T}}$ has order $(\mathbf{F}',\vec{\omega})$
and is $\vec{s'}$-column reduced since $\mathbf{P}$ dominates the
$\vec{s'}$-degrees of $\bar{\mathbf{T}}$ on the left side by \prettyref{lem:qToBqDegree}.
It remains to show that any $\bar{\mathbf{q}}\in\langle(\mathbf{F}',\vec{\omega},\vec{s'})\rangle$
is a linear combination of the columns of $\mathbf{\bar{\mathbf{T}}}$.

Let $\mathbf{q}$ be the top $n$ entries of $\bar{\mathbf{q}}$.
Then by \prettyref{lem:bqToqOrder}, $\mathbf{q}\in\left\langle \left(\mathbf{F},\sigma,\vec{s}\right)\right\rangle $,
hence is a linear combination of the columns of $\mathbf{P}$, that
is $\mathbf{q}=\mathbf{P}\mathbf{u}$ with $\mathbf{u}=\mathbf{P}^{-1}\mathbf{q}\in\mathbb{K}\left[x\right]^{n\times1}$.
Subtracting the contribution of $\mathbf{P}$ from $\bar{\mathbf{q}}$,
we get \[
\mathbf{q}'=\bar{\mathbf{q}}-\mathbf{B}\mathbf{P}\mathbf{u}=\bar{\mathbf{q}}-\mathbf{B}\mathbf{q}=\left[\begin{array}{c}
\mathbf{0}\\
\mathbf{v}\end{array}\right],\]
 which is still in $\langle(\mathbf{F}',\vec{\omega},\vec{s'})\rangle$,
that is, \[
\mathbf{F}'\mathbf{q}'=\begin{bmatrix}\mathbf{0}\\
\mathbf{I}_{2m}\mathbf{v}\end{bmatrix}\equiv\mathbf{0}\mod x^{\vec{\omega}}.\]
 This forces $\mathbf{v}$ to be a linear combination of the columns
of $x^{2\delta}\mathbf{I}_{2m}$, the bottom right submatrix of $\bar{\mathbf{T}}$.
Now $\bar{\mathbf{q}}=\bar{\mathbf{T}}\left[\mathbf{u}^{T},\mathbf{v}^{T}\right]^{T}$
as required.\end{pf}
\begin{cor}
\label{cor:FtauBasisForm}Let $\tau\in\mathbb{Z}$ be any degree bound.
Let $\mathbf{P}_{\tau}\in\mathbb{K}\left[x\right]^{n\times t}$ be
a $\left(\mathbf{F},\sigma,\vec{s}\right)_{\tau}$-basis. Let $\bar{\mathbf{q}}\in\langle(\mathbf{F}',\vec{\omega},\vec{s'})\rangle_{\tau}$
and $\mathbf{q}$ be the top $n$ entries of $\bar{\mathbf{q}}$.
Then $\bar{\mathbf{q}}$ must have the form \[
\bar{\mathbf{q}}=\mathbf{B}\mathbf{P}_{\tau}\mathbf{u}+x^{2\delta}\begin{bmatrix}\mathbf{0}\\
\mathbf{v}\end{bmatrix}=\mathbf{B}\mathbf{q}+x^{2\delta}\begin{bmatrix}\mathbf{0}\\
\mathbf{v}\end{bmatrix}\]
 for some polynomial vector $\mathbf{u}\in\mathbb{K}\left[x\right]^{t\times1}$
and $\mathbf{v}\in\mathbb{K}\left[x\right]^{2m\times1}$. In particular,
if $\deg_{\vec{s'}}\bar{\mathbf{q}}<2\delta$, then $\bar{\mathbf{q}}=\mathbf{B}\mathbf{P}_{\tau}\mathbf{u}=\mathbf{B}\mathbf{q}$. \end{cor}
\begin{pf}
This follows directly from \prettyref{lem:FtoAbasis} with $\vec{s'}$-degrees
restricted to $\tau$.\end{pf}
\begin{lem}
\label{lem:2delta-1Basis}If $\bar{\mathbf{S}}^{\left(1\right)}$
is a $(\check{\mathbf{F}},\vec{\omega},\vec{s'})_{2\delta-1}$-basis,
then the matrix $\mathbf{S}^{\left(1\right)}$ consisting of its first
$n$ rows is a $\left(\mathbf{F},\sigma,\vec{s}\right)_{2\delta-1}$-basis.\end{lem}
\begin{pf}
By \prettyref{lem:bqToqOrder}, $\mathbf{S}^{\left(1\right)}$ has
order $\left(\mathbf{F},\sigma\right)$. By \prettyref{cor:linearCombinationOfFirstnRows},
any $\mathbf{q}\in\left\langle \left(\mathbf{F},\sigma,\vec{s}\right)\right\rangle _{2\delta-1}$
is a linear combination of $\mathbf{S}^{\left(1\right)}$. It remains
to show that $\mathbf{S}^{\left(1\right)}$ is $\vec{s}$-column reduced.

By \prettyref{cor:FtauBasisForm}, $\bar{\mathbf{S}}^{\left(1\right)}=\mathbf{B}\mathbf{S}^{\left(1\right)}$,
and by \prettyref{lem:bqToqOrder}, the columns of $\mathbf{S}^{\left(1\right)}$
are in $\left\langle \left(\mathbf{F},\sigma,\vec{s}\right)\right\rangle _{2\delta-1}$.
Thus, by \prettyref{lem:qToBqDegree}, $\mathbf{S}^{\left(1\right)}$
determines the $\vec{s'}$-column degrees of $\mathbf{S}^{\left(1\right)}$.
Therefore, $\bar{\mathbf{S}}^{\left(1\right)}$ being $\vec{s'}$-column
reduced implies that $\mathbf{S}^{\left(1\right)}$ is $\vec{s}$-column
reduced.\end{pf}
\begin{lem}
\label{lem:2deltaBasis} Let $\bar{\mathbf{S}}^{\left(12\right)}=[\bar{\mathbf{S}}^{\left(1\right)},\bar{\mathbf{S}}^{\left(2\right)}]$
be a $(\mathbf{F}',\vec{\omega},\vec{s'})_{2\delta}$-basis, with
$\deg_{\vec{s'}}\bar{\mathbf{S}}^{\left(1\right)}\le2\delta-1$ and
$\deg_{\vec{s'}}\bar{\mathbf{S}}^{\left(2\right)}=2\delta$, and $\mathbf{S}^{\left(12\right)},\mathbf{S}^{\left(1\right)},\mathbf{S}^{\left(2\right)}$
the first $n$ rows of $\bar{\mathbf{S}}^{\left(12\right)},\bar{\mathbf{S}}^{\left(1\right)},\bar{\mathbf{S}}^{\left(2\right)}$,
respectively. Let $I$ be the column rank profile (the lexicographically
smallest sequence of column indices that indicates a full column rank
submatrix) of $\mathbf{S}^{\left(12\right)}$. %which contains all columns of $\mathbf{S}^{\left(1\right)}$ by \prettyref{lem:2delta-1Basis}.
Then the submatrix\textbf{ $\mathbf{S}_{I}^{\left(12\right)}$ }comprised
of the columns of $\mathbf{S}^{\left(12\right)}$ indexed by $I$
is a $\left(\mathbf{F},\sigma,\vec{s}\right)_{2\delta}$-basis. \end{lem}
\begin{pf}
Consider doing $\vec{s}$-column reduction on $\mathbf{S}^{\left(12\right)}$.
From \prettyref{lem:2delta-1Basis}, we know that $\mathbf{S}^{\left(1\right)}$
is a $\left(\mathbf{F},\sigma,\vec{s}\right)_{2\delta-1}$-basis.
Therefore, only $\mathbf{S}^{\left(2\right)}$ may be $\vec{s}$-reduced.
If a column $\mathbf{c}$ of $\mathbf{S}^{\left(2\right)}$ can be
further $\vec{s}$-reduced, then it becomes an element of $\left\langle \left(\mathbf{F},\sigma,\vec{s}\right)\right\rangle _{2\delta-1}$,
which is generated by $\mathbf{S}^{\left(1\right)}$. Thus $\mathbf{c}$
must be reduced to zero by $\mathbf{S}^{\left(1\right)}$. The only
nonzero columns of $\mathbf{S}^{\left(12\right)}$ remaining after
$\vec{s}$-column reduction are therefore the columns that cannot
be $\vec{s}$-reduced. Hence $\mathbf{S}^{\left(12\right)}$ $\vec{s}$-reduces
to \textbf{$\mathbf{S}_{I}^{\left(12\right)}$}. In addition, \textbf{$\mathbf{S}_{I}^{\left(12\right)}$}
has order $\left(\mathbf{F},\sigma\right)$ as $\mathbf{S}^{\left(12\right)}$
has order $\left(\mathbf{F},\sigma\right)$ by \prettyref{lem:bqToqOrder}.
From \prettyref{cor:linearCombinationOfFirstnRows} any $\mathbf{q}\in\left\langle \left(\mathbf{F},\sigma,\vec{s}\right)\right\rangle _{2\delta}$
is a linear combination of $\mathbf{S}^{\left(12\right)}$ and hence
is also a linear combination of $\mathbf{S}_{I}^{\left(12\right)}$. 
\end{pf}
To extract $\mathbf{S}_{I}^{\left(12\right)}$ from $\mathbf{S}^{\left(12\right)}$,
note that doing $\vec{s}$-column reduction on $\mathbf{S}^{\left(12\right)}$
is equivalent to the more familiar problem of doing column reduction
on $x^{\vec{s}}\mathbf{S}^{\left(12\right)}$. As $\mathbf{S}^{\left(12\right)}$
$\vec{s}$-column reduces to \textbf{$\mathbf{S}_{I}^{\left(12\right)}$},
this corresponds to determining the column rank profile of the\emph{
leading column coefficient matrix }of\emph{ }\textbf{$x^{\vec{s}}\mathbf{S}^{\left(12\right)}$}%
\begin{comment}
\emph{ }$S^{\left(12\right)}=\lcoeff(x^{\vec{s}}\cdot\mathbf{S}^{\left(12\right)})$ 
\end{comment}
{}\emph{. }Recall that the leading column coefficient matrix of a matrix
$\mathbf{A}=\left[\mathbf{a}_{1},\dots,\mathbf{a}_{k}\right]$ used
for column reduction is\emph{ \begin{eqnarray*}
\lcoeff\left(\mathbf{A}\right) & = & \left[\lcoeff\left(\mathbf{a}_{1}\right),\dots,\lcoeff\left(\mathbf{a}_{k}\right)\right]\\
 & = & \left[\coeff\left(\mathbf{a}_{1},\deg\left(\mathbf{a}_{1}\right)\right),\dots,\coeff\left(\mathbf{a}_{k},\deg\left(\mathbf{a}_{k}\right)\right)\right].\end{eqnarray*}
 }The column rank profile of $\lcoeff(x^{\vec{s}}\mathbf{S}^{\left(12\right)})$
can be determined by (the transposed version of) LSP factorization
\citep{IbarraMH82}, which factorizes $\lcoeff(x^{\vec{s}}\mathbf{S}^{\left(12\right)})=PSU$
as the product of a permutation matrix $P$, a matrix $S$ with its
nonzero columns forming a lower triangular submatrix, and an upper
triangular matrix $U$ with $1$'s on the diagonal. The indices, $I$,
of the nonzero columns of $S$ then give $\mathbf{S}_{I}^{\left(12\right)}$
in $\mathbf{S}^{\left(12\right)}$. 
\begin{thm}
\label{thm:mainTheorem}Let $\bar{\mathbf{S}}=[\bar{\mathbf{S}}^{\left(12\right)},\bar{\mathbf{S}}^{\left(3\right)}]$
be a $(\mathbf{F}',\vec{\omega},\vec{s'})$-basis, with $\deg_{\vec{s'}}\bar{\mathbf{S}}^{\left(12\right)}\le2\delta$
and $\deg_{\vec{s'}}\bar{\mathbf{S}}^{\left(3\right)}\ge2\delta+1$,
and $\mathbf{S},\mathbf{S}^{\left(12\right)},\mathbf{S}^{\left(3\right)}$
the first $n$ rows of $\bar{\mathbf{S}},\bar{\mathbf{S}}^{\left(12\right)},\bar{\mathbf{S}}^{\left(3\right)}$,
respectively. If $I$ is the column rank profile of $\mathbf{S}^{\left(12\right)}$,
then the submatrix \textbf{ $[\mathbf{S}_{I}^{\left(12\right)},\mathbf{S}^{\left(3\right)}]$
}of $\mathbf{S}$ is a $\left(\mathbf{F},\sigma,\vec{s}\right)$-basis. \end{thm}
\begin{pf}
By \prettyref{lem:bqToqOrder}, $\mathbf{S}$ has order $\left(\mathbf{F},\sigma\right)$,
and so $[\mathbf{S}_{I}^{\left(12\right)},\mathbf{S}^{\left(3\right)}]$
also has order $\left(\mathbf{F},\sigma\right)$. By \prettyref{cor:linearCombinationOfFirstnRows},
any $\mathbf{q}\in\left\langle \left(\mathbf{F},\sigma,\vec{s}\right)\right\rangle $
is a linear combination of the columns of $\mathbf{S}$, and so $\mathbf{q}$
is also a linear combination of the columns of $[\mathbf{S}_{I}^{\left(12\right)},\mathbf{S}^{\left(3\right)}]$.
It only remains to show that $[\mathbf{S}_{I}^{\left(12\right)},\mathbf{S}^{\left(3\right)}]$
is $\vec{s}$-column reduced.

Let $\mathbf{P}$ be a $\left(\mathbf{F},\sigma,\vec{s}\right)$-basis
and $\mathbf{\bar{\mathbf{T}}}$ be the $(\mathbf{F}',\vec{\omega},\vec{s'})$-basis
constructed from $\mathbf{P}$ as in \prettyref{lem:FtoAbasis}. Let
$\bar{\mathbf{T}}^{\left(3\right)}$ be the columns of $\bar{\mathbf{T}}$
with $\vec{s'}$-degrees greater than $2\delta,$ and $\mathbf{P}^{\left(3\right)}$
be the columns of $\mathbf{P}$ with $\vec{s}$-degrees greater than
$2\delta.$ Assume without loss of generality that $\mathbf{S},$
$\mathbf{P}$, and $\bar{\mathbf{T}}$ have their columns sorted according
to their $\vec{s}$-degrees and $\vec{s'}$-degrees, respectively.
Then $\deg_{\vec{s}}\mathbf{S}^{\left(3\right)}\le\deg_{\vec{s'}}\bar{\mathbf{S}}^{\left(3\right)}=\deg_{\vec{s'}}\bar{\mathbf{T}}^{\left(3\right)}=\deg_{\vec{s}}\mathbf{P}^{\left(3\right)}$.
Combining this with the $\vec{s}$-minimality of $\mathbf{S}_{I}^{\left(12\right)}$
from \prettyref{lem:2deltaBasis}, it follows that $\deg_{\vec{s}}[\mathbf{S}_{I}^{\left(12\right)},\mathbf{S}^{\left(3\right)}]\le\deg_{\vec{s}}\mathbf{P}$.
This combined with the fact that $[\mathbf{S}_{I}^{\left(12\right)},\mathbf{S}^{\left(3\right)}]$
still generates $\left\langle \left(\mathbf{F},\sigma,\vec{s}\right)\right\rangle $
implies that $\deg_{\vec{s}}[\mathbf{S}_{I}^{\left(12\right)},\mathbf{S}^{\left(3\right)}]=\deg_{\vec{s}}\mathbf{P}$.
Therefore, $[\mathbf{S}_{I}^{\left(12\right)},\mathbf{S}^{\left(3\right)}]$
is a $\left(\mathbf{F},\sigma,\vec{s}\right)$-basis. \end{pf}
\begin{cor}
\label{cor:extractingFbasisFromGbasis}Let $\bar{\mathbf{S}}$ be
a $(\mathbf{F}',\vec{\omega},\vec{s'})$-basis with its columns sorted
in an increasing order of their $\vec{s'}$ degrees, and $\mathbf{S}$
the first $n$ rows of $\bar{\mathbf{S}}$. If $J$ is the column
rank profile of $\lcoeff(x^{\vec{s}}\mathbf{S})$, then the submatrix
$\mathbf{S}_{J}$ of $\mathbf{S}$ indexed by $J$ is a $\left(\mathbf{F},\sigma,\vec{s}\right)$-basis.\end{cor}
\begin{pf}
This follows directly from \prettyref{thm:mainTheorem}. 
\end{pf}
This rank profile $J$ can be determined by LSP factorization on $\lcoeff(x^{\vec{s}}\cdot\mathbf{S}^{\left(12\right)})$.
%as discussed before.

\begin{exmp}
\label{exm:auxiliaryTransformation}For the problem in \prettyref{exm:StorjohannTransformation},
$\check{\mathbf{F}}$ is given by \begin{align*}
 & \left[{\begin{array}{cccccr}
x+x^{2}+x^{3}+x^{4}+x^{5}+x^{6} & \ 1+x+x^{5}+x^{6}+x^{7}\  & 1+x^{2}+x^{6}+x^{7} & \ 1+x+x^{3}+x^{7}\  & 0\  & 0\\
1+x+x^{2}+x^{3} & x^{3} & 1+x^{2}+x^{3} & x & 1\  & 0\\
1+x+x^{2} & x+x^{2}+x^{3} & 1+x+x^{2}+x^{3} & x^{3} & 0\  & 1\end{array}}\right],\end{align*}
 and a $\left(\mathbf{F}',\left[8,4,4\right],\vec{0}\right)$-basis
is given as \[
\left[\begin{array}{rccc|cc}
1 & x & 1 & x^{2} & x^{2}+x^{4} & 1+x^{2}+x^{3}+x^{4}\\
0 & 1 & x^{2}+x^{3} & 0 & x^{3} & 0\\
1 & 1+x & x & x^{3}+x^{4} & 0 & x+x^{2}+x^{3}\\
1 & 0 & 0 & 0 & 0 & 0\\
\hline 0 & 1 & 1+x^{2} & x^{2} & x^{2}+x^{3} & 1+x^{2}+x^{3}+x^{4}\\
0 & 1 & 1 & x^{2}+x^{4} & x^{2}+x^{3} & 1+x^{3}\end{array}\right].\]
 Column reduction on the top 4 rows gives the top left $4\times4$
submatrix, which is a \textbf{$(\mathbf{F},8,\vec{0})$}-basis. 
\end{exmp}
The following two lemmas verify Storjohann's result in the case of
degree parameter $\delta=\sigma/4$. More specifically, we show that
the matrix of the top $n$ rows of a $(\bar{\mathbf{F}},2\delta,\vec{s'})_{\delta-1}$-basis
is a $\left(\mathbf{F},\sigma,\vec{s}\right)_{\delta-1}$-basis, with
the transformed input matrix \begin{equation}
\bar{\mathbf{F}}=\left[\begin{array}{l|cr}
\mathbf{F}_{0}+\mathbf{F}_{1}x^{\delta} & \mathbf{0} & \mathbf{0}\\
\hline \mathbf{F}_{1}+\mathbf{F}_{2}x^{\delta} & \mathbf{I}_{m} & \mathbf{0}\\
\mathbf{F}_{2}+\mathbf{F}_{3}x^{\delta} & \mathbf{0} & \mathbf{I}_{m}\end{array}\right]\equiv\mathbf{F}'\mod x^{2\delta}.\label{eq:storjohannTransformation4parts}\end{equation}

\begin{lem}
\label{lem:A_delta-1Form}If $\bar{\mathbf{q}}\in\langle(\bar{\mathbf{F}},2\delta,\vec{s'})\rangle_{\delta-1}$
and $\mathbf{q}$ denotes the first $n$ entries of $\bar{\mathbf{q}}$,
then $\bar{\mathbf{q}}$ must have the form \[
\bar{\mathbf{q}}=\mathbf{B}\mathbf{q}=\left[\begin{array}{r}
\mathbf{q}\\
x^{-\delta}\mathbf{F}_{0}\cdot\mathbf{q}\\
x^{-2\delta}\left(\mathbf{F}_{0}+\mathbf{F}_{1}x^{\delta}\right)\cdot\mathbf{q}\end{array}\right]\]
 and $\mathbf{q}\in\left\langle \left(\mathbf{F},\sigma,\vec{s}\right)\right\rangle _{\delta-1}$.\end{lem}
\begin{pf}
Let $\mathbf{q},\mathbf{q}_{2},\mathbf{q}_{3}$ consist of the top
$n$ entries, middle $m$ entries, and bottom $m$ entries, respectively,
of $\mathbf{\bar{\mathbf{q}}}$ so that \begin{align}
\bar{\mathbf{F}}\bar{\mathbf{q}} & \equiv\left[\begin{array}{r}
\mathbf{F}_{0}\mathbf{q}+x^{\delta}\mathbf{F}_{1}\mathbf{q}\\
\mathbf{q}_{2}+\mathbf{F}_{1}\mathbf{q}+x^{\delta}\mathbf{F}_{2}\mathbf{q}\\
\mathbf{q}_{3}+\mathbf{F}_{2}\mathbf{q}+x^{\delta}\mathbf{F}_{3}\mathbf{q}\end{array}\right]\equiv\mathbf{0}\mod x^{2\delta}.\label{eq:Aq}\end{align}
 From the first and second block rows, we get $\mathbf{F}_{0}\mathbf{q}+x^{\delta}\mathbf{F}_{1}\mathbf{q}\equiv\mathbf{0}\mod x^{2\delta}$
and $\mathbf{q}_{2}+\mathbf{F}_{1}\mathbf{q}\equiv\mathbf{0}\mod x^{\delta}$,
which implies \begin{equation}
\mathbf{F}_{0}\mathbf{q}\equiv x^{\delta}\mathbf{q}_{2}\mod x^{2\delta}.\label{eq:q1q2}\end{equation}
 Similarly, from the second and the third rows, we get $\mathbf{q}_{2}+\mathbf{F}_{1}\mathbf{q}+x^{\delta}\mathbf{F}_{2}\mathbf{q}\equiv\mathbf{0}\mod x^{2\delta}$
and $\mathbf{q}_{3}+\mathbf{F}_{2}\mathbf{q}\equiv\mathbf{0}\mod x^{\delta}$,
which implies $\mathbf{q}_{2}+\mathbf{F}_{1}\mathbf{q}\equiv x^{\delta}\mathbf{q}_{3}\mod x^{2\delta}$.

Since $\deg\mathbf{q}\le\deg_{\vec{s}}\mathbf{q}=\delta-1$, we have
$\deg\mathbf{F}_{0}\mathbf{q}\le2\delta-2$, hence from \prettyref{eq:q1q2}
$\deg\mathbf{q}_{2}\le\delta-2$ and $\mathbf{q}_{2}x^{\delta}=\mathbf{F}_{0}\mathbf{q}$.
Similarly, $\deg\mathbf{q}_{3}\le\delta-2$ and $\mathbf{q}_{3}x^{2\delta}=\mathbf{q}_{2}x^{\delta}+\mathbf{F}_{1}\mathbf{q}x^{\delta}=\mathbf{F}_{0}\mathbf{q}+\mathbf{F}_{1}\mathbf{q}x^{\delta}$.
Substituting this to $\mathbf{F}\mathbf{q}=(\mathbf{F}_{0}\mathbf{q}+\mathbf{F}_{1}\mathbf{q}x^{\delta})+(\mathbf{F}_{2}\mathbf{q}x^{2\delta}+\mathbf{F}_{3}\mathbf{q}x^{3\delta})$,
we get $\mathbf{F}\mathbf{q}=\mathbf{q}_{3}x^{2\delta}+(\mathbf{F}_{2}\mathbf{q}x^{2\delta}+\mathbf{F}_{3}\mathbf{q}x^{3\delta})\equiv\mathbf{0}\mod x^{4\delta}$
using the bottom block row of \prettyref{eq:Aq}.\end{pf}
\begin{lem}
\label{lem:delta-1Basis} If $\bar{\mathbf{S}}_{\delta-1}$ is a $(\bar{\mathbf{F}},2\delta,\vec{s'})_{\delta-1}$-basis,
then the matrix of its first $n$ rows, $\mathbf{S}_{\delta-1}$,
is a $\left(\mathbf{F},\sigma,\vec{s}\right)_{\delta-1}$-basis.\end{lem}
\begin{pf}
By \prettyref{lem:A_delta-1Form}, $\mathbf{S}_{\delta-1}$ has order
$\left(\mathbf{F},\sigma\right)$. Following Lemmas \ref{lem:qToBqOrder}
and \ref{lem:qToBqDegree} and Corollaries \ref{cor:sToBs} and \ref{cor:linearCombinationOfFirstnRows}
(replacing $\vec{\omega}$ by $2\delta$), we conclude that any $\mathbf{q}\in\left\langle \left(\mathbf{F},\sigma,\vec{s}\right)\right\rangle _{\delta-1}$
is a linear combination of the columns of $\mathbf{S}_{\delta-1}$.
In addition, since $\bar{\mathbf{S}}_{\delta-1}=\mathbf{B}\mathbf{S}_{\delta-1}$
by \prettyref{lem:A_delta-1Form} and the columns of $\mathbf{S}_{\delta-1}$
are in $\left\langle \left(\mathbf{F},\sigma,\vec{s}\right)\right\rangle _{\delta-1}$,
it follows that $\mathbf{S}_{\delta-1}$ determines the $\vec{s'}$-column
degrees of $\bar{\mathbf{S}}_{\delta-1}$ by \prettyref{lem:qToBqDegree}.
Hence $\bar{\mathbf{S}}_{\delta-1}$ $\vec{s'}$-column reduced implies
that $\mathbf{S}_{\delta-1}$ is $\vec{s}$-column reduced. 
\end{pf}

\subsection{\label{sub:generalTransform}More General Results}

Let us now consider an immediate extension of the above results. Suppose
that instead of a $\left(\mathbf{F},\sigma,\vec{s}\right)$-basis
we now want to compute a $(\bar{\mathbf{F}}^{\left(i\right)},2\delta^{\left(i\right)},\vec{s}^{\left(i\right)})$-basis
with a Storjohann transformed input matrix \[
\bar{\mathbf{F}}^{\left(i\right)}=\left[\begin{array}{c|cccc}
\mathbf{F}_{0}+\mathbf{F}_{1}x^{\delta^{\left(i\right)}} & \mathbf{0}_{m} & \cdots & \mathbf{\cdots} & \mathbf{0}_{m}\\
\hline \mathbf{F}_{1}+\mathbf{F}_{2}x^{\delta^{\left(i\right)}} & \mathbf{I}_{m}\\
\mathbf{F}_{2}+\mathbf{F}_{3}x^{\delta^{\left(i\right)}} &  & \mathbf{I}_{m}\\
\vdots &  &  & \ddots\\
\mathbf{F}_{l^{\left(i\right)}-1}+\mathbf{F}_{l^{\left(i\right)}}x^{\delta^{\left(i\right)}} &  &  &  & \mathbf{I}_{m}\end{array}\right]_{ml^{(i)}\times(n+m(l^{(i)}-1))}\]
 constructed with degree parameter $\delta^{\left(i\right)}=2^{i}d$
for some integer $i$ between $2$%
\begin{comment}
the base case is $i=1$ and problem is not to be subdivided 
\end{comment}
{} and $\log\left(\sigma/d\right)-1$, and a shift $\vec{s}^{\left(i\right)}=[\vec{s},0,\dots,0]$
(with $m(l^{\left(i\right)}-1)$ 0's), where $l^{\left(i\right)}=\sigma/\delta^{\left(i\right)}-1$
is the number of block rows%
\footnote{Recall that $d=m\sigma/n$ is the average degree of the input matrix
$\mathbf{F}$ if we treat $\mathbf{F}$ as a square $n\times n$ matrix.
Also, $i$ starts at $2$ because $i=1$ is our base case in the computation
of an order basis, which may become more clear in the next section.
The base case can be computed efficiently using the method of Giorgi
et al. \citeyearpar{Giorgi2003} directly and does not require the
transformation discussed in this section.%
}. To apply a transformation analogous to \prettyref{eq:extendedSTransformTop},
we write each $\mathbf{F}_{j}=\mathbf{F}_{j0}+\mathbf{F}_{j1}\delta^{\left(i-1\right)}$
and set \begin{equation}
\mathbf{F}'^{\left(i\right)}=\left[\begin{array}{l|c}
\mathbf{F}_{00}+\mathbf{F}_{01}x^{\delta^{\left(i-1\right)}}+\mathbf{F}_{10}x^{2\delta^{\left(i-1\right)}}+\mathbf{F}_{11}x^{3\delta^{\left(i-1\right)}} & ~~\mathbf{0}~~\\
\hline \mathbf{F}_{01}+\mathbf{F}_{10}x^{\delta^{\left(i-1\right)}}\\
\mathbf{F}_{10}+\mathbf{F}_{11}x^{\delta^{\left(i-1\right)}}+\mathbf{F}_{20}x^{2\delta^{\left(i-1\right)}}+\mathbf{F}_{21}x^{3\delta^{\left(i-1\right)}}\\
\mathbf{F}_{11}+\mathbf{F}_{20}x^{\delta^{\left(i-1\right)}}\\
\vdots & ~~\mathbf{I}~~\\
\mathbf{F}_{\left(l^{\left(i\right)}-1\right)0}+\mathbf{F}_{\left(l^{\left(i\right)}-1\right)1}x^{\delta^{\left(i-1\right)}}+\mathbf{F}_{l^{\left(i\right)}0}x^{2\delta^{\left(i-1\right)}}+\mathbf{F}_{l^{\left(i\right)}1}x^{3\delta^{\left(i-1\right)}}\\
\mathbf{F}_{\left(l^{\left(i\right)}-1\right)1}+\mathbf{F}_{l^{\left(i\right)}0}x^{\delta^{\left(i-1\right)}}\\
\mathbf{F}_{l^{\left(i\right)}0}+\mathbf{F}_{l^{\left(i\right)}1}x^{\delta^{\left(i-1\right)}}\end{array}\right],\label{eq:extendedStorjohannTransform}\end{equation}
 %
\begin{comment}
This is not ideal, but no better idea. 
\end{comment}
{} and $\vec{\omega}^{\left(i\right)}=\left[\left[[2\delta^{\left(i\right)}]^{m},[\delta^{\left(i\right)}]^{m}\right]^{l^{\left(i\right)}},[\delta^{\left(i\right)}]^{m}\right]$,
where $\left[\circ\right]^{k}$ represents $\circ$ repeated $k$
times%
\begin{comment}
Not sure if using this notation is a good thing to do, but it saves
space and makes presentation easier 
\end{comment}
{}. The order entries $2\delta^{\left(i\right)}$, $\delta^{\left(i\right)}$
in $\vec{\omega}^{\left(i\right)}$ correspond to the degree $2\delta^{\left(i\right)}-1$,
degree $\delta^{\left(i\right)}-1$ rows in $\mathbf{F}'^{\left(i\right)}$
respectively. Let \[
\mathbf{E}^{\left(i\right)}=\left[\begin{array}{c||cc|cc|cc|cc||cc}
\mathbf{I}_{n} &  &  &  &  &  &  &  &  & \mathbf{0}_{n,m} & \mathbf{0}_{n,m}\\
\hline\hline  & \mathbf{0}_{m} & \mathbf{I}_{m} &  &  &  &  &  &  & \ \\
\hline  &  &  & \mathbf{0}_{m} & \mathbf{I}_{m} &  &  &  &  & \ \\
\hline  &  &  &  &  & \ddots & \ddots &  &  & \ \\
\hline  &  &  &  &  &  &  & \mathbf{0}_{m} & \mathbf{I}_{m}\end{array}\right]\]
 with $l^{\left(i\right)}-1$ blocks of $\left[\mathbf{0}_{m},\mathbf{I}_{m}\right]$
and hence an overall dimension of $(n+m(l^{\left(i\right)}-1))\times(n+m(l^{\left(i-1\right)}-1))$.
Thus $\mathbf{E}^{\left(i\right)}\mathbf{M}$ picks out from $\mathbf{M}$
the first $n$ rows and the even block rows from the remaining rows
except the last block row for a matrix $\mathbf{M}$ with $n+m(l^{\left(i-1\right)}-1)$
rows. In particular, if $i=\log\left(n/m\right)-1$, then $(\mathbf{F}'^{\left(i\right)},\vec{\omega}^{\left(i\right)},\vec{s}^{\left(i-1\right)})=(\mathbf{F}',\vec{\omega},\vec{s'})$,
which for $d=m\sigma/n$ gives the problem considered earlier in \prettyref{sub:particularCase},
and $\mathbf{E}^{\left(i\right)}=\left[\mathbf{I}_{n},\mathbf{0}_{n,m},\mathbf{0}_{n,m}\right]$
is used to select the top $n$ rows of a $(\mathbf{F}',\vec{\omega},\vec{s'})$-basis
for a $\left(\mathbf{F},\sigma,\vec{s}\right)$-basis to be extracted.

We can now state the analog of \prettyref{cor:extractingFbasisFromGbasis}: 
\begin{thm}
\label{thm:extractingOrderBasis}Let $\mathbf{S}'^{\left(i\right)}$
be a $(\mathbf{F}'^{\left(i\right)},\vec{\omega}^{\left(i\right)},\vec{s}^{\left(i-1\right)})$-basis
with its columns sorted in an increasing order of their $\vec{s}^{\left(i-1\right)}$
degrees. Let $\hat{\mathbf{S}}^{\left(i\right)}=\mathbf{E}^{\left(i\right)}\mathbf{S}'^{\left(i\right)}$.
Let $J$ be the column rank profile of $\lcoeff(x^{\vec{s}^{\left(i\right)}}\hat{\mathbf{S}}^{\left(i\right)})$.
Then $\hat{\mathbf{S}}_{J}^{\left(i\right)}$ is a $(\bar{\mathbf{F}}^{\left(i\right)},2\delta^{\left(i\right)},\vec{s}^{\left(i\right)})$-basis.\end{thm}
\begin{pf}
One can follow the same arguments used before from \prettyref{lem:qToBqOrder}
to \prettyref{cor:extractingFbasisFromGbasis}. Alternatively, this
can be derived from \prettyref{cor:extractingFbasisFromGbasis} by
noticing the redundant block rows that can be disregarded after applying
transformation \prettyref{eq:extendedSTransformTop} directly to the
input matrix $\bar{\mathbf{F}}^{\left(i\right)}$. 
\end{pf}
\prettyref{lem:delta-1Basis} can also be extended in the same way
to capture Storjohann's transformation with more general degree parameters: 
\begin{lem}
\label{lem:linkStorjohanTransform}If $\bar{\mathbf{P}}_{1}^{\left(i-1\right)}$
is a $(\bar{\mathbf{F}}^{\left(i-1\right)},2\delta^{\left(i-1\right)},\vec{s}^{\left(i-1\right)})_{\delta^{\left(i-1\right)}-1}$-basis,
then $\mathbf{E}^{\left(i\right)}\bar{\mathbf{P}}_{1}^{\left(i-1\right)}$
is a $(\bar{\mathbf{F}}^{\left(i\right)},2\delta^{\left(i\right)},\vec{s}^{\left(i\right)})_{\delta^{\left(i-1\right)}-1}$-basis
and the matrix of the top $n$ rows of $\bar{\mathbf{P}}_{1}^{\left(i-1\right)}$
is a $(\mathbf{F},\sigma,\vec{s})_{\delta^{\left(i-1\right)}-1}$-basis.\end{lem}
\begin{pf}
Again, this can be justified as done in \prettyref{lem:delta-1Basis}.
Alternatively, one can apply Storjohann's transformation with degree
parameter $\delta^{\left(i-1\right)}$ to $\bar{\mathbf{F}}^{\left(i\right)}$
as in \prettyref{eq:storjohannTransformation4parts}. The lemma then
follows from \prettyref{lem:delta-1Basis} after noticing the redundant
block rows that can be disregarded. 
\end{pf}
Notice that if $i=\log\left(n/m\right)-1,$ then \prettyref{thm:extractingOrderBasis}
and \prettyref{lem:linkStorjohanTransform} specialize to \prettyref{cor:extractingFbasisFromGbasis}
and \prettyref{lem:delta-1Basis}. 



\section{Computation of Order Bases}

\label{sec:Order-Basis-Computation}

In this section, we establish a link between two different Storjohann
transformed problems by dividing the transformed problem from the
previous section into two subproblems and then simplifying the second
subproblem. This leads to a recursive method for computing order bases.
We also present an equivalent, iterative method for computing order
bases. The iterative approach is usually more efficient in practice,
as it uses just $O(1)$ iterations in the generic case.


\subsection{\label{sub:Dividing-to-Subproblems}Dividing into Subproblems }

In \prettyref{sec:transform} we have shown that the problem of computing
a $\left(\mathbf{F},\sigma,\vec{s}\right)$-basis can be converted
to the problem of computing a $(\mathbf{F}',\vec{\omega},\vec{s'})$-basis
and, more generally, that the computation of a $(\bar{\mathbf{F}}^{\left(i\right)},2\delta^{\left(i\right)},\vec{s}^{\left(i\right)})$-basis,
a Storjohann transformed problem with degree parameter $\delta^{\left(i\right)}$,
can be converted to the problem of computing a $(\mathbf{F}'^{\left(i\right)},\vec{\omega}^{\left(i\right)},\vec{s}^{\left(i-1\right)})$-basis.
We now consider dividing the new converted problem into two subproblems.

The first subproblem is to compute a $(\mathbf{F}'^{\left(i\right)},2\delta^{\left(i-1\right)},\vec{s}^{\left(i-1\right)})$-basis
or equivalently a $(\bar{\mathbf{F}}^{\left(i-1\right)},2\delta^{\left(i-1\right)},\vec{s}^{\left(i-1\right)})$-basis
$\bar{\mathbf{P}}^{\left(i-1\right)}$, a Storjohann transformed problem
with degree parameter $\delta^{\left(i-1\right)}$. The second subproblem
is computing a $(\mathbf{F}'^{\left(i\right)}\bar{\mathbf{P}}^{\left(i-1\right)},\vec{\omega}^{\left(i\right)},\vec{t}^{\left(i-1\right)})$-basis
$\bar{\mathbf{Q}}^{\left(i\right)}$ using the residual $\mathbf{F}'^{\left(i\right)}\bar{\mathbf{P}}^{\left(i-1\right)}$
from the first subproblem along with a degree shift $\vec{t}^{\left(i-1\right)}=\deg_{\vec{s}^{\left(i-1\right)}}\bar{\mathbf{P}}^{\left(i-1\right)}$.
From Theorem 5.1 in \citep{BL1997} we then know that the product
$\bar{\mathbf{P}}^{\left(i-1\right)}\bar{\mathbf{Q}}^{\left(i\right)}$
is a $(\mathbf{F}'^{\left(i\right)},\vec{\omega}^{\left(i\right)},\vec{s}^{\left(i-1\right)})$-basis
and $\deg_{\vec{s}^{\left(i-1\right)}}\bar{\mathbf{P}}^{\left(i-1\right)}\bar{\mathbf{Q}}^{\left(i\right)}=\deg_{\vec{t}^{\left(i-1\right)}}\bar{\mathbf{Q}}^{\left(i\right)}$. 
\begin{exmp}
\label{exm:subproblems} Let us continue with \prettyref{exm:StorjohannTransformation}
and \prettyref{exm:auxiliaryTransformation} in order to compute a
$\left(\mathbf{F},8,\vec{0}\right)$-basis (or equivalently a $(\bar{\mathbf{F}}^{\left(2\right)},8,\vec{0})$-basis).
This can be determined by computing a $(\mathbf{F}'^{\left(2\right)},[8,4,4],\vec{0})$-basis
as shown in \prettyref{exm:auxiliaryTransformation} where we have
$\mathbf{F}'^{\left(2\right)}=\mathbf{F}'$. Computing a $(\mathbf{F}'^{\left(2\right)},[8,4,4],\vec{0})$-basis
can be divided into two subproblems. The first subproblem is computing
a $(\bar{\mathbf{F}}^{\left(1\right)},4,\vec{0})$-basis $\bar{\mathbf{P}}^{\left(1\right)}$,
the Storjohann partial linearized problem in \prettyref{exm:StorjohannTransformation}.
The residual \[
\mathbf{F}'^{(2)}\bar{\mathbf{P}}^{(1)}=\left[{\begin{array}{rcccccccccc}
0 & \  & x^{8} & \  & x^{6}+x^{9} & \  & x^{4}+x^{6}+x^{9} & \  & x^{6}+x^{8}+x^{9}+x^{10} & \  & x^{5}+x^{8}\\
0 &  & 0 &  & x^{5} &  & x^{4}+x^{6} &  & x^{4}+x^{6} &  & x^{5}+x^{6}\\
0 &  & x^{4} &  & x^{5} &  & x^{5} &  & x^{4}+x^{5}+x^{6} &  & x^{4}\end{array}}\right]\]
 is then used as the input matrix for the second subproblem. The shift
for the second subproblem $\vec{t}^{(1)}=[0,1,2,3,3,3]$ is the list
of column degrees of $\bar{\mathbf{P}}^{(1)}$ and so the second subproblem
is to compute a $(\mathbf{F}'^{(2)}\bar{\mathbf{P}}^{(1)},\left[8,4,4\right],[0,1,2,3,3,3])$-basis,
which is \begin{equation}
\bar{\mathbf{Q}}^{(2)}=\left[{\begin{array}{cccccc}
~~1~ & ~~0~ & ~0~ & ~0~ & ~0~ & ~0~\\
0 & 1 & 0 & 0 & 0 & 0\\
0 & 0 & 1 & ~x^{2}~ & ~x~ & ~1~\\
0 & 0 & 0 & 0 & x & 0\\
0 & 0 & 1 & 0 & 0 & 0\\
0 & 0 & 0 & 0 & 1 & ~x~\end{array}}\right].\label{eq:Qbar2}\end{equation}
 Then $\bar{\mathbf{P}}^{(1)}\bar{\mathbf{Q}}^{(2)}$ gives the $(\mathbf{F}'^{\left(2\right)},[8,4,4],\vec{0})$-basis
shown in \prettyref{exm:auxiliaryTransformation}. 
\end{exmp}
We now show that the dimension of the second subproblem can be significantly
reduced. First, the row dimension can be reduced by over a half. Let
$\hat{\mathbf{P}}^{\left(i-1\right)}=\mathbf{E}^{\left(i\right)}\bar{\mathbf{P}}^{\left(i-1\right)}$. 
\begin{lem}
\label{lem:simplifySecondSubproblem}A $(\bar{\mathbf{F}}^{\left(i\right)}\hat{\mathbf{P}}^{\left(i-1\right)},2\delta^{\left(i\right)},\vec{t}^{\left(i-1\right)})$-basis
is a $(\mathbf{F}'^{\left(i\right)}\bar{\mathbf{P}}^{\left(i-1\right)},\vec{\omega}^{\left(i\right)},\vec{t}^{\left(i-1\right)})$-basis.\end{lem}
\begin{pf}
This follows because $\bar{\mathbf{F}}^{\left(i\right)}\hat{\mathbf{P}}^{\left(i-1\right)}$
is a submatrix of $\mathbf{F}'^{\left(i\right)}\bar{\mathbf{P}}^{\left(i-1\right)}$
after removing rows which already have the correct order $2\delta^{\left(i-1\right)}$. 
\end{pf}
The column dimension of the second subproblem can be reduced by disregarding
the $(\bar{\mathbf{F}}^{\left(i\right)},2\delta^{\left(i\right)},\vec{s}^{\left(i\right)})_{\delta^{\left(i-1\right)}-1}$-basis
which has already been computed. More specifically, after sorting
the columns of $\bar{\mathbf{P}}^{\left(i-1\right)}$ in an increasing
order of their $\vec{s}^{\left(i-1\right)}$-degrees, let $[\bar{\mathbf{P}}_{1}^{\left(i-1\right)},\bar{\mathbf{P}}_{2}^{\left(i-1\right)}]=\bar{\mathbf{P}}^{\left(i-1\right)}$
be such that $\deg_{\vec{s}^{\left(i-1\right)}}\bar{\mathbf{P}}_{1}^{\left(i-1\right)}\le\delta^{\left(i-1\right)}-1$
and $\deg_{\vec{s}^{\left(i-1\right)}}\bar{\mathbf{P}}_{2}^{\left(i-1\right)}\ge\delta^{\left(i-1\right)}$.
Then $\hat{\mathbf{P}}_{1}^{\left(i-1\right)}=\mathbf{E}^{\left(i\right)}\bar{\mathbf{P}}_{1}^{\left(i-1\right)}$
is a $(\bar{\mathbf{F}}^{\left(i\right)},2\delta^{\left(i\right)},\vec{s}^{\left(i\right)})_{\delta^{\left(i-1\right)}-1}$-basis
by \prettyref{lem:linkStorjohanTransform}. In the second subproblem,
the remaining basis elements of a $(\bar{\mathbf{F}}^{\left(i\right)},2\delta^{\left(i\right)},\vec{s}^{\left(i\right)})$-basis
can then be computed without $\bar{\mathbf{P}}_{1}^{\left(i-1\right)}$.

Let $\hat{\mathbf{P}}_{2}^{\left(i-1\right)}=\mathbf{E}^{\left(i\right)}\bar{\mathbf{P}}_{2}^{\left(i-1\right)}$,
$\vec{b}^{\left(i-1\right)}=\deg_{\vec{s}^{\left(i-1\right)}}\bar{\mathbf{P}}_{2}^{\left(i-1\right)}$,
$\bar{\mathbf{Q}}_{2}^{\left(i\right)}$ be a $(\bar{\mathbf{F}}^{\left(i\right)}\hat{\mathbf{P}}_{2}^{\left(i-1\right)},2\delta^{\left(i\right)},\vec{b}^{\left(i-1\right)})$-basis
(or equivalently a $(\mathbf{F}'^{\left(i\right)}\bar{\mathbf{P}}_{2}^{\left(i-1\right)},\vec{\omega}^{\left(i\right)},\vec{b}^{\left(i-1\right)})$-basis),
and $k^{\left(i-1\right)}$ be the column dimension of $\bar{\mathbf{P}}_{1}^{\left(i-1\right)}$.
We then have the following result. 
\begin{lem}
\label{lem:disregardComputedBasisElements} The matrix \[
\bar{\mathbf{Q}}^{\left(i\right)}=\left[\begin{array}{cc}
\mathbf{I}_{k^{\left(i-1\right)}}\\
 & \bar{\mathbf{Q}}_{2}^{\left(i\right)}\end{array}\right]\]
 is a $(\bar{\mathbf{F}}^{\left(i\right)}\hat{\mathbf{P}}^{\left(i-1\right)},2\delta^{\left(i\right)},\vec{t}^{\left(i-1\right)})$-basis
(equivalently a $(\mathbf{F}'^{\left(i\right)}\bar{\mathbf{P}}^{\left(i-1\right)},\vec{\omega}^{\left(i\right)},\vec{t}^{\left(i-1\right)})$-basis).\end{lem}
\begin{pf}
First note that $\bar{\mathbf{Q}}^{\left(i\right)}$ has order $(\bar{\mathbf{F}}^{\left(i\right)}\hat{\mathbf{P}}^{\left(i-1\right)},2\delta^{\left(i\right)})$
as \[
\bar{\mathbf{F}}^{\left(i\right)}\hat{\mathbf{P}}^{\left(i-1\right)}\bar{\mathbf{Q}}^{\left(i\right)}=[\bar{\mathbf{F}}^{\left(i\right)}\hat{\mathbf{P}}_{1}^{\left(i-1\right)},\bar{\mathbf{F}}^{\left(i\right)}\hat{\mathbf{P}}_{2}^{\left(i-1\right)}\bar{\mathbf{Q}}_{2}^{\left(i\right)}]\equiv0\mod x^{2\delta^{\left(i\right)}}.\]
 In addition, $\bar{\mathbf{Q}}^{\left(i\right)}$ has minimal $\vec{t}^{\left(i-1\right)}$
degrees as $\bar{\mathbf{Q}}_{2}^{\left(i\right)}$ is $\vec{b}$-minimal.
Hence, by \prettyref{lem:orderBasisEquivalence}, $\bar{\mathbf{Q}}^{\left(i\right)}$
is a $(\bar{\mathbf{F}}^{\left(i\right)}\cdot\hat{\mathbf{P}}^{\left(i-1\right)},2\delta^{\left(i\right)},\vec{t}^{\left(i-1\right)})$-basis. 
\end{pf}
%
\begin{comment}
Alternatively, one can also argue that since $\hat{\mathbf{P}}_{1}^{\left(i-1\right)}$
already has order $(\bar{\mathbf{F}}^{\left(i\right)},2\delta^{\left(i\right)})$,
it cannot contribute in any way to and cannot be affected in any way
by the computations of a $(\bar{\mathbf{F}}^{\left(i\right)}\hat{\mathbf{P}}^{\left(i-1\right)},2\delta^{\left(i\right)},\vec{b}^{\left(i-1\right)})$-basis,
hence it is sufficient to just use $\bar{\mathbf{F}}^{\left(i\right)}\hat{\mathbf{P}}_{2}^{\left(i-1\right)}$
to compute a $(\bar{\mathbf{F}}^{\left(i\right)}\cdot\hat{\mathbf{P}}_{2}^{\left(i-1\right)},2\delta^{\left(i\right)},\vec{b}^{\left(i-1\right)})$-basis. 
\end{comment}
{}

\prettyref{lem:disregardComputedBasisElements} immediately leads
to the following. 
\begin{lem}
\label{lem:computationAtTopLevel}Let $\hat{\mathbf{S}}=[\hat{\mathbf{P}}_{1}^{\left(i-1\right)},\hat{\mathbf{P}}_{2}^{\left(i-1\right)}\bar{\mathbf{Q}}_{2}^{\left(i\right)}]$,
and let $I$ be the column rank profile of $\lcoeff(x^{\vec{s}^{\left(i\right)}}\hat{\mathbf{S}})$.
Then $\hat{\mathbf{S}}_{I}$ is a $(\bar{\mathbf{F}}^{\left(i\right)},2\delta^{\left(i\right)},\vec{s}^{\left(i\right)})$-basis.\end{lem}
\begin{pf}
From \prettyref{lem:disregardComputedBasisElements}, $\bar{\mathbf{Q}}^{\left(i\right)}$
is a $(\mathbf{F}'^{\left(i\right)}\bar{\mathbf{P}}^{\left(i-1\right)},\vec{\omega}^{\left(i\right)},\vec{t}^{\left(i-1\right)})$-basis
and hence $\bar{\mathbf{P}}^{\left(i-1\right)}\bar{\mathbf{Q}}^{\left(i\right)}$
is a $(\mathbf{F}'^{\left(i\right)},\vec{\omega}^{\left(i\right)},\vec{s}^{\left(i-1\right)})$-basis.
Since $[\hat{\mathbf{P}}_{1}^{\left(i-1\right)},\hat{\mathbf{P}}_{2}^{\left(i-1\right)}\bar{\mathbf{Q}}_{2}^{\left(i\right)}]=\mathbf{E}^{\left(i\right)}\bar{\mathbf{P}}^{\left(i-1\right)}\bar{\mathbf{Q}}^{\left(i\right)}$,
the result follows from \prettyref{thm:extractingOrderBasis}. %
\begin{comment}
$=\mathbf{E}^{\left(i\right)}[\bar{\mathbf{P}}_{1}^{\left(i-1\right)},\bar{\mathbf{P}}_{2}^{\left(i-1\right)}\bar{\mathbf{Q}}_{2}^{\left(i\right)}]$ 
\end{comment}
{}\end{pf}
\begin{exmp}
Continuing with \prettyref{exm:StorjohannTransformation}, \prettyref{exm:auxiliaryTransformation},
and \prettyref{exm:subproblems}, %
\begin{comment}
notice that after the first subproblem, the second subproblem of computing
$\bar{\mathbf{Q}}^{(2)}$ in \prettyref{exm:subproblems} is really
a smaller problem of computing the lower right $4\times4$ submatrix
$\bar{\mathbf{Q}}_{2}^{(2)}$, which is a $(\bar{\mathbf{F}}^{\left(2\right)}\hat{\mathbf{P}}_{2}^{\left(1\right)},8,\vec{b}^{\left(1\right)})$-basis
(or equivalently a $(\mathbf{F}'^{\left(2\right)}\bar{\mathbf{P}}_{2}^{\left(1\right)},[8,4,4],\vec{b}^{\left(1\right)})$-basis),
where $\bar{\mathbf{P}}_{2}^{\left(1\right)}$ is the last $4$ columns
of $\bar{\mathbf{P}}^{(1)}$, $\vec{b}^{(1)}=[2,3,3,3]$ is the list
of column degrees of $\bar{\mathbf{P}}_{2}^{(1)}$, and $\hat{\mathbf{P}}_{2}^{\left(1\right)}$
is the first $4$ rows of $\bar{\mathbf{P}}_{2}^{(1)}$. 
\end{comment}
{}notice that in the computation of the second subproblem, instead of
using $\mathbf{F}'^{\left(2\right)},$ $\bar{\mathbf{P}}^{\left(1\right)}$,
$\bar{\mathbf{Q}}^{(2)}$, and $\bar{\mathbf{P}}^{(1)}\bar{\mathbf{Q}}^{(2)}$,
the previous lemmas show that we can just use their submatrices, $\bar{\mathbf{F}}^{(2)}$
the top left $1\times4$ submatrix of $\mathbf{F}'^{\left(2\right)}$,
$\hat{\mathbf{P}}_{2}^{(1)}$ the top right $4\times4$ submatrix
of $\bar{\mathbf{P}}^{(1)}$, $\bar{\mathbf{Q}}_{2}^{(2)}$ the bottom
right $4\times4$ submatrix of $\bar{\mathbf{Q}}^{(2)}$, and $\hat{\mathbf{P}}_{2}^{(1)}\bar{\mathbf{Q}}_{2}^{(2)}$
the top right $4\times4$ submatrix of $\bar{\mathbf{P}}^{(1)}\bar{\mathbf{Q}}^{(2)}$of
lower dimensions. 
\end{exmp}
\prettyref{lem:computationAtTopLevel} gives us a way of computing
a $\left(\mathbf{F},\sigma,\vec{s}\right)$-basis. We can set $i$
to $\log\left(n/m\right)-1$ so that $(\bar{\mathbf{F}}^{\left(i\right)},2\delta^{\left(i\right)},\vec{s}^{\left(i\right)})$=$\left(\mathbf{F},\sigma,\vec{s}\right)$,
and compute a $(\bar{\mathbf{F}}^{\left(i\right)},2\delta^{\left(i\right)},\vec{s}^{\left(i\right)})$-basis.
By \prettyref{lem:computationAtTopLevel}, this can be divided into
two subproblems. The first produces $[\hat{\mathbf{P}}_{1}^{\left(i-1\right)},\hat{\mathbf{P}}_{2}^{\left(i-1\right)}]=\hat{\mathbf{P}}^{\left(i-1\right)}=\mathbf{E}^{\left(i\right)}\bar{\mathbf{P}}^{\left(i-1\right)}$
from computing a $(\bar{\mathbf{F}}^{\left(i-1\right)},2\delta^{\left(i-1\right)},\vec{s}^{\left(i-1\right)})$-basis
$\bar{\mathbf{P}}^{\left(i-1\right)}$. The second subproblem then
computes a $(\bar{\mathbf{F}}^{\left(i\right)}\hat{\mathbf{P}}_{2}^{\left(i-1\right)},2\delta^{\left(i\right)},\vec{b}^{\left(i-1\right)})$-basis
$\bar{\mathbf{Q}}_{2}^{\left(i\right)}$. Note the first subproblem
of computing a $(\bar{\mathbf{F}}^{\left(i-1\right)},2\delta^{\left(i-1\right)},\vec{s}^{\left(i-1\right)})$-basis
can again be divided into two subproblems just as before. This can
be repeated recursively until we reach the base case with degree parameter
$\delta^{\left(1\right)}=2d$. The total number of recursion levels
is therefore $\log\left(n/m\right)-1$.

Notice that the transformed matrix $\mathbf{F}'^{\left(i\right)}$
is not used explicitly in the computation, even though it is crucial
for deriving our results.


\subsection{The Iterative View}

In this subsection we present our algorithm, which uses an iterative
version of the computation discussed above.%
\begin{comment}
The recursive top-down approach of the previous subsection is useful
for giving an overall picture of the computation process. Algorithm
\prettyref{alg:mab} uses an equivalent corresponding bottom-up iterative
approach. 
\end{comment}
{}%
\begin{comment}
, allowing the complexity to be more easily analyzed. 
\end{comment}
{}%
\begin{comment}
. In practice, bottom-up iterative approaches are more efficient than
the corresponding top-down recursive approaches. For our purpose,
it is also easier to analyze the computational cost of the iterative
procedure. 
\end{comment}
{} The iterative version is usually more efficient in practice, considering
that the generic case has balanced output that can be computed with
just one iteration, whereas the recursive method has to go through
$\log(n/m)-1$ levels of recursion.

\prettyref{alg:mab} uses a subroutine $\mab$, the algorithm from
Giorgi et al. \citeyearpar{Giorgi2003}, for computing order bases
with balanced input. Specifically, $\left[\mathbf{Q},\vec{a}\right]=\mab(\mathbf{G},\sigma,\vec{b})$
computes a $(\mathbf{G},\sigma,\vec{b})$-basis and also returns its
$\vec{b}$-column degrees $\vec{a}$. The other subroutine $\StorjohannTransform$
is the transformation described in \prettyref{sub:storjohannTransformation}.

\prettyref{alg:mab} proceeds as follows. In the first iteration,
which is the base case of the recursive approach, we set the degree
parameter $\delta^{\left(1\right)}$ to be twice the average degree
$d$ and apply Storjohann's transformation to produce a new input
matrix $\bar{\mathbf{F}}^{\left(1\right)}$, which has $l^{\left(1\right)}$
block rows. Then a $(\bar{\mathbf{F}}^{\left(1\right)},2\delta^{\left(1\right)},\vec{s}^{\left(1\right)})$-basis
$\bar{\mathbf{P}}^{\left(1\right)}$ is computed. Note this is in
fact the first subproblem of computing a $(\bar{\mathbf{F}}^{\left(2\right)},2\delta^{\left(2\right)},\vec{s}^{\left(2\right)})$-basis,
which is another Storjohann transformed problem and also the problem
of the second iteration. At the second iteration, we work on a new
Storjohann transformed problem with the degree doubled and the number
of block rows $l^{\left(2\right)}=(l^{\left(1\right)}-1)/2$ reduced
by over a half. The column dimension is reduced by using the result
from the previous iteration. More specifically, we know that the basis
$\bar{\mathbf{P}}^{\left(1\right)}$ already provides a $(\bar{\mathbf{F}}^{\left(2\right)},2\delta^{\left(2\right)},\vec{s}^{\left(2\right)})_{\delta^{\left(1\right)}-1}$-basis
$\hat{\mathbf{P}}_{1}^{\left(1\right)}$, which can be disregarded
in the remaining computation. The remaining work in the second iteration
is to compute a $(\bar{\mathbf{F}}^{\left(2\right)}\hat{\mathbf{P}}_{2}^{\left(1\right)},2\delta^{\left(2\right)},\vec{b}^{\left(1\right)})$-basis
$\bar{\mathbf{Q}}^{\left(2\right)}$, where $\vec{b}^{\left(1\right)}=\deg_{\vec{s}^{\left(1\right)}}\bar{\mathbf{P}}_{2}^{\left(1\right)}$,
and then to combine it with the result from the previous iteration
to form a matrix $[\hat{\mathbf{P}}_{1}^{\left(1\right)},\hat{\mathbf{P}}_{2}^{\left(1\right)}\bar{\mathbf{Q}}^{\left(2\right)}]$
in order to extract a $(\bar{\mathbf{F}}^{\left(2\right)},2\delta^{\left(2\right)},\vec{s}^{\left(2\right)})$-basis
$\bar{\mathbf{P}}^{\left(2\right)}$.

With a $(\bar{\mathbf{F}}^{\left(2\right)},2\delta^{\left(2\right)},\vec{s}^{\left(2\right)})$-basis
computed, we can repeat the same process to use it for computing a
$(\bar{\mathbf{F}}^{\left(3\right)},2\delta^{\left(3\right)},\vec{s}^{\left(3\right)})$-basis.
Continue, using the computed $(\bar{\mathbf{F}}^{\left(i-1\right)},2\delta^{\left(i-1\right)},\vec{s}^{\left(i-1\right)})$-basis
to compute a $(\bar{\mathbf{F}}^{\left(i\right)},2\delta^{\left(i\right)},\vec{s}^{\left(i\right)})$-basis,
until all $n$ elements of a $\left(\mathbf{F},\sigma,\vec{s}\right)$-basis
have been determined.

\input{algorithm1.tex}

%
\begin{comment}
\begin{thm}
Algorithm \prettyref{alg:mab} computes a $\left(\mathbf{F},\sigma,\vec{s}\right)$-basis
correctly.\end{thm}
\begin{pf}
This follows from \prettyref{lem:simplifySecondSubproblem}, \prettyref{lem:disregardComputedBasisElements},
and \prettyref{lem:computationAtTopLevel}. 
\end{pf}

\end{comment}
{} 



\section{Computational Complexity}

\label{sec:complexity}

In this section, we analyze the computational complexity of \prettyref{alg:mab}.
\begin{lem}
\prettyref{alg:mab} computes a $\left(\mathbf{F},\sigma,\vec{s}\right)$-basis
in no more than $\log\left(n/m\right)-1$ iterations.\end{lem}
\begin{pf}
Each iteration $i$ computes a $(\bar{\mathbf{F}}^{\left(i\right)},2\delta^{\left(i\right)},\vec{s}^{\left(i\right)})$-basis.
At iteration ${i}^{*}=\log(n/m)-1$, the degree parameter is $\sigma/2$
and $(\bar{\mathbf{F}}^{\left({i}^{*}\right)},2\delta^{\left({i}^{*}\right)},\vec{s}^{\left({i}^{*}\right)})=\left(\mathbf{F},\sigma,\vec{s}\right)$.\end{pf}
\begin{lem}
\label{lem:remainingNumberElements}If the shift $\vec{s}=\left[0,\dots,0\right]$,
then a $\left(\mathbf{F},\sigma,\vec{s}\right)_{\delta^{\left(i\right)}-1}$-basis
(or equivalently a $(\bar{\mathbf{F}}^{\left(i\right)},2\delta^{\left(i\right)},\vec{s}^{\left(i\right)})_{\delta^{\left(i\right)}-1}$-basis)
computed at iteration $i$ has at least $n-n/2^{i}$ elements, and
hence at most $n/2^{i}$ elements remain to be computed. If the shift
$\vec{s}$ is balanced, that is, $\max\vec{s}\in O(d)$ assuming $\min\vec{s}=0$,
then the number $n^{\left(i\right)}$ of remaining basis elements
at iteration $i$ is $O(n/2^{i})$.\end{lem}
\begin{pf}
The uniform case follows from the idea of \citet{storjohann-villard:2005}
on null space basis computation discussed in \prettyref{sub:Unbalanced-Output}.
For the balanced case, the average column degree is bounded by $cd=cm\sigma/n$
for some constant $c$. The first iteration $\lambda$ such that $\delta^{\left(\lambda\right)}$
reaches $cd$ is therefore a constant. That is, $\delta^{\left(\lambda\right)}=2^{\lambda}d\ge cd>\delta^{\left(\lambda-1\right)}$
and hence $\lambda=\left\lceil \log c\right\rceil $. By the same
argument as in the uniform case, the number of remaining basis elements
$n^{\left(i\right)}\le n/2^{i-\lambda}=2^{\lambda}(n/2^{i})\in O(n/2^{i})$
at iteration $i\ge\lambda$. For iterations $i<\lambda$, certainly
$n^{\left(i\right)}\le n<2^{\lambda}(n/2^{i})\in O(n/2^{i})$.\end{pf}
\begin{thm}
\label{thm:balancedCost}If the shift $\vec{s}$ is balanced with
$\min\left(\vec{s}\right)=0$, then \prettyref{alg:mab} computes
a $\left(\mathbf{F},\sigma,\vec{s}\right)$-basis with a cost of $O\left(n^{\omega}\bar{\M}(d)\log\sigma)\right)=O\left(n^{\omega}d\log d\log\log d\log\sigma)\right)\subset O^{\sim}\left(n^{\omega}d\right)$
field operations. \end{thm}
\begin{pf}
The computational cost depends on the degree, the row dimension, and
the column dimension of the problem at each iteration. The degree
parameter $\delta^{\left(i\right)}$ is $2^{i}d$ at iteration $i$.
The number of block rows $l^{\left(i\right)}$ is $\sigma/\delta^{\left(i\right)}-1$,
which is less than $\sigma/(2^{i}d)=n/(2^{i}m)$ at iteration $i$.
The row dimension is therefore less than $n/2^{i}$ at iteration $i$.

The column dimension of interest at iteration $i$ is the column dimension
of $\hat{\mathbf{P}}_{2}^{\left(i-1\right)}$ (equivalently the column
dimension of $\bar{\mathbf{P}}_{2}^{\left(i-1\right)}$), which is
the sum of two components, $n^{\left(i-1\right)}+(l^{\left(i-1\right)}-1)m$.
The first component $n^{\left(i-1\right)}\in O(n/2^{i})$ by \prettyref{lem:remainingNumberElements}.
The second component $(l^{\left(i-1\right)}-1)m<n/2^{i-1}-m<n/2^{i-1}$
comes from the size of the identity matrix added in Storjohann's transformation.
Therefore, the overall column dimension of the problem at iteration
$i$ is $O(n/2^{i})$.

At each iteration, the four most expensive operations are the multiplications
at \prettyref{line:matrixProduct1} and \prettyref{line:matrixProduct2},
the order basis computation at \prettyref{line:orderBasisComputation},
and extracting the basis at \prettyref{line:LSP}.

The matrices $\bar{\mathbf{F}}^{\left(i\right)}$ and $\hat{\mathbf{P}}_{2}^{\left(i-1\right)}$
have degree $O(2^{i}d)$ and dimensions $O(n/2^{i})\times O\left(n\right)$
and $O\left(n\right)\times O(n/2^{i})$. The multiplication cost is
therefore $2^{i}\MM(n/2^{i},2^{i}d)$ field operations, which is bounded
by\begin{align}
2^{i}\MM(n/2^{i},2^{i}d) & \in O\left(2^{i}\left(n/2^{i}\right)^{\omega}\bar{\M}(2^{i}d)\right)\label{eq:MMtoM}\\
 & \subset O\left(n^{\omega}\left(2^{i}\right)^{1-\omega}\bar{\M}\left(2^{i}\right)\bar{\M}(d)\right)\label{eq:separateMultiplication}\\
 & \subset O\left(n^{\omega}\left(2^{i}\right)^{1-\omega}\left(2^{i}\right)^{\omega-1}\bar{\M}(d)\right)\label{eq:MultiplicationToExponent}\\
 & \subset O\left(n^{\omega}\bar{\M}(d)\right).\nonumber \end{align}
 Line \ref{eq:separateMultiplication} follows from $\bar{\M}(ab)\le\bar{\M}(a)\bar{\M}(b)$.
Line \ref{eq:MultiplicationToExponent} follows from $\bar{\M}(t)\in O(t^{\omega-1})$.

The matrices $\hat{\mathbf{P}}_{2}^{\left(i-1\right)}$ and $\bar{\mathbf{Q}}^{\left(i\right)}$
of the second multiplication have the same degree $O(2^{i}d)$ and
dimensions $O\left(n\right)\times O(n/2^{i})$ and $O(n/2^{i})\times O(n/2^{i})$
and can also be multiplied with a cost of $O\left(n^{\omega}\bar{\M}(d)\right)$
field operations. The total cost of the multiplications over $O(\log\left(n/m\right))$
iterations is therefore $O\left(n^{\omega}\bar{\M}(d)\log(n/m)\right)$.

The input matrix $\mathbf{G}^{\left(i\right)}=\bar{\mathbf{F}}^{\left(i\right)}\hat{\mathbf{P}}_{2}^{\left(i-1\right)}$
of the order basis computation problem at iteration $i$ has dimension
$O(n/2^{i})\times O(n/2^{i})$ and the order of the problem is $2\delta^{\left(i\right)}\in O(2^{i}d)$.
Therefore, the cost of the order basis computation at iteration $i$
is $O\left(\left(n/2^{i}\right)^{\omega}\M'\left(2^{i}d\right)\log\left(2^{i}d\right)\right)$.
The total cost over $O(\log\left(n/m\right))$ iterations is bounded
by \begin{align*}
 & O\left(\sum_{i=1}^{\infty}\left(\left(n/2^{i}\right)^{\omega}\bar{\M}\left(2^{i}d\right)\log\left(2^{i}d\right)\right)\right)\\
\subset & O\left(\sum_{i=1}^{\infty}\left(\left(n/2^{i}\right)^{\omega}\bar{\M}\left(2^{i}\right)\log\left(2^{i}\right)\bar{\M}\left(d\right)\log\left(d\right)\right)\right)\\
\subset & O\left(\sum_{i=1}^{\infty}\left(n^{\omega}\left(2^{i}\right)^{-\omega}\left(2^{i}\right)^{\omega-1}\bar{\M}\left(d\right)\log\left(d\right)\right)\right)\\
\subset & O\left(n^{\omega}\bar{\M}\left(d\right)\log\left(d\right)\sum_{i=1}^{\infty}\left(2^{-i}\right)\right)\\
\subset & O\left(n^{\omega}\bar{\M}\left(d\right)\log\left(d\right)\right).\end{align*}


Finally, extracting an order basis by LSP factorization costs $O\left(n^{\omega}\right)$,
which is dominated by the other costs. Combining the above gives $O\left(n^{\omega}\bar{\M}\left(d\right)\log(n/m)+n^{\omega}\bar{\M}\left(d\right)\log d\right)=O\left(n^{\omega}\bar{\M}\left(d\right)\log\sigma)\right)$
as the total cost of the algorithm. 
\end{pf}




\section{Unbalanced Shifts}

\label{sec:Unbalanced-Shift}

\prettyref{sec:complexity} shows that \prettyref{alg:mab} can efficiently
compute a $\left(\mathbf{F},\sigma,\vec{s}\right)$-basis when the
shift $\vec{s}$ is balanced. When the $\vec{s}$ is unbalanced (something
important for example in normal form computation), then \prettyref{alg:mab}
still returns a correct answer but may be less efficient. The possible
inefficiency results because there may not be enough partial results
from the intermediate subproblems to sufficiently reduce the column
dimension of the subsequent subproblem. This is clear from the fact
that the column degrees of the output can be much larger and no longer
sum up to $O\left(m\sigma\right)$ as in the balanced shift case.
The shifted $\vec{s}$-column degrees, however, still behave well.
In particular, the total $\vec{s}$-degree increase is still bounded
by $m\sigma$ as stated in \prettyref{lem:size}, while the shifted
degree of any column can also increase by up to $\sigma$. Recall
that \prettyref{lem:size} states that for any shift $\vec{s}$, there
exists a $\left(\mathbf{F},\sigma,\vec{s}\right)$-basis still having
a total size bounded by $nm\sigma$ which gives hope for efficient
computation.

In this section, we describe an algorithm for an important special
case of unbalanced shift -- when the input shift $\vec{s}$ satisfies
the condition:\[
\sum_{i=1}^{n}(\max(\vec{s})-\vec{s}_{i})\le m\sigma.\]
 For simpler presentation, we use the equivalent condition \begin{equation}
\vec{s}\le0\mbox{ and }\sum_{i}-\vec{s}_{i}\le m\sigma,\label{con:unbalancedCondition}\end{equation}
 which can always be obtained from the previous condition by using
$\vec{s}-\max\vec{s}$ as the new shift. Note that translating every
entry of the shift by the same constant does not change the problem.

In the balanced shift case, a central problem is to find a way to
handle unbalanced column degrees of the output order basis. In this
section, the unbalanced shift makes row degrees of the output also
unbalanced, which is a major problem that needs to be resolved. Here
we note a second transformation by \citet{Storjohann:2006} which
converts the input in such a way that each high degree row of the
output becomes multiple rows of lower degrees. We refer to this as
Storjohann's second transformation to distinguish it from that described
in \prettyref{sub:storjohannTransformation}. The transformed problem
can then be computed efficiently using \prettyref{alg:mab}. After
the computation, rows can then be combined appropriately to form a
basis of the original problem. The method is computationally efficient.

Unfortunately, the bases computed this way are not minimal and hence
do not in general produce our reduced order bases. In the following,
we describe a transformation that incorporates Storjohann's second
transformation and guarantees the minimality of some columns of the
output, hence providing a partial order basis. We can then work on
the remaining columns iteratively as done in the balanced shift case
to compute a full order basis.

Condition \eqref{con:unbalancedCondition} essentially allows us to
locate the potential high degree rows that need to be balanced. In
more general cases, we may not know in advance which are the high
degree rows that need to be balanced, so our approach given in this
section does not work directly. This suggests that one possible future
direction to pursue is to find an effective way to estimate the row
degree of the result pivot entries. Such an estimate may allow us
to apply the method given in this section efficiently for general
unbalanced shifts.


\subsection{Transform to Balanced Shifts}

We now describe the transformation for balancing the high degree rows
of the resulting basis. Consider the problem of computing a $\left(\mathbf{F},\sigma,\vec{s}\right)$-basis,
where the input shift $\vec{s}$ satisfies the conditions \eqref{con:unbalancedCondition}.
Let $\alpha,\beta\in\mathbb{Z}_{>0}$ be two parameters. For each
shift entry $s_{i}$ in $\vec{s}$ with $-s_{i}>\alpha+\beta$, let
\[
r_{i}=\mbox{rem}\left(-s_{i}-\alpha-1,\beta\right)+1\]
 be the remainder when $-s_{i}-\alpha$ is divided by $\beta$, and
where $r_{i}=\beta$ in the case where the remainder is $0$, and
set \[
q_{i}=\begin{cases}
1 & \mbox{if }-s_{i}\le\alpha+\beta\\
1+\left(-s_{i}-\alpha-r_{i}\right)/\beta & \mbox{otherwise}\end{cases}\]
 Then, for each $q_{i}>1$, we expand the corresponding $i$th column
$\mathbf{f}_{i}$ of $\mathbf{F}$ and shift $s_{i}$ to \begin{eqnarray*}
\tilde{\mathbf{F}}^{\left(i\right)} & = & \left[~\mathbf{f}_{i},~x^{r_{i}}\mathbf{f}_{i},~x^{r_{i}+\beta}\mathbf{f}_{i},~\dots~,~x^{r_{i}+(q_{i}-2)\beta}\mathbf{f}_{i}\right],~~\tilde{s}_{i}=\left[-\alpha-\beta,~\dots~,-\alpha-\beta\right]\end{eqnarray*}
 with $q_{i}$ entries in each case. When $q_{i}=1$, the corresponding
shift entry and input column remain the same, that is, $\tilde{s}_{i}=s_{i}$,
and $\tilde{\mathbf{F}}^{\left(i\right)}=\mathbf{f}_{i}$. Then for
the transformed problem, the new shift becomes $\bar{s}=[\tilde{s}_{1},\dots,\tilde{s}_{n}]\in\mathbb{Z}_{\le0}^{\bar{n}}$,
and the new input matrix becomes $\bar{\mathbf{F}}=[\tilde{\mathbf{F}}^{\left(1\right)},\dots,\tilde{\mathbf{F}}^{\left(n\right)}]\in\mathbb{K}\left[x\right]^{m\times\bar{n}}$,
with the new column dimension $\bar{n}$ satisfies $\bar{n}=\sum_{i=1}^{n}q_{i}$.
Note that every entry of the new shift $\bar{s}$ is an integer from
$-\alpha-\beta$ to $0$. Let \[
\mathbf{E}=\left[\begin{array}{ccccc|ccc|ccccccc}
1 & x^{r_{1}} & x^{r_{1}+\beta} & \cdots & x^{r_{1}+(q_{1}-2)\beta} &  &  & \\
\hline  &  &  &  &  & \ddots &  & \\
 &  &  &  &  &  & \  & \ddots\\
\hline  &  &  &  &  &  &  &  & 1 & x^{r_{n}} & x^{r_{n}+\beta} & \cdots & x^{r_{n}+(q_{n}-2)\beta}\end{array}\right]_{n\times\bar{n}}.\]
 Then $\bar{\mathbf{F}}=\mathbf{F}\mathbf{E}$. Storjohann's second
transformation is determined by setting $\alpha=-1$, a value not
allowed in our transformation (we show later in \prettyref{thm:correctHighDegreeElements}
that this value is not useful in our case). One can verify that the
new dimension \[
\bar{n}=\sum_{i=1}^{n}q_{i}\le n+\sum_{i=1}^{n}-s_{i}/\beta\le m\sigma/\beta+n.\]
 Thus by setting $\beta\in\Theta\left(m\sigma/n\right)=\Theta\left(d\right)$,
we can make $\bar{n}\in\Theta\left(n\right)$. Furthermore, by also
setting $\alpha\in\Theta\left(d\right)$, we have a balanced shift
problem since \[
\max\bar{s}-\min\bar{s}\le-\min\bar{s}\le\alpha+\beta\in\Theta(d).\]
 Hence \prettyref{alg:mab} can compute a $\left(\bar{\mathbf{F}},\sigma,\bar{s}\right)$-basis
with cost $O^{\sim}\left(n^{\omega}d\right)$ in this case.

With a $\left(\bar{\mathbf{F}},\sigma,\bar{s}\right)$-basis $\bar{\mathbf{P}}\in\mathbb{K}\left[x\right]^{\bar{n}\times\bar{n}}$
computed, let us now consider $\mathbf{E}\bar{\mathbf{P}}\in\mathbb{K}\left[x\right]^{n\times\bar{n}}$.
While it is easy to see that $\mathbf{E}\bar{\mathbf{P}}$ has order
$\left(\mathbf{F},\sigma\right)$ since \textbf{$\mathbf{F}\mathbf{E}\bar{\mathbf{P}}=\bar{\mathbf{F}}\bar{\mathbf{P}}\equiv0\mod x^{\sigma}$},
in general it is not a minimal basis (in fact, $\mathbf{E}\bar{\mathbf{P}}$
is not even square). However, our transformation does guarantee that
the highest degree columns of $\mathbf{E}\bar{\mathbf{P}}$ having
$\vec{s}$-degrees exceed $-\alpha$ are minimal. That is, the columns
of $\mathbf{E}\bar{\mathbf{P}}$ whose $\vec{s}$-degrees exceed $-\alpha$
are exactly the columns of a $\left(\mathbf{F},\sigma,\vec{s}\right)$-basis
whose $\vec{s}$-degrees exceed $-\alpha$. We have therefore correctly
computed a partial $\left(\mathbf{F},\sigma,\vec{s}\right)$-basis. 
\begin{exmp}
\label{exm:unbalancedShift} Let us use the same input as in \prettyref{exm:StorjohannTransformation},
but with shift $\vec{s}=[0,-3,-5,-6]$, and parameters $\alpha=\beta=1$.
Then we get the transformed input \begin{align*}
\bar{\mathbf{F}}=[ & x+x^{2}+x^{3}+x^{4}+x^{5}+x^{6},\,~1+x+x^{5}+x^{6}+x^{7},\,~x+x^{2}+x^{6}+x^{7}+x^{8},\\
 & 1+x^{2}+x^{4}+x^{5}+x^{6}+x^{7},\,~x+x^{3}+x^{5}+x^{6}+x^{7}+x^{8},\,~x^{2}+x^{4}+x^{6}+x^{7}+x^{8}+x^{9},\\
 & x^{3}+x^{5}+x^{7}+x^{8}+x^{9}+x^{10},\,~1+x+x^{3}+x^{7},\,~x+x^{2}+x^{4}+x^{8},\\
 & x^{2}+x^{3}+x^{5}+x^{9},\,~x^{3}+x^{4}+x^{6}+x^{10},\,~x^{4}+x^{5}+x^{7}+x^{11}]\end{align*}
 having $12$ components, and $\bar{s}=[0,-2,-2,-2,-2,-2,-2,-2,-2,-2,-2,-2]$.
In this case $r_{1}=r_{2}=r_{3}=r_{4}=1$, $q_{1}=1$, $q_{2}=2$,
$q_{3}=4$, $q_{4}=5$ and the transformation matrix is \[
\mathbf{E}=\left[{\begin{array}{c|cc|cccc|ccccc}
~1~ & ~0~ & ~0~ & ~0~ & ~0~ & ~0~ & ~0~ & ~0~ & ~0~ & ~0~ & ~0~ & ~0~\\
\hline 0 & 1 & x & 0 & 0 & 0 & 0 & 0 & 0 & 0 & 0 & 0\\
\hline 0 & 0 & 0 & 1 & x & x^{2} & x^{3} & 0 & 0 & 0 & 0 & 0\\
\hline 0 & 0 & 0 & 0 & 0 & 0 & 0 & 1 & x & x^{2} & x^{3} & x^{4}\end{array}}\right].\]
 Using the earlier algorithm for balanced shift, we compute a $(\bar{\mathbf{F}},8,\bar{s})$-basis
\[
\bar{\mathbf{P}}=\left[{\begin{array}{cccccccccccc}
0 & ~0~ & ~0~ & ~0~ & 0 & 0 & 0 & 0 & 0 & 0 & 0 & 1\\
\hline ~x~ & 1 & 0 & 0 & 1 & 0 & ~x~ & 0 & 0 & 0 & ~x~ & ~0~\\
0 & 0 & 1 & 0 & 0 & ~x~ & 1+~x~ & ~x~ & ~x~ & ~x~ & 1 & 0\\
\hline ~x~ & 1 & 0 & 1 & 1+x & 1 & ~x~ & 0 & 0 & 0 & 0 & 1\\
~x~ & 0 & 1 & 1 & 1+x & 1+x & 1 & ~x~ & ~x~ & 0 & 0 & 0\\
~x~ & 0 & 0 & 1 & 1+x & 1+x & 1 & ~x~ & 0 & 1 & 0 & 0\\
~x~ & 0 & 0 & 1 & 1 & 0 & 0 & 1 & 0 & 0 & 0 & 0\\
\hline 0 & 0 & 0 & 1 & ~x~ & 1 & 0 & 0 & 0 & 0 & 0 & 1\\
0 & 1 & 0 & 0 & 0 & 0 & 0 & 0 & 0 & 0 & 0 & 0\\
0 & 0 & 1 & 0 & 0 & 0 & 0 & 0 & 0 & 0 & 0 & 0\\
0 & 0 & 0 & 1 & 0 & ~x~ & 1 & 1 & 1 & 1 & 0 & 0\\
0 & 0 & 0 & 1 & 0 & 0 & 0 & 0 & 0 & 0 & 0 & 0\end{array}}\right]\]
 with $\bar{s}$-degrees $[-1,-2,-2,-2,-1,-1,-1,-1,-1,-1,-1,0]$.
Only the last column has $\bar{s}$-degree exceeding $-\alpha=-1$
and so is the only column guaranteed to give a correct $(\mathbf{F},8,\vec{s})$-basis.
Comparing \[
\mathbf{E}\bar{\mathbf{P}}=\left[{\begin{array}{cccccccccccc}
0 & ~0~ & ~0~ & 0 & ~0~ & 0 & ~0~ & ~0~ & ~0~ & ~0~ & ~0~ & ~1~\\
x & 1 & x & 0 & 1 & x^{2} & x^{2} & x^{2} & x^{2} & x^{2} & 0 & 0\\
x+x^{2}+x^{3}+x^{4} & 1 & x & 1+x+x^{2}+x^{3} & 1 & 1+x+x^{3} & x^{2} & x^{2} & x^{2} & x^{2} & 0 & 1\\
0 & x & x^{2} & 1+x^{3}+x^{4} & x & 1+x^{4} & x^{3} & x^{3} & x^{3} & x^{3} & 0 & 1\end{array}}\right]\]
 to a $(\mathbf{F},8,\vec{s})$-basis \[
\mathbf{P}=\left[{\begin{array}{cccc}
~0~ & 0 & 0 & ~1~\\
1 & 0 & 0 & 0\\
1 & x^{2}+x^{3}+x^{4} & 1+x+x^{2}+x^{3} & 1\\
x & x^{2} & 1+x^{3}+x^{4} & 1\end{array}}\right]\]
 with $\vec{s}$-degrees $[-3,-1,-2,~0~]$, we see that the last column
of $\mathbf{E}\bar{\mathbf{P}}$ is a element of a $(\mathbf{F},8,\vec{s})$-basis.

If we set $\alpha=2,\beta=1$, then the new transformed problem gives
\[
\bar{\mathbf{P}}=\left[{\begin{array}{ccccccccc}
~0~ & ~0~ & ~0~ & ~0~ & ~0~ & ~0~ & ~0~ & ~0~ & ~1~\\
1 & 0 & 0 & x & 1+x & x & x & \, x\, & \,0\,\\
1 & x^{2} & 1 & x & 1 & x & \, x\, & 0 & 1\\
0 & x^{2} & 1 & \, x\, & 1 & \, x\, & 0 & 1 & 0\\
0 & x^{2} & 1+x & 1 & 0 & 1 & 0 & 0 & 0\\
0 & x^{2} & 1 & 0 & \, x\, & 0 & 0 & 0 & 1\\
1 & 0 & 0 & 0 & 0 & 0 & 0 & 0 & 0\\
\,0\, & \,0\, & \, x\, & 1+x & 1 & 1 & 1 & 1 & 0\\
0 & 0 & x & 1 & 0 & 0 & 0 & 0 & 0\end{array}}\right]\]
 with $\bar{s}$-degrees $[-3,-1,-2,-2,-2,-2,-2,-2,~0~]$. In this
case the second column also has $\bar{s}$-degree exceeding $-\alpha=-2$,
and so it is guaranteed to produce another element of a $(\mathbf{F},8,\vec{s})$-basis.
Computing \[
\mathbf{E}\bar{\mathbf{P}}=\left[{\begin{array}{ccccccccc}
~0~ & ~0~ & ~0~ & ~0~ & ~0~ & ~0~ & ~0~ & ~0~ & ~1~\\
1 & 0 & 0 & x & 1+x & x & x & x & 0\\
1 & x^{2}+x^{3}+x^{4} & 1+x+x^{2}+x^{3} & x & 1+x & x & x & x & 1\\
x & x^{2} & 1+x^{3}+x^{4} & x^{2} & x+x^{2} & ~x^{2} & ~x^{2} & ~x^{2} & 1\end{array}}\right],\]
 we notice the second column is indeed an element of a $(\mathbf{F},8,\vec{s})$-basis. 
\end{exmp}

\subsection{Correspondence Between the Original Problem and the Transformed Problem}

We now work towards establishing the correspondence between the high
degree columns of a $\left(\bar{\mathbf{F}},\sigma,\bar{s}\right)$-basis
whose $\bar{s}$-degrees exceed $-\alpha$ and those of a $\left(\mathbf{F},\sigma,\vec{s}\right)$-basis
whose $\vec{s}$-degrees exceed $-\alpha$. A useful link is provided
by the following a matrix .

Set \[
\mathbf{A}_{i}=\begin{bmatrix}~~x^{r_{i}}\\
-1 & ~~x^{\beta}\\
 & -1 & \ddots\\
 &  & \ddots & ~~x^{\beta}\\
 &  &  & -1\end{bmatrix}_{q_{i}\times(q_{i}-1)}\mbox{ and \qquad}\mathbf{A}=\left[\begin{array}{ccc}
\mathbf{A}_{1}\\
 & \ddots\\
 &  & \mathbf{A}_{n}\end{array}\right]_{\bar{n}\times(\bar{n}-n)}.\]
 If $q_{i}=1$, $\mathbf{A}_{i}$ has dimension $1\times0$, which
just adds a zero row and no column in $\mathbf{A}$.

We now show that for any $\bar{\mathbf{w}}\in\left\langle \left(\bar{\mathbf{F}},\sigma,\bar{s}\right)\right\rangle $,
$\bar{\mathbf{w}}$ can be transformed by $\mathbf{A}$ to one of
the two forms that correspond to the original problem and transformed
problem. This is made more precise in the following lemma. We then
use unimodular equivalence of these two forms to show the equivalence
between the high degree part of the result from the transformed problem
and that of the original problem. 
\begin{lem}
Let \[
\bar{\mathbf{w}}=\begin{bmatrix}\bar{\mathbf{w}}_{1}\\
\vdots\\
\bar{\mathbf{w}}_{n}\end{bmatrix}\in\langle(\bar{\mathbf{F}},\sigma,\bar{s})\rangle\mbox{ with }\bar{\mathbf{w}}_{i}=\begin{bmatrix}\bar{w}_{i,0}\\
\vdots\\
\bar{w}_{i,q_{i}-1}\end{bmatrix}_{q_{i}\times1}.\]
 Then there exists a vector $\mathbf{u}\in\mathbb{K}\left[x\right]^{\left(\bar{n}-n\right)\times1}$
such that $\bar{\mathbf{w}}+\mathbf{A}\mathbf{u}$ has one of the
following two forms. 
\begin{description}
\item [{{{(a)}}}] The first form is \textup{\[
\mathbf{w}^{[1]}=\begin{bmatrix}\mathbf{w}_{1}^{[1]}\\
\vdots\\
\mathbf{w}_{n}^{[1]}\end{bmatrix}\mbox{ with }\mathbf{w}_{i}^{[1]}=\begin{bmatrix}w_{i}\\
0\\
\vdots\\
0\end{bmatrix}_{q_{i}\times1},\]
 } where \textup{$w_{i}=\bar{w}_{i,0}+\bar{w}_{i,1}x^{r_{i}}+\bar{w}_{i,2}x^{r_{i}+\beta}+\cdots+\bar{w}_{i,q_{i}-1}x^{r_{i}+(q_{i}-2)\beta}$.} 
\item [{{{(b)}}}] The second form is \[
\mathbf{w}^{[2]}=\begin{bmatrix}\mathbf{w}_{1}^{[2]}\\
\vdots\\
\mathbf{w}_{n}^{[2]}\end{bmatrix}\mbox{ with }\mathbf{w}_{i}^{[2]}=\begin{bmatrix}w_{i,0}\\
\vdots\\
w_{i,q_{i}-1}\end{bmatrix},\]
 where $\deg w_{i,j}<r_{i}\le\beta$ when $j=0$ and $\deg w_{i,j}<\beta$
when $j\in\{1,\dots,q_{i}-2\}$. There is no degree restriction on
$w_{i,q_{i}-1}$. 
\end{description}
\end{lem}
\begin{pf}
The first form is obtained by setting \[
\mathbf{u}^{[1]}=\begin{bmatrix}\mathbf{u}_{1}^{[1]}\\
\vdots\\
\mathbf{u}_{n}^{[1]}\end{bmatrix}\mbox{ with }\mathbf{u}_{i}^{[1]}=\left[\begin{array}{r}
\bar{w}_{i,1}+\bar{w}_{i,2}x^{\beta}+\bar{w}_{i,3}x^{2\beta}+\cdots+\bar{w}_{i,q_{i}-1}x^{(q_{i}-2)\beta}\\
\bar{w}_{i,2}+\bar{w}_{i,3}x^{\beta}+\cdots+\bar{w}_{i,q_{i}-1}x^{(q_{i}-3)\beta}\\
\vdots~~~~~\\
\bar{w}_{i,q_{i}-1}\end{array}\right].\]
 Then $\bar{\mathbf{w}}+\mathbf{A}\mathbf{u}^{[1]}$ gives the first
form. Note that $\mathbf{u}_{i}^{[1]}$ is empty if $q_{i}=1$ and
$\bar{\mathbf{w}}_{i}=\mathbf{w}_{i}^{[1]}=[\bar{w}_{i,0}]$ is not
changed by the transformation.

The second form can be obtained based on the first form. Let \[
t_{i,j}=\begin{cases}
r_{i} & \mbox{if }j=0\\
\beta & \mbox{if }j\in\{1,\dots,q_{i}-2\}\end{cases}\]
 and write $w_{i}$ from the first form as \begin{equation}
w_{i}=w_{i,0}+w_{i,1}x^{r_{i}}+w_{i,2}x^{r_{i}+\beta}+\cdots+w_{i,q_{i}-1}x^{r_{i}+(q_{i}-2)\beta}\label{eq:wiSeparatedForm}\end{equation}
 with $\deg w_{i,j}<t_{i,j}$ for $j<q_{i}-1$. Note that in general
$w_{i,j}\ne\bar{w}_{i,j}$, as $\deg\bar{w}_{i,j}$ may not be less
than $t_{i,j}$. Now set \[
\mathbf{v}=\begin{bmatrix}\mathbf{v}_{1}\\
\vdots\\
\mathbf{v}_{n}\end{bmatrix}\mbox{ with }\mathbf{v}_{i}=\left[\begin{array}{r}
w_{i,1}+w_{i,2}x^{\beta}+w_{i,3}x^{2\beta}+\cdots+w_{i,q_{i}-1}x^{(q_{i}-2)\beta}\\
w_{i,2}+w_{i,3}x^{\beta}+\cdots+w_{i,q_{i}-1}x^{(q_{i}-3)\beta}\\
\vdots~~~~~\\
w_{i,q_{i}-1}\end{array}\right]\]
 and $\mathbf{u}^{[2]}=\mathbf{u}^{[1]}-\mathbf{v}$, which comes
from the unimodular transformation \[
\left[\bar{\mathbf{w}},\mathbf{A}\right]\left[\begin{array}{c|c}
1\\
\hline \mathbf{\mathbf{u}}^{[1]} & \mathbf{I}\end{array}\right]\left[\begin{array}{c|c}
1\\
\hline -\mathbf{v} & \mathbf{I}\end{array}\right]=\left[\bar{\mathbf{w}},\mathbf{A}\right]\left[\begin{array}{c|c}
1\\
\hline \mathbf{u}^{[1]}-\mathbf{v} & \mathbf{I}\end{array}\right].\]
 Then $\mathbf{w}^{[2]}=\bar{\mathbf{w}}+\mathbf{A}\mathbf{u}^{[2]}$
is in the second form. Again note that $\mathbf{v}_{i}$ and $\mathbf{u}_{i}^{[2]}$
are empty if $q_{i}=1$ and $\mathbf{w}_{i}^{[2]}=\bar{\mathbf{w}}_{i}=[\bar{w}_{i,0}]$.\end{pf}
\begin{lem}
\label{lem:degreeCorrespondence}Let $\bar{\mathbf{w}}\in\left\langle \left(\bar{\mathbf{F}},\sigma,\bar{s}\right)\right\rangle $
and $\mathbf{w}^{[2]}$ be in the second form. If $\deg_{\vec{s}}\mathbf{E}\bar{\mathbf{w}}>-\alpha$
or $\deg_{\bar{s}}\mathbf{w}^{[2]}>-\alpha$, then $\deg_{\vec{s}}\mathbf{E}\bar{\mathbf{w}}=\deg_{\bar{s}}\mathbf{w}^{[2]}$. \end{lem}
\begin{pf}
Consider the $i$th entry $w_{i}$ of $\mathbf{E}\bar{\mathbf{w}}$
and the corresponding entries $\mathbf{w}_{i}^{[2]}=\left[w_{i,0},\dots,w_{i,q_{i}-1}\right]^{T}$
in $\mathbf{w}^{[2]}$. If $q_{i}=1$, then $w_{i}=w_{i,0}$ and the
corresponding shifts satisfies $s_{i}=\bar{s}_{\ell(i)}$, where $\ell(i)=\sum_{k=1}^{i}q_{k}$.
Hence $\deg w_{i}+s_{i}=\deg w_{i,0}+\bar{s}_{\ell(i)}$. Thus we
only need to consider the case where $q_{i}>1$. Write $w_{i}$ as
in Equation \prettyref{eq:wiSeparatedForm}. Note that $\deg w_{i,q_{i}-1}=\deg w_{i}-r_{i}-\beta\left(q_{i}-2\right)$
and hence $\deg w_{i,q_{i}-1}-\alpha-\beta=\deg w_{i}-r_{i}-\alpha-\beta\left(q_{i}-1\right)$,
that is, $\deg w_{i,q_{i}-1}+\bar{s}_{\ell(i)}=\deg w_{i}+s_{i}$.
It follows that \begin{eqnarray*}
\deg_{\vec{s}}\mathbf{E}\bar{\mathbf{w}} & = & \max_{i}(\deg w_{i}+s_{i})=\max_{i}(\deg w_{i,q_{i}-1}+\bar{s}_{\ell(i)})\\
 & \le & \max_{i,j}\left(\deg w_{i,j}+\bar{s}_{\ell(i-1)+j+1}\right)=\deg_{\bar{s}}\mathbf{w}^{[2]}.\end{eqnarray*}
 The only possible indices $j$ where the inequality can be strict
occur when $j<q_{i}-1$. But $\deg w_{i,j}<\beta$ for all $j<q_{i}-1$,
which implies $\deg w_{i,j}+\bar{s}_{\ell(i-1)+j+1}=\deg w_{i,j}-\alpha-\beta<-\alpha$,
and so it follows that the entries at these indices $j$ do not contribute
to $\deg_{\bar{s}}\mathbf{w}^{[2]}$ when $\deg_{\bar{s}}\mathbf{w}^{[2]}>-\alpha$
or $\deg_{\vec{s}}\mathbf{E}\bar{\mathbf{w}}=\max_{i}(\deg w_{i,q_{i}-1}+\bar{s}_{\ell(i)})>-\alpha$.
In other words, if one of them exceeds $-\alpha$, then $\deg_{\bar{s}}\mathbf{w}^{[2]}$
and $\deg_{\vec{s}}\mathbf{E}\bar{\mathbf{w}}$ are determined only
by entries at indices $j=q_{i}-1$, but the equality always holds
for these entries. \end{pf}
\begin{rem}
Notice that the first form $\mathbf{w}^{\left[1\right]}$ of $\bar{\mathbf{w}}$
has nonzero entries only at indices $I=[1,q_{1}+1,\dots,\sum_{k=1}^{n-1}q_{k}+1]$.
Let $\mathbf{B}$ be a $\bar{n}\times n$ matrix with $1$'s at position
$(\sum_{k=1}^{n-1}q_{k}+1,i)$ and 0's everywhere else. Then the first
form satisfies\textup{ $\mathbf{w}^{[1]}=\mathbf{B}\mathbf{E}\bar{\mathbf{w}}$.}
Hence \prettyref{lem:degreeCorrespondence} provides the degree correspondence
between the degrees of the first form $\mathbf{B}\mathbf{E}\bar{\mathbf{w}}$,
which is just $\mathbf{E}\bar{\mathbf{w}}$ with zero rows added,
and the second form $\bar{\mathbf{w}}^{[2]}$ of $\bar{\mathbf{w}}$.\end{rem}
\begin{cor}
\label{cor:degreeCorrespondence}Let $\bar{\mathbf{w}}\in\left\langle \left(\bar{\mathbf{F}},\sigma,\bar{s}\right)\right\rangle $
and $\mathbf{w}^{[2]}$ be its second form. Then $\deg_{\vec{s}}\mathbf{E}\bar{\mathbf{w}}>-\alpha$
if and only if $\deg_{\bar{s}}\mathbf{w}^{[2]}>-\alpha$.\end{cor}
\begin{pf}
The proof follows directly from \prettyref{lem:degreeCorrespondence}.\end{pf}
\begin{lem}
\label{lem:degEwLessEqDegw}Let $\bar{\mathbf{w}}\in\left\langle \left(\bar{\mathbf{F}},\sigma,\bar{s}\right)\right\rangle $.
Then $\deg_{\vec{s}}\mathbf{E}\bar{\mathbf{w}}\le\deg_{\bar{s}}\bar{\mathbf{w}}$. \end{lem}
\begin{pf}
As in \prettyref{lem:degreeCorrespondence}, consider the $i$th entry
$w_{i}$ of $\mathbf{E}\bar{\mathbf{w}}$ and the corresponding entries
$\bar{\mathbf{w}}_{i}=\left[\bar{w}_{i,0},\dots,\bar{w}_{i,q_{i}-1}\right]^{T}$
in $\bar{\mathbf{w}}$. If $q_{i}=1$, then $\deg w_{i}+s_{i}=\deg w_{i,0}+\bar{s}_{\ell(i)}$
as before. Thus we just need to consider the case $q_{i}>1$, where
the shifts for $\bar{\mathbf{w}}_{i}$ are $-\alpha-\beta$. Since
$w_{i}=\bar{w}_{i,0}+\bar{w}_{i,1}x^{r_{i}}+\bar{w}_{i,2}x^{r_{i}+\beta}+\cdots+\bar{w}_{i,q_{i}-1}x^{r_{i}+(q_{i}-2)\beta},$
we get \[
\deg w_{i}=\max\left\{ \deg\bar{w}_{i,0},\deg\bar{w}_{i,1}+r_{i},\deg\bar{w}_{i,2}+r_{i}+\beta,\dots,\deg\bar{w}_{i,q_{i}-2}+r_{i}+(q_{i}-2)\beta\right\} .\]
 Then \begin{eqnarray*}
\deg w_{i}+s_{i} & = & \deg w_{i}-r_{i}-\alpha-\beta(q_{i}-1)\\
 & = & \max\left\{ \deg\bar{w}_{i,0}-r_{i}-\alpha-\beta(q_{i}-1),~\deg\bar{w}_{i,1}-\alpha-\beta(q_{i}-1),~\dots,\right.\\
 &  & \left.~~~~~~~~~~~~~~\dots,\deg\bar{w}_{i,q_{i}-2}-\alpha-\beta\right\} \\
 & \le & \max\left\{ \deg\bar{w}_{i,0}-\alpha-\beta,\deg\bar{w}_{i,1}-\alpha-\beta,\dots,\deg\bar{w}_{i,q_{i}-2}-\alpha-\beta\right\} ,\end{eqnarray*}
 and so $\deg_{\vec{s}}\mathbf{E}\bar{\mathbf{w}}\le\deg_{\bar{s}}\bar{\mathbf{w}}$.\end{pf}
\begin{cor}
\label{cor:P2Degree}Let $\bar{\mathbf{P}}=[\bar{\mathbf{P}}_{1},\bar{\mathbf{P}}_{2}]$
be a $\left(\bar{\mathbf{F}},\sigma,\bar{s}\right)$-basis, where
$\deg_{\bar{s}}\bar{\mathbf{P}}_{1}\le-\alpha$ and $\deg_{\bar{s}}\bar{\mathbf{P}}_{2}>-\alpha$.
Let $\bar{\mathbf{P}}_{2}^{[2]}$ be the second form of $\bar{\mathbf{P}}_{2}$.
Then $\deg_{\bar{s}}\bar{\mathbf{P}}_{2}=\deg_{\bar{s}}\bar{\mathbf{P}}_{2}^{[2]}=\deg_{\vec{s}}\mathbf{E}\bar{\mathbf{P}}_{2}$.
Hence $[\bar{\mathbf{P}}_{1},\bar{\mathbf{P}}_{2}^{[2]}]$ is also
a $(\bar{\mathbf{F}},\sigma,\bar{s})$-basis.\end{cor}
\begin{pf}
Since any column $\bar{\mathbf{p}}$ of $\bar{\mathbf{P}}_{2}$ satisfies
$\deg_{\bar{s}}\bar{\mathbf{p}}>-\alpha,$ from \prettyref{lem:degreeCorrespondence}
and \prettyref{lem:degEwLessEqDegw}, we get \[
\deg_{\bar{s}}\bar{\mathbf{p}}^{[2]}=\deg_{\vec{s}}\mathbf{E}\bar{\mathbf{p}}\le\deg_{\bar{s}}\bar{\mathbf{p}}.\]
 The inequality is in fact an equality, since otherwise, $\bar{\mathbf{p}}$
in $\bar{\mathbf{P}}$ can be replaced by $\bar{\mathbf{p}}^{[2]}$
to get a basis of lower degree, contradicting the minimality of $\bar{\mathbf{P}}$.
Note that $\bar{\mathbf{P}}$ with its column $\bar{\mathbf{p}}$
replaced by $\bar{\mathbf{p}}^{[2]}$ remains to be a $\left(\bar{\mathbf{F}},\sigma,\bar{s}\right)$-basis,
since $\bar{\mathbf{p}}^{[2]}=\bar{\mathbf{p}}+\mathbf{A}\mathbf{u}$
involves column operations with only columns in $\bar{\mathbf{P}}_{1}$
as $\mathbf{A}$ has $\bar{s}$-degrees bounded by $-\alpha$ and
hence is generated by $\bar{\mathbf{P}}_{1}$.\end{pf}
\begin{lem}
\label{lem:PtoBPA}If $\mathbf{P}$ is a $\left(\mathbf{F},\sigma,\vec{s}\right)$-basis,
then $\left[\mathbf{B}\mathbf{P},\mathbf{A}\right]$ is a basis for
$\left\langle \left(\bar{\mathbf{F}},\sigma,\bar{s}\right)\right\rangle $.\end{lem}
\begin{pf}
Any $\bar{\mathbf{w}}\in\left\langle \left(\bar{\mathbf{F}},\sigma,\bar{s}\right)\right\rangle $
can be transformed by $\mathbf{A}$ to the first form \[
\mathbf{w}^{[1]}=\bar{\mathbf{w}}+\mathbf{A}\mathbf{u}^{[1]}=\mathbf{B}\mathbf{E}\bar{\mathbf{w}},\]
 where $\mathbf{E}\bar{\mathbf{w}}\in\left\langle \left(\mathbf{F},\sigma,\vec{s}\right)\right\rangle $
is generated by $\mathbf{P}$. That is, \[
\bar{\mathbf{w}}=\mathbf{w}^{[1]}-\mathbf{A}\mathbf{u}^{[1]}=\mathbf{B}\mathbf{E}\bar{\mathbf{w}}-\mathbf{A}\mathbf{u}^{[1]}=\mathbf{B}\mathbf{P}\mathbf{v}-\mathbf{A}\mathbf{u}^{[1]}=\left[\mathbf{B}\mathbf{P},\mathbf{A}\right][\mathbf{v},-\mathbf{u}^{[1]}]^{T}.\]
 One can also see that the columns of $\mathbf{A}$ and the columns
of $\mathbf{B}\mathbf{P}$ are linearly independent, as each zero
row of $\mathbf{B}\mathbf{P}$ has a $-1$ from a column of $\mathbf{A}$. 
\end{pf}
%
\begin{comment}
\begin{lem}
If $\mathbf{P}_{1}$ is a $\left(\mathbf{F},\sigma,\vec{s}\right)_{-\alpha}$-basis
, then $\left[\mathbf{B}\mathbf{P}_{1},\mathbf{A}\right]$ is a basis
for $\left\langle \left(\bar{\mathbf{F}},\sigma,\bar{s}\right)\right\rangle _{-\alpha}$.\end{lem}
\begin{pf}
We know that if $\bar{\mathbf{w}}$ has order $\left(\bar{\mathbf{F}},\sigma\right)$
then $\mathbf{E}\bar{\mathbf{w}}$ has order $\left(\vec{\mathbf{F}},\sigma\right)$.
Also if $\deg_{\bar{s}}\bar{\mathbf{w}}\le-\alpha$, then $\deg_{\vec{s}}\mathbf{E}\bar{\mathbf{w}}\le-\alpha$
by \prettyref{lem:degEwLessEqDegw}. Therefore, if $\bar{\mathbf{w}}\in\left\langle \left(\bar{\mathbf{F}},\sigma,\bar{s}\right)\right\rangle _{-\alpha}$
, then $\mathbf{E}\bar{\mathbf{w}}\in\left\langle \left(\mathbf{F},\sigma,\vec{s}\right)\right\rangle _{-\alpha}$.
Now apply the same procedure as in \prettyref{lem:PtoBPA}, we get
$\bar{\mathbf{w}}=\mathbf{w}^{[1]}-\mathbf{A}\mathbf{u}^{[1]}=\mathbf{B}\mathbf{E}\bar{\mathbf{w}}-\mathbf{A}\mathbf{u}^{[1]}=\mathbf{B}\mathbf{P}_{1}\mathbf{v}-\mathbf{A}\mathbf{u}^{[1]}=\left[\mathbf{B}\mathbf{P}_{1},\mathbf{A}\right][\mathbf{v},-\mathbf{u}^{[1]}]^{T}.$
Also as before, the columns of $\mathbf{A}$ and the columns of $\mathbf{B}\mathbf{P}_{1}$
are linearly independent. 
\end{pf}

\end{comment}
{} 
\begin{lem}
\label{lem:EPgeneration}If $\bar{\mathbf{P}}$ is a $\left(\bar{\mathbf{F}},\sigma,\bar{s}\right)$-basis,
then $\mathbf{E}\bar{\mathbf{P}}$ generates $\left\langle \left(\mathbf{F},\sigma,\vec{s}\right)\right\rangle $.
That is, for any $\mathbf{w}\in\left\langle \left(\mathbf{F},\sigma,\vec{s}\right)\right\rangle $,
there is an $\mathbf{u}\in\mathbb{K}\left[x\right]^{\bar{n}\times1}$
such that $\mathbf{w}=\mathbf{E}\bar{\mathbf{P}}\mathbf{u}$.\end{lem}
\begin{pf}
For any $\left(\mathbf{F},\sigma,\vec{s}\right)$-basis $\mathbf{P}$,
the columns of $\mathbf{B}\mathbf{P}$ are in $\langle(\bar{\mathbf{F}},\sigma,\bar{s})\rangle$
generated by $\bar{\mathbf{P}}$, that is, $\mathbf{B}\mathbf{P}=\bar{\mathbf{P}}\mathbf{U}$
for some $\mathbf{U}\in\mathbb{K}[x]^{\bar{n}\times n}$. Hence $\mathbf{E}\mathbf{B}\mathbf{P}=\mathbf{P}$
is generated by $\mathbf{E}\bar{\mathbf{P}}$. That is, $\mathbf{P}=\mathbf{E}\bar{\mathbf{P}}\mathbf{U}$.
Then any $\mathbf{w}\in\left\langle \left(\mathbf{F},\sigma,\vec{s}\right)\right\rangle $,
which satisfies $\mathbf{w}=\mathbf{P}\mathbf{v}$ for some $\mathbf{v}\in\mathbb{K}[x]^{n\times1}$,
satisfies $\mathbf{w}=\mathbf{E}\bar{\mathbf{P}}\mathbf{U}\mathbf{v}$. 
\end{pf}
%
\begin{comment}
\begin{lem}
\label{lem:EP1generation}If $\bar{\mathbf{P}}_{1}$ is a $\left(\bar{\mathbf{F}},\sigma,\bar{s}\right)_{-\alpha}$-basis,
then $\mathbf{E}\bar{\mathbf{P}}_{1}$ generates $\left\langle \left(\mathbf{F},\sigma,\vec{s}\right)\right\rangle _{-\alpha}$. \end{lem}
\begin{pf}
Suppose $\mathbf{P}_{1}$ is a $\left(\mathbf{F},\sigma,\vec{s}\right)_{-\alpha}$-basis,
then the columns of $\mathbf{B}\mathbf{P}_{1}$ can be transformed
to the second form with $\bar{s}$-degrees less than or equal to$-\alpha$
by \prettyref{cor:degreeCorrespondence}, hence generated by $\bar{\mathbf{P}}_{1}$.
That is, $\mathbf{B}\mathbf{P}_{1}+\mathbf{A}\mathbf{U}$ is generated
by $\bar{\mathbf{P}}_{1}$. Therefore, $\mathbf{E}(\mathbf{B}\mathbf{P}_{1}+\mathbf{A}\mathbf{U})=\mathbf{E}\mathbf{B}\mathbf{P}_{1}=\mathbf{P}_{1}$
is generated by $\mathbf{E}\bar{\mathbf{P}}_{1}$. 
\end{pf}

\end{comment}
{}

We are now ready to prove the main result on the correspondence between
a high degree part of a basis of the transformed problem and that
of the original problem. 
\begin{thm}
\label{thm:correctHighDegreeElements}Let $\bar{\mathbf{P}}=[\bar{\mathbf{P}}_{1},\bar{\mathbf{P}}_{2}]$
be a $\left(\bar{\mathbf{F}},\sigma,\bar{s}\right)$-basis, where
$\deg_{\bar{s}}\bar{\mathbf{P}}_{1}\le-\alpha$ and $\deg_{\bar{s}}\bar{\mathbf{P}}_{2}>-\alpha$.
Then $\mathbf{E}\bar{\mathbf{P}}_{2}$ is the matrix of the columns
of a $\left(\mathbf{F},\sigma,\vec{s}\right)$-basis whose $\vec{s}$-degrees
exceed $-\alpha$. \end{thm}
\begin{pf}
We want to show that $[\mathbf{P}_{1},\mathbf{E}\bar{\mathbf{P}}_{2}]$
is a $\left(\mathbf{F},\sigma,\vec{s}\right)$-basis for any $\left(\mathbf{F},\sigma,\vec{s}\right)_{-\alpha}$-basis
$\mathbf{P}_{1}$. First, $\mathbf{E}\bar{\mathbf{P}}$ has order
$(\mathbf{F},\sigma)$ since $\bar{\mathbf{F}}\bar{\mathbf{P}}=\mathbf{F}\mathbf{E}\bar{\mathbf{P}}$
and $\bar{\mathbf{P}}$ has order $\left(\bar{\mathbf{F}},\sigma\right)$.
Also, since $\mathbf{E}\bar{\mathbf{P}}$ generates $\left\langle \left(\mathbf{F},\sigma,\vec{s}\right)\right\rangle $
by \prettyref{lem:EPgeneration}, and from \prettyref{cor:degreeCorrespondence}
$\mathbf{E}\bar{\mathbf{P}}_{1}$ has $\vec{s}$-degree bounded by
$-\alpha$ hence is generated by $\mathbf{P}_{1}$, it follows that
$\left[\mathbf{P}_{1},\mathbf{E}\bar{\mathbf{P}}_{2}\right]$ generates
$\left\langle \left(\mathbf{F},\sigma,\vec{s}\right)\right\rangle $.

It only remains to show that the $\vec{s}$-degrees of $\mathbf{E}\bar{\mathbf{P}}_{2}$
are minimal. Suppose not, then $[\mathbf{P}_{1},\mathbf{E}\bar{\mathbf{P}}_{2}]$
can be reduced to $[\mathbf{P}_{1},\tilde{\mathbf{P}}_{2}]$ where
$\tilde{\mathbf{P}}_{2}$ has a column having lower $\vec{s}$-degree
than that of the corresponding column in $\mathbf{E}\bar{\mathbf{P}}_{2}$.
That is, assuming the columns of $\tilde{\mathbf{P}}_{2}$ and $\mathbf{E}\bar{\mathbf{P}}_{2}$
are in non-decreasing $\vec{s}$-degrees order, then we can find the
first index $i$ where the $\vec{s}$-degree of $i$th column of $\tilde{\mathbf{P}}_{2}$
is lower than the $\vec{s}$-degree of the $i$th column of $\mathbf{E}\bar{\mathbf{P}}_{2}$.
It follows that $[\mathbf{B}\mathbf{P}_{1},\mathbf{B}\mathbf{E}\bar{\mathbf{P}}_{2}]$
can be reduced to $[\mathbf{B}\mathbf{P}_{1},\mathbf{B}\tilde{\mathbf{P}}_{2}]$
and $[\mathbf{B}\mathbf{P}_{1},\mathbf{B}\mathbf{E}\bar{\mathbf{P}}_{2},\mathbf{A}]$
can be reduced to $[\mathbf{B}\mathbf{P}_{1},\mathbf{B}\tilde{\mathbf{P}}_{2},\mathbf{A}]$.
Since $[\mathbf{B}\mathbf{P}_{1},\mathbf{B}\tilde{\mathbf{P}}_{2},\mathbf{A}]$
generates $\langle(\bar{\mathbf{F}},\sigma,\bar{s})\rangle$ by \prettyref{lem:PtoBPA},
it can be reduced to $\bar{\mathbf{P}}=[\bar{\mathbf{P}}_{1},\bar{\mathbf{P}}_{2}]$.
But it can also be reduced to $[\bar{\mathbf{P}}_{1},\tilde{\mathbf{P}}_{2}^{[2]},\mathbf{A}]$
with $\tilde{\mathbf{P}}_{2}^{[2]}$ the second form of $\mathbf{B}\tilde{\mathbf{P}}_{2}$,
and to $[\bar{\mathbf{P}}_{1},\tilde{\mathbf{P}}_{2}^{[2]}]$ as the
columns of $\mathbf{A}$ are generated by the $\left(\bar{\mathbf{F}},\sigma,\bar{s}\right)_{-\alpha}$-basis
$\bar{\mathbf{P}}_{1}$.

In order to reach a contradiction we just need to show that $\tilde{\mathbf{P}}_{2}^{[2]}$
has a column with $\bar{s}$-degree less than that of the corresponding
column in $\bar{\mathbf{P}}_{2}$. Let $\tilde{\mathbf{w}}$ be the
first column of $\tilde{\mathbf{P}}_{2}$ with $\vec{s}$-degree less
than that of the corresponding column $\mathbf{w}$ in $\mathbf{E}\bar{\mathbf{P}}_{2}$
and let $\bar{\mathbf{w}}$ be the corresponding column in $\bar{\mathbf{P}}_{2}$.
By \prettyref{cor:P2Degree} $\deg_{\vec{s}}\mathbf{w}=\deg_{\bar{s}}\bar{\mathbf{w}}$.
Let $\tilde{\mathbf{w}}^{[2]}$ be the second form of $\mathbf{B}\tilde{\mathbf{w}}$,
which is a column in $\tilde{\mathbf{P}}_{2}^{[2]}$ corresponding
to the column $\bar{\mathbf{w}}$ in $\bar{\mathbf{P}}_{2}$. We know
that either $\deg_{\bar{s}}\tilde{\mathbf{w}}^{[2]}\le-\alpha$ or
$\deg_{\bar{s}}\tilde{\mathbf{w}}^{[2]}=\deg_{\vec{s}}\tilde{\mathbf{w}}$
by \prettyref{lem:degreeCorrespondence}, as $\mathbf{E}\tilde{\mathbf{w}}^{[2]}=\mathbf{E}(\mathbf{B}\tilde{\mathbf{w}}+\mathbf{A}\mathbf{u})=\tilde{\mathbf{w}}$.
In either case, $\deg_{\bar{s}}\tilde{\mathbf{w}}^{[2]}<\deg_{\bar{s}}\bar{\mathbf{w}}$,
as $\deg_{\bar{s}}\bar{\mathbf{w}}$ is greater than both $-\alpha$
and $\deg_{\vec{s}}\tilde{\mathbf{w}}$. Hence we have $[\bar{\mathbf{P}}_{1},\tilde{\mathbf{P}}_{2}^{[2]}]$
is another $\left(\bar{\mathbf{F}},\sigma,\bar{s}\right)$-basis with
lower $\bar{s}$-degrees than $\mathbf{\bar{P}}$, contradicting with
the minimality of $\bar{\mathbf{P}}$. 
\end{pf}

\subsection{Achieving Efficient Computation}

\prettyref{thm:correctHighDegreeElements} essentially tells us that
a high degree part of a $\left(\mathbf{F},\sigma,\vec{s}\right)$-basis
can be determined by computing a $\left(\bar{\mathbf{F}},\sigma,\bar{s}\right)$-basis,
something we know can be done efficiently. Notice the parallel between
the situation here and in the earlier balanced shift case, where the
transformed problem also allows us to compute a partial $\left(\mathbf{F},\sigma,\vec{s}\right)$-basis,
albeit a low degree part, in each iteration.

After a $\left(\bar{\mathbf{F}},\sigma,\bar{s}\right)$-basis, or
equivalently a high degree part of a $\left(\mathbf{F},\sigma,\vec{s}\right)$-basis,
is computed, for the remaining problem of computing the remaining
basis elements, we can in fact reduce the dimension of the input $\mathbf{F}$
by removing some of its columns corresponding to the high shift entries. 
\begin{thm}
\label{thm:zeroHighShiftEntries}Suppose without loss of generality
that the entries of $\vec{s}$ are in non-decreasing order. Let $I$
be the index set containing the indices of entries $s_{i}$ in $\vec{s}$
such that $s_{i}\le-\alpha$. Let $\mathbf{F}_{I}$ be the columns
of $\mathbf{F}$ indexed by $I$. Then a $\left(\mathbf{F}_{I},\sigma,\vec{s}\right)_{-\alpha}$-basis
\textbf{$\mathbf{P}_{1}$} gives a $\left(\mathbf{F},\sigma,\vec{s}\right)_{-\alpha}$-basis
$\left[\mathbf{P}_{1}^{T},\mathbf{0}\right]^{T}$. 
\end{thm}
%The theorem follows from the following claim.
%\end{pf}
%\begin{claim}
%\label{cla:zeroHighShiftEntries}

\begin{pf}
For any $\mathbf{p}\in\mathbb{K}\left[x\right]^{n\times1}$ and $\deg_{\vec{s}}\mathbf{p}\le-\alpha$,
note that if the $i$th entry of the shift satisfies $s_{i}\le-\alpha$,
then the corresponding entry $p_{i}$ of $\mathbf{p}$ is zero. %\end{claim}
%\begin{pf}
Otherwise, if $p_{i}\ne0$ then the $\vec{s}$-degree of $\mathbf{p}$
is at least $s_{i}>-\alpha$, contradicting the assumption that the
$\vec{s}$-degree of $\mathbf{p}$ is lower than or equal to $-\alpha$. 
\end{pf}
Thus, these zero entries do not need to be considered in the remaining
problem of computing a $\left(\mathbf{F},\sigma,\vec{s}\right)_{-\alpha}$-basis.
As such the corresponding columns from the input matrix $\mathbf{F}$
can be removed. 
\begin{exmp}
Let us return to \prettyref{exm:unbalancedShift}. When the parameters
$\alpha=\beta=1$, after computing an element of a $(\mathbf{F},8,\vec{s})$-basis
with $\vec{s}$-degree $0$ that exceeds $-\alpha=-1$, the first
row of any $\left(\mathbf{F},\sigma,\vec{s}\right)_{-1}$-basis must
be zero by \prettyref{thm:zeroHighShiftEntries} (since the first
entry of $\vec{s}=[0,-3,-5,-6]$ is $0>-\alpha$). This is illustrated
by the $(\mathbf{F},8,\vec{s})$-basis $\mathbf{P}$ given in \prettyref{exm:unbalancedShift}.
This implies that the first column of $\mathbf{F}$ is not needed
in the subsequent computation of the remaining basis elements.\end{exmp}
\begin{cor}
\label{cor:numberBasisElements}If the shift $\vec{s}$ satisfies
condition \eqref{con:unbalancedCondition} and $c$ is a constant
greater than or equal to $1$, then a $\left(\mathbf{F},\sigma,\vec{s}\right)_{-cd}$-basis
has at most $n/c$ basis elements.\end{cor}
\begin{pf}
Since $d=m\sigma/n\ge-\sum_{i=1}^{n}s_{i}/n$ under condition \eqref{con:unbalancedCondition},
there cannot be more than $n/c$ entries of $\vec{s}$ less than or
equal to $-cd$. By \prettyref{thm:zeroHighShiftEntries}, the only
possible nonzero rows of a $\left(\mathbf{F},\sigma,\vec{s}\right)_{-cd}$-basis
are the ones corresponding to (with the same indices as) the shift
entries that are less than or equal to $-cd$. Hence there cannot
be more than $n/c$ nonzero rows and at most $n/c$ columns, as the
columns are linearly independent. 
\end{pf}
We now have a situation similar to that found in the balanced shift
case. Namely, for each iteration we transform the problem using appropriate
parameters $\alpha$ and $\beta$ to efficiently compute the basis
elements with degrees greater than $-\alpha$. Then we can remove
columns from the input matrix $\mathbf{F}$ corresponding to the shift
entries that are greater than $-\alpha$. We can then repeat the same
process again, with a larger $\alpha$ and $\beta$, in order to compute
more basis elements. 
\begin{thm}
If the shift $\vec{s}$ satisfies condition \eqref{con:unbalancedCondition},
then a $\left(\mathbf{F},\sigma,\vec{s}\right)$-basis can be computed
with cost $O^{\sim}\left(n^{\omega-1}m\sigma\right)=O^{\sim}\left(n^{\omega}d\right)$. \end{thm}
\begin{pf}
We give the following constructive proof. Initially, we set transformation
parameters $\alpha_{1}=\beta_{1}=2d$ with $d=m\sigma/n\ge-\sum_{i=1}^{n}s_{i}/n$.
\prettyref{alg:mab} works efficiently on the transformed problem
as the shift $\bar{s}^{\left(1\right)}$ is balanced and the dimension
of $\bar{\mathbf{F}}_{1}$ remains $O\left(n\right)$. By \prettyref{thm:correctHighDegreeElements}
this gives the basis elements of $\left(\mathbf{F},\sigma,\vec{s}\right)$-basis
with $\vec{s}$-degree exceeding $-\alpha_{1}=-2d$. By \prettyref{cor:numberBasisElements},
the number of basis elements remaining to be computed is at most $n/2$,
hence the number of elements correctly computed is at least $n/2$.
By \prettyref{thm:zeroHighShiftEntries}, this also allows us to remove
at least half of the columns from the input $\mathbf{F}$ and correspondingly
at least half of the rows from the output for the remaining problem.
Thus the new input matrix $\mathbf{F}_{2}$ has a new column dimension
$n_{2}\le n/2$ and the corresponding shift $\vec{s}^{\left(2\right)}$
has $n_{2}$ entries. The average degree of the new problem is $d_{2}~=~m\sigma/n_{2}$.

For the second iteration, we set $\alpha_{2}$ and $\beta_{2}$ to
$2d_{2}$. Since \[
\alpha_{2}=2m\sigma/n_{2}\ge-2\sum_{i=1}^{n}s_{i}/n_{2}\ge-2\sum_{i=1}^{n_{2}}s_{i}^{\left(2\right)}/n_{2},\]
 this allows us to reduce the dimension $n_{3}$ of $\mathbf{F}_{3}$
to at most $n_{2}/2$ after finishing computing a $\left(\bar{\mathbf{F}}_{2},\sigma,\bar{s}^{\left(2\right)}\right)_{-\alpha_{1}}$-basis.
Again, this can be done using \prettyref{alg:mab} with a cost of
$O^{\sim}\left(n_{2}^{\omega}d_{2}\right)=O^{\sim}\left(n^{\omega}d\right)$
as the shift $\bar{a}_{2}$ is balanced and the dimension of $\bar{\mathbf{F}}_{2}$
is $O\left(n_{2}\right)$. Repeating this process, at iteration $i$,
we set $\alpha_{i}=\beta_{i}=2d_{i}=2m\sigma/n_{i}$. The transformed
problem has a balanced shift $\bar{a}_{i}$ and column dimension $O\left(n_{i}\right)$.
So a $\left(\bar{\mathbf{F}}_{i},\sigma,\bar{s}^{\left(i\right)}\right)_{-\alpha_{i-1}}$-basis
can be computed with a cost of \[
O^{\sim}\left(n_{i}^{\omega}d_{i}\right)=O^{\sim}\left(n_{i}^{\omega-1}m\sigma\right)\subset O^{\sim}\left(n^{\omega-1}m\sigma\right)=O^{\sim}\left(n^{\omega}d\right).\]
 Since \[
\alpha_{i}=2m\sigma/n_{i}\ge-2\sum_{i=1}^{n}s_{i}/n_{i}\ge-2\sum_{i=1}^{n_{i}}s_{i}^{\left(i\right)}/n_{i},\]
 the column dimension $n_{i+1}$ of the next problem can again be
reduced by a half. After iteration $i$, at most $n/2^{i}$ $\left(\mathbf{F},\sigma,\vec{s}\right)$-basis
elements remain to be computed. We can stop this process when the
column dimension $n_{i}$ of the input matrix $\mathbf{F}_{i}$ reaches
the row dimension $m$, as an order basis can be efficiently computed
in such case. Therefore, a complete $\left(\mathbf{F},\sigma,\vec{s}\right)$-basis
can be computed in at most $\log(n/m)$ iterations, so the overall
cost is $O^{\sim}\left(n^{\omega}d\right)$. 
\end{pf}
%
\begin{algorithm}
\caption{$\umab\left(\mathbf{F},\sigma,\vec{s}\right)$ }


\label{alg:umab} 
\begin{algor}
\item [{{{*}}}] Input: $\mathbf{F}\in K\left[x\right]^{m\times n}$,
$\sigma\in\mathbb{Z}_{\ge0}$, $\vec{s}$ satisfies condition \eqref{con:unbalancedCondition}.
\item [{{{*}}}] Output: $\mathbf{P}\in K\left[x\right]^{n\times n}$,
an $\left(\mathbf{F},\sigma,\vec{s}\right)$-basis.
\item [{{{*}}}] Uses:
\item [{{*}}] (a) $\TransformUnbalanced$ : converts an unbalanced shift
problem to a balanced one using the transformation described in \prettyref{sec:Unbalanced-Shift}.
Returns transformed input matrix, transformed shift, and transformation
matrix.
\item [{{*}}] (b) %{\em OrderBasis} 
$\mmab$ : computes order basis with balanced shift. 
\item [{{*}}]~\end{algor}
\begin{algor}[1]
\item [{{{*}}}]  $i:=1;$ $\mathbf{P}=[\,]$;
\item [{{{*}}}] $\mathbf{F}^{\left(i\right)}:=\mathbf{F}$, $\vec{s}^{\left(i\right)}:=\vec{s}$;
\item [{{while}}] $\columnDimension(\mathbf{P})\ne n$
\item [{{{*}}}] $d_{i}=\left\lceil m\sigma/\columnDimension(\mathbf{F}^{\left(i\right)})\right\rceil $;
\item [{{{*}}}] $\alpha_{i}:=\beta_{i}:=2d_{i};$
\item [{{{*}}}] $\left[\bar{\mathbf{F}}^{\left(i\right)},\bar{s}^{\left(i\right)},\mathbf{E}\right]:=\TransformUnbalanced\left(\mathbf{F}^{\left(i\right)},\vec{s}^{\left(i\right)},\alpha_{i},\beta_{i}\right)$;
\item [{{{*}}}] $\bar{\mathbf{P}}^{\left(i\right)}:=\mmab\left(\bar{\mathbf{F}}^{\left(i\right)},\sigma,\bar{s}^{\left(i\right)}\right)$;
\item [{{{*}}}] Set $\mathbf{P}^{\left(i\right)}$ to be the columns
of $\mathbf{E}\bar{\mathbf{P}}^{\left(i\right)}$ with $\bar{s}_{i}$-column
degrees in $(-\alpha_{i},-\alpha_{i-1}]$;
\item [{{{*}}}] $\mathbf{P}:=\left[\mathbf{P}^{\left(i\right)},\mathbf{P}\right]$;
\item [{{{*}}}] Set $I$ as the set of indices $i$ satisfying $s_{i}\le-\alpha_{i}$;
\item [{{{*}}}] $\mathbf{F}^{\left(i+1\right)}:=\mathbf{F}_{I}^{\left(i\right)}$,
$\vec{s}^{\left(i+1\right)}:=\vec{s}_{I}^{\left(i\right)}$;
\item [{{{*}}}] $i:=i+1$;
\item [{{endwhile}}] ~
\item [{{{*}}}] return $\mathbf{P}$ ; 
\end{algor}

\end{algorithm}

 Finally, we remark that when the condition
\eqref{con:unbalancedCondition} is relaxed to $\sum_{i=1}^{n}-s_{i}\in O\left(m\sigma\right)$,
so that $\sum_{i=1}^{n}-s_{i}\le cm\sigma$ for a constant $c$, we
can still compute a $\left(\mathbf{F},\sigma,\vec{s}\right)$-basis
with the same complexity, by setting $\alpha_{i}=\beta_{i}=2cm\sigma/n_{i}$
at each iteration $i$ and following the same procedure as above.
The cost at each iteration $i$ remains $O^{\sim}\left(n^{\omega}d\right)$,
and the entire computation still uses at most $\log(n/m)$ iterations. 


\input{conclusion.tex} 

\bibliographystyle{elsart-harv}
\bibliography{paper}

\end{document}
